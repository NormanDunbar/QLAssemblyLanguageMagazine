%%%%%%%%%%%%%%%%%%%%%%%%%%%%%%%%%%%%%%%%%
% The Legrand Orange Book
% LaTeX Template
% Version 2.0 (9/2/15)
%
% This template has been downloaded from:
% http://www.LaTeXTemplates.com
%
% Mathias Legrand (legrand.mathias@gmail.com) with modifications by:
% Vel (vel@latextemplates.com)
%
% License:
% CC BY-NC-SA 3.0 (http://creativecommons.org/licenses/by-nc-sa/3.0/)
%
% Compiling this template:
% This template uses biber for its bibliography and makeindex for its index.
% When you first open the template, compile it from the command line with the 
% commands below to make sure your LaTeX distribution is configured correctly:
%
% 1) pdflatex main
% 2) makeindex main.idx -s StyleInd.ist
% 3) biber main
% 4) pdflatex main x 2
%
% After this, when you wish to update the bibliography/index use the appropriate
% command above and make sure to compile with pdflatex several times 
% afterwards to propagate your changes to the document.
%
% This template also uses a number of packages which may need to be
% updated to the newest versions for the template to compile. It is strongly
% recommended you update your LaTeX distribution if you have any
% compilation errors.
%
% Important note:
% Chapter heading images should have a 2:1 width:height ratio,
% e.g. 920px width and 460px height.
%
%%%%%%%%%%%%%%%%%%%%%%%%%%%%%%%%%%%%%%%%%

%----------------------------------------------------------------------------------------
%	PACKAGES AND OTHER DOCUMENT CONFIGURATIONS
%----------------------------------------------------------------------------------------

\documentclass[11pt,fleqn]{book} % Default font size and left-justified equations


%----------------------------------------------------------------------------------------
\input{../Common/structure} % Insert the structure.tex file which contains the majority of the structure behind the template

%----------------------------------------------------------------------------------------
%	INDEXING and Other Macros. NDunbar.
%----------------------------------------------------------------------------------------

% These need to be used with the text twice, as in:
% The best text processing system is \latex\program{\latex} ...
\DeclareRobustCommand{\program}[1]{\index{Programs and applications!#1}}
\DeclareRobustCommand{\mc6800x}[1]{\index{MC6800x Instructions!#1}}
\DeclareRobustCommand{\address}[1]{\index{Addressing Modes!#1}}

% These generate the text in the main body in teletype mode as well as 
% adding them to the index:
% Use \vector(BP\_INIT} to link new functions to SuperBasic.
\DeclareRobustCommand{\vector}[1]{\texttt{#1}\index{Vectored Utilities!#1}}
\DeclareRobustCommand{\trap}[1]{\texttt{#1}\index{Trap Calls!#1}}
\DeclareRobustCommand{\pe}[1]{\texttt{#1}\index{Pointer Environment Vectors!#1}}

% These generate text in a certain style, see the actual command for details,
% when used. These do not get indexed.
\DeclareRobustCommand{\opcode}[1]{\texttt{#1}}
\DeclareRobustCommand{\register}[1]{\texttt{#1}}
\DeclareRobustCommand{\asmlabel}[1]{\texttt{#1}}


 % Insert my own macros.
%----------------------------------------------------------------------------------------
% NDunbar: Put your PDF details here ....
%
% These will be copied to the title page, where required, and also into the
% PDF file's properties.
%----------------------------------------------------------------------------------------
% Magazine cover - main title
\DeclareRobustCommand{\eMagTitle}{QL Assembly Language Mailing List}

% Magazine cover - Issue number
\DeclareRobustCommand{\eMagIssue}{5}

% Magazine info - download URL, and various other uses.
\DeclareRobustCommand{\eMagLeadingZeros}{00} % Used to keep things with three digits.

% Magazine cover - main Author
\DeclareRobustCommand{\eMagAuthor}{Norman Dunbar}

% generated PDF File's metadata.
\hypersetup{
	pdftitle={\eMagTitle~-~\eMagLeadingZeros\eMagIssue},
	pdfauthor={\eMagAuthor},
	pdfsubject={Assembly Language Programming on your Sinclair QL}
}

 % Insert the config for this particular issue.

\begin{document}

%----------------------------------------------------------------------------------------
%	TITLE PAGE
%----------------------------------------------------------------------------------------

\begingroup
\thispagestyle{empty}
\begin{tikzpicture}[remember picture,overlay]
\coordinate [below=12cm] (midpoint) at (current page.north);
\node at (current page.north west)
{\begin{tikzpicture}[remember picture,overlay]
\node[anchor=north west,inner sep=0pt] at (0,0) {\includegraphics[width=\paperwidth]{CoverImage}}; % Background image
\draw[anchor=north] (midpoint) node [fill=white!30!white,fill opacity=0.6,text opacity=1,inner sep=1cm]{\Huge\centering\bfseries\sffamily\parbox[c][][t]{\paperwidth}{\centering \eMagTitle\\[15pt] % Book title
{\Large Issue~\eMagIssue}\\[20pt] % Subtitle
{\huge \eMagAuthor}}}; % Author name
\end{tikzpicture}};
\end{tikzpicture}
\vfill
\endgroup

%----------------------------------------------------------------------------------------
%	COPYRIGHT PAGE
%----------------------------------------------------------------------------------------

\newpage
~\vfill
\thispagestyle{empty}

\noindent \textsc{Published by MeMyselfEye Publishing ;-)}\\ % Publisher

\noindent
\textbf{Download from:}\\
\url{https://github.com/NormanDunbar/QLAssemblyLanguageMagazine/blob/Issue\_007/Issue\_007/Assembly\_Language\_007.pdf}\\ % Download URL

\textbf{Licence:}\\
\noindent Licensed under the Creative Commons Attribution-NonCommercial 3.0 Unported License (the ``License''). You may not use this file except in compliance with the License. You may obtain a copy of the License at \url{http://creativecommons.org/licenses/by-nc/3.0}. Unless required by applicable law or agreed to in writing, software distributed under the License is distributed on an \textsc{``as is'' basis, without warranties or conditions of any kind}, either express or implied. See the License for the specific language governing permissions and limitations under the License.\\ % License information

%\noindent \textit{First printing, October 2015} \\% Printing/edition date

%\noindent This pdf document was created on \textit{\pdfcreationdate}.

\noindent This pdf document was created on \textit{\the\day/\the\month/\the\year} at \textit{\DTMcurrenttime}.

\noindent Copyright \copyright 2019 Norman Dunbar\\ % Copyright notice

%----------------------------------------------------------------------------------------
%	TABLE OF CONTENTS
%----------------------------------------------------------------------------------------

\chapterimage{ChapterImage} % Table of contents heading image

\pagestyle{empty} % No headers


\tableofcontents % Print the table of contents itself
%\listoftables
%\listoffigures
\lstlistoflistings

\cleardoublepage % Forces the first chapter to start on an odd page so it's on the right

\pagestyle{fancy} % Print headers again

%----------------------------------------------------------------------------------------
% Here come the contents of the book...
%----------------------------------------------------------------------------------------

%----------------------------------------------------------------------------------------
%Here come the contents of the book...
%----------------------------------------------------------------------------------------

%Pull in the Chapter - Preface.
\chapter{Preface}
\section{Feedback}\label{section: feedback}
Please send all feedback to \url{assembly@dunbar-it.co.uk}. You may also send articles to
this address, however, please note that anything sent to this email address may be used in a future issue of the eMagazine. Please mark your email clearly if you do not wish this to happen.

This eMagazine is created in \LaTeX  source format, aka plain text with a few formatting commands thrown in for good measure, so I can cope with almost any format you might want to send me. As long as I can get plain text out of it, I can convert it to a suitable source format with reasonable ease. 

I use a Linux system to generate this eMagazine so I can read most, if not all, Word or MS Office documents, Quill, Plain text, email etc formats. Text87 might be a problem though!

\section{Subscribing to The Mailing List}
This eMagazine is available by watching for updates on the ql-users email list, \url{ql-users@q-v-d.com} of by subscribing to the QL Forum at \url{https://qlforum.co.uk} and watching out for new messages from me about available issues.

\section{Contacting The Mailing List}
I'm rather hoping that this mailing list will not be a one-way affair, like QL Today appeared to be. I'm very open to suggestions, opinions, articles etc from my readers, otherwise how do I know what I'm doing is right or wrong?

Sadly, George Gwilt has passed away and will no longer continue to keep me correct on matters where I get stuff completely wrong, as before with \emph{QL Today}. I know George did ask, way back when I proposed this eMagazine, if there would be a contact address, so I've set up an email address for the list, so that you can make comments etc as you wish. The email address is:

\url{assembly@qdosmsq.dunbar-it.co.uk}

Any emails sent there will find me. Please note, anything sent to that email address will be considered for publication, so I would appreciate your name at the very least if you intend to send something. If you do not wish your email to be considered for publication, please mark it clearly as such, thanks. I look forward to hearing from you all, from time to time.

If you do have an article to contribute, I'll happily accept it in almost any format - email, text, Word, Libre/Open Office odt, Quill, PC Quill, etc etc. Ideally, a \LaTeX source document is the best format, because I
can simply include those directly, but I doubt I'll be getting many of those! But not to worry, if you have something, I'll hopefully manage to include it.


% And all the other chapters ...
\chapter{Bubble Sorts}
Part of a little program that I'm working on requires the characters of a word to be sorted into order, ascending in this case, and as there's no trap or vector in QDOSMSQ to allow this to be easily done, I've had to work out my own. The bubble sort is one of the simplest sorting algorithms that there is, however, it is pretty inefficient as much of the work it does is checking over data that it has already sorted in any previous pass. Also, the more data there are to sort, the longer it takes to sort. Much longer in fact.

Looking on Wikipedia for some slightly improved versions, I found the one below. It doesn't reduce the number of swaps that take place, but it does `know' that when it has made a pass through the array of bytes, in this case, the last item that it swapped is the lowest possible value for this pass, and anything from that point on in the array is already sorted. By `knowing' it does at least reduce the number of  comparisons that have to be made on each pass, which reduces the run time of the sort.

The data is sorted by moving  the higher values - in this version - down the array, one place at a time, until the array's bottom end contains all the sorted data, while the top end contains the data that are yet to be sorted. Hopefully, the following will make things a bit clearer, the pseudo code was obtained from Wikipedia.


\begin{lstlisting}[frame=none,numbers=none,caption={Bubblesort Algorithm}]
;--------------------------------------------------------------------
 Blatantly stolen from Wikipedia! 
 Very slightly modified by Norman Dunbar.

 An improved BubbleSort which `knows' that after each pass, the lowest 
 item(s) must be already sorted.

 For example:

 9 1 5 3 4, after pass 0, becomes:
 1 5 3 4 9 so we stop at `4' next time, not at `9'.
;--------------------------------------------------------------------
bubbleSort( A : list of sortable items )
    n = length(A)
    repeat
      newn = 0
      for i = 1 to n-1 
         Temp = A[i-1]
         if Temp > A[i] then
        A[i-1] = A[i]
        A[i] = Temp
        newn = i
         end if
      end for
      n = newn
    until n = 0
end procedure
;--------------------------------------------------------------------
\end{lstlisting}

From the above algorithm, we can see that a byte of data will be looked at and using comparisons and swaps, will `bubble' its way to the lower end of the array - that's the bit furthest from the word count in a QDOSMSQ string, for example.

An example is called for, we start with the test harness which sets up a tiny array of 4 upper case letters, with a leading word count, and sorts it. 


\begin{lstlisting}[firstnumber=1,caption={Bubblesort Test Harness}]
start
        lea stuff,a1          ; Where the data are
        bsr.s print_it        ; Print data to #1 unsorted
        bsr.s bubblesort      ; D0.L will be zero
        bsr.s print_it        ; Print sorted data to #1 
        rts

stuff   dc.w stuff_end-stuff-2
        dc.b 'C','A','D','B'
stuff_end   equ *
\end{lstlisting}

The code above needs to call a helper routine to print the before and after data, that code follows and is a slightly modified version of the code to find channel \#1 and print a string, from the last issue where we were printing the name list.

\begin{lstlisting}[firstnumber=last,caption={Bubblesort Test Harness}]
;--------------------------------------------------------------------
; Some hopefully familiar code from last issue, to print some data
; to channel #1 which MUST BE OPEN.
;--------------------------------------------------------------------
bv_chbas   equ $30            ; Offset to channel table.

;--------------------------------------------------------------------
; Find #1 in the channel table. We shouldn't be off the end of the 
; table, so NOT CHECKED.
; We assume #1 is open too, so that's NOT CHECKED for either.
;--------------------------------------------------------------------
print_it   
       move.l a1,-(a7)        ; A1 is in use, preserve it

findChan   
       moveq #40,d1           ; Offset to entry #1
       move.l bv_chbas(a6),a0 ; Channel table base offset
       adda.l d1,a0           ; Required entry for #1
       move.l 0(a6,a0.l),a0   ; A0 is ID of channel #1

;--------------------------------------------------------------------
; Print the text we read from the name list to channel #1.
; Corrupts D1-D3/A1. Preserves A0/A2-A3. D0 = error code.
;--------------------------------------------------------------------
printText
       move.w ut_mtext,a2     ; Vector to print a string
       jsr (a2)               ; Print it

;--------------------------------------------------------------------
; Print a linefeed to channel #1.
; Corrupts D1/A1. Preserves D2-D3/A0/A2-A3. D0 = error code.
;--------------------------------------------------------------------
linefeed
       moveq #io_sbyte,d0     ; Print a byte trap
       moveq #10,d1           ; Linefeed character
       moveq #-1,d3           ; Timeout
       trap #3                ; Do it
  
       move.l     (a7)+,a1    ; Retrieve A1
       rts
\end{lstlisting}


So far so simple, the following is my version of the pseudo code from Wikipedia, converted into assembly language. The labels are named in such a way as, hopefully, to give you an idea of where we are in the pseudo code as converted. Some bits don't convert exactly, the FOR loop, for example, starts with D2=0 and gets incremented by 1 before the loop, not at the end as per a normal FOR loop. But you get the idea, I hope!

The working registers are listed in the comments so that you can, if you wish, follow what's going on.

\begin{lstlisting}[firstnumber=last,caption={Bubblesort}]
;--------------------------------------------------------------------
; ENTRY:
;
; A1.L = Start address of bytes to be sorted. Word count first.
;
;--------------------------------------------------------------------
; WORKING:
;
; A1.L = Start Address of bytes to be sorted, word count first.
; A2.L = Bytes being compared right now. (-1(a2) and (a2)).
; D0.W = `n' = end of unsorted data.
; D1.B = Temp for swapping.
; D2.W = `i' = loop counter.
; D3.W = `newn' = last item sorted.
;--------------------------------------------------------------------
; EXIT:
;
; D0.L = 0.
; A1.L = Preserved - Start address of sorted bytes' word count.
; All other registers preserved.
;--------------------------------------------------------------------
bubblesort
        movem.l d1-d3/a1-a2,-(a7)
        move.w (a1)+,d0       ; N = length(a)
        beq.s bs_done
        subq.w #1,d0          ; We need n-1 when testing

bs_repeat   equ *             ; Repeat
        movea.l a1,a2         ;   A2 = First unsorted byte
        moveq #0,d3           ;   Newn = 0

bs_for_loop
        moveq #0,d2           ;   For i = 1 to n-1

bs_next
        addq.b #1,d2
        move.b (a2)+,d1       ;      Temp = A[i-1]
        cmp.b (a2),d1         ;      If Temp > A[i] then
        bls.s bs_end_if       ;       Skip swap if A[i-1] <= A[i]

bs_swap
        move.b (a2),-1(a2)    ;       A[i-1] = A[i]
        move.b d1,(a2)        ;       A[i] = Temp
        move.w d2,d3          ;       Newn = i

bs_end_if   equ *             ;     end if
        cmp.w d2,d0           ;     I = n-1 yet?
        bne.s bs_next         ;   End for
        move.w d3,d0          ;   N = newn
        tst.w d0              ;   N = 0 yet?

bs_until
        bne.s bs_repeat       ; Until n = 0

bs_done
        movem.l (a7)+,d1-d3/a1-a2
        clr.l d0
        rts
\end{lstlisting}

So, type the above into a file, save it, assemble it in the usual manner with Gwasl and then load it into a reserved area of memory (mine is 98 bytes long) and simply CALL it. You should see two lines of text on channel \#1. The second line being the sorted version of the first.

\subsection{Useful Improvements}

The above is fine for sorting the characters in a QDOSMSQ string, and that's the only sorting I actually \emph{need} for my current little project, however, with a couple of minor changes, we can make it even more useful and allow us to sort words, longs and even arrays of strings, if we wish. One way to do this would be to duplicate the code above as many times as we need and edit it accordingly, but that is wasteful even in these days of QPC and other emulators allowing multi-megabytes of RAM. We need a little redesign.

If we extract the compare and swap code to a separate subroutine, we can call it from the main loop, but rather than using a \opcode{BSR} instruction, we can use an address register to hold the compare and swap code's address, and use \opcode{JSR (An)} instead. That way, we only need to set up the address register once, with the desired compare and swap code's address, and we can reuse most of the above code.

Here's the slightly more useful version of the above code - which can replace the above, from line 51 onwards.

\begin{lstlisting}[firstnumber=51,caption={Better Bubblesort}]
;--------------------------------------------------------------------
; ENTRY:
; For entry at label bubblesort:
;
; A1.L = Start address of data to be sorted. Word count first.
;
;--------------------------------------------------------------------
; WORKING:
;
; A1.L = Start Address of data to be sorted, word count first.
; A2.L = Data being compared right now. (-1(a2) and (a2)).
; A3.L = Address of the Compare and swap routine.
; D0.W = `n' = end of unsorted data.
; D1.B = Temp for swapping.
; D2.W = `i' = loop counter.
; D3.W = `newn' = last item sorted.
;--------------------------------------------------------------------
; EXIT:
;
; D0.L = 0.
; A1.L = Preserved - Start address of sorted bytes' word count.
; All other registers preserved.
;--------------------------------------------------------------------
bubblesort
        movem.l d1-d3/a1-a2,-(a7)
        move.w (a1)+,d0       ; N = length(a)
        beq.s bs_done
        subq.w #1,d0          ; We need n-1 when testing

bs_repeat   equ *             ; Repeat
        movea.l a1,a2         ;   A2 = First unsorted byte
        moveq #0,d3           ;   Newn = 0

bs_for_loop 
        moveq #0,d2           ;   For i = 1 to n-1

bs_next
        addq.b #1,d2
        jsr (a3)              ;      Compare and swap if necessary

bs_end_if   equ *             ;     end if
        cmp.w d2,d0           ;     I = n-1 yet?
        bne.s bs_next         ;   End for
        move.w d3,d0          ;   N = newn
        tst.w d0              ;   N = 0 yet?

bs_until
        bne.s bs_repeat       ; Until n = 0

bs_done
        movem.l (a7)+,d1-d3/a1-a2
        clr.l d0
        rts
\end{lstlisting}

In the three example compare and swap routines, see \lstlistingname~\ref{lst:casBytes}, \ref{lst:casWords} and \ref{lst:casLongs}, the usage of the working registers is described in \tablename{~\ref{tab:BubblesortCompareAndSwapRegisters}}.

\begin{table}[htbp]
\centering
\begin{tabular}{c p{0.8\textwidth}}
\toprule
\textbf{Register} &\textbf{Description}  \\
\midrule
%
A1.L & Start Address of data to be sorted.\\
A2.L & Data being compared right now.\\
A3.L & Address of the Compare and swap routine.\\
D0.W & `n' = end of unsorted data.\\
D1.B & Temp for swapping\\
D2.W & `i' = loop counter\\
D3.W & `newn' = last item sorted\\
%
\bottomrule
\end{tabular}
\caption{Working Registers for Bubblesort Compare and Swap Code}
\label{tab:BubblesortCompareAndSwapRegisters}
\end{table}

\begin{lstlisting}[firstnumber=last,caption={Bubblesort - Compare and Swap - Bytes},label={lst:casBytes}]
cas_b
        move.b (a2)+,d1       ; Temp = A[i-1]
        cmp.b (a2),d1         ; If Temp > A[i] then
        bls.s casb_exit       ;   Skip swap if A[i-1] <= A[i]

casb_swap
        move.b (a2),-1(a2)    ; A[i-1] = A[i]
        move.b d1,(a2)        ; A[i] = Temp
        move.w d2,d3          ; Newn = i

casb_exit   rts
\end{lstlisting}

The first action required by the code is to grab the current value to be compared. This is pointed to by A2 on entry and is incremented to point at the next entry. In the above, this is byte sized, but see \lstlistingname~\ref{lst:casBytes}, \ref{lst:casWords} and \ref{lst:casLongs} for subroutines that compare and swap word and long word sized data. The data from the table is loaded into the `temp' variable, also known as D1.size, where size is .B, .W or .L appropriately depending on which compare and swap code we are running.

The comparison between table entries A[i-1] and A[i], from the pseudo code description, actually compares `temp' with `A[i]', or D1.size with (A2), but it's the same comparison. 

In the event that the data in D1 is larger (in this case) than the data in the table pointed to by A2, a swap is made and we set `newn' to the index of the last swap made. We only swap when D1 is larger, that way we don't end up swapping data that are the same. We are running an inefficient algorithm after all, there's no need to make it any more inefficient than we have to.

The `newn' variable tells the main loop of the code to stop comparing because whatever index into the table was last swapped, is where the sorted part of the table begins. We don't need to compare our current value (in D1) with any entries in the table from `newn' onwards. 

The following two subroutines can be used to sort arrays of word and/or long words. All that was changed was the size of the data loaded into D1, the \opcode{CMP} instruction and the data that are swapped around.

\begin{lstlisting}[firstnumber=last,caption={Bubblesort - Compare and Swap - Words},label={lst:casWords}]
cas_w
        move.w (a2)+,d1       ; Temp = A[i-1]
        cmp.w (a2),d1         ; If Temp > A[i] then
        bls.s casw_exit       ;   Skip swap if A[i-1] <= A[i]

casw_swap
        move.w (a2),-2(a2)    ; A[i-1] = A[i]
        move.w d1,(a2)        ; A[i] = Temp
        move.w d2,d3          ; Newn = i

casw_exit   rts
\end{lstlisting}


\begin{lstlisting}[firstnumber=last,caption={Bubblesort - Compare and Swap - Long Words},label={lst:casLongs}]
cas_l
        move.l (a2)+,d1       ; Temp = A[i-1]
        cmp.l (a2),d1         ; If Temp > A[i] then
        bls.s casl_exit       ;   Skip swap if A[i-1] <= A[i]

casl_swap
        move.l (a2),-4(a2)    ; A[i-1] = A[i]
        move.l d1,(a2)        ; A[i] = Temp
        move.w d2,d3          ; Newn = i

casl_exit   rts
\end{lstlisting}

In our test harness, the code requires to be modified to add a pointer to the desired compare and swap routine in register A3, as follows:

\begin{lstlisting}[firstnumber=1,caption={Bubblesort Test Harness Revisited}]
start
        lea stuff,a1          ; Where the data are
        lea cas_b,a3          ; Compare and swap bytes
        bsr.s print_it        ; Print data to #1 unsorted
        bsr.s bubblesort      ; D0.L will be zero
        bsr.s print_it        ; Print sorted data to #1 
        rts

stuff
        dc.w stuff_end-stuff-2
        dc.b 'C','A','D','B'

stuff_end   equ *
\end{lstlisting}

If we were sorting an array of word or long word data, we would simply point A3 at the appropriate subroutine, and that's the only difference.

So far, so good, we have the ability to sort bytes, word and long word based data. What about strings? Well, they are a little different and comparing strings is slightly more complicated than a simple \opcode{cmp.l (a2),d1} instruction, for example. I'll continue with string sorting in the next issue, for now, we can be satisfied with bytes, words and long words. 

There, I think that's all sorted now!

\chapter{Printing Multiple Strings at Once}
Have you ever needed to print multiple strings, one after the other, perhaps with a linefeed between each one? Neither have I until recently. So if you ever find yourself needing to do exactly that, then the following short utility might be of some help.

\begin{lstlisting}[firstnumber=1,caption={Multiprint Utility}]
;--------------------------------------------------------------------
; MULTIPRINT: Prints numerous strings to the channel in A0.L from a
; table of strings at A1.L. The table format is as follows:
;
; strings  dc.w n              ; How many strings?
; s1       dc.w s1e-s1-2       ; Size of string 1
;          dc.b '...'          ; Bytes of string 1
; s1e      ds.w 0              ; Padding byte if required
; s2       dc.w s2e-s2-2       ; Size of string 2
;          dc.b '...'          ; Bytes of string 2
; s2e      ds.w 0              ; Padding byte if required
;                              ; And so on.
;--------------------------------------------------------------------
; REGISTER USAGE:
;
; ENTRY:
;
; A0.L = Channel ID to be used for output.
; A1.L = Start of strings table.
;
; EXIT:
;
; D0.L = Error code or zero. Z flag set accordingly.
; A1.L = Corrupted.
; All other registers preserved.
;--------------------------------------------------------------------
; ENTRY POINTS:
;
; MULTIPRINT - Enter here to print the table of strings exactly as is
; with no additional linefeeds etc between strings. If you want any
; linefeeds, you need to define them in the strings.
;
; MULTIPRINT_LF - Enter here to print the strings with a linefeed 
; printed after each one. There will be a linefeed at the end, after
; the final string too.
;--------------------------------------------------------------------
; WORKING REGISTERS:
;
; D7.L = $0A if linefeeds are requested, zero otherwise.
; D6.W = Strings still to print counter.
; A0.L = Channel ID being printed to.
; A1.L = Running pointer to next string to print.
; A2.L = Used to call QDOSMSQ vector to print a string.
; Others - As required by QDOSMSQ vectors and trap calls.
;--------------------------------------------------------------------

timeout   equ -1              ; Timeout for TRAP #3 calls
linefeed  equ $0A             ; Linefeed character

;--------------------------------------------------------------------
; MULTIPRINT_LF.
;--------------------------------------------------------------------
Multiprint_lf
    move.l d7,-(a7)           ; Save Linefeed indicator
    moveq #linefeed,d7        ; We want linefeeds
    bra.s mp_saveregs         ; And drop in below
       
;--------------------------------------------------------------------
; MULTIPRINT.
;--------------------------------------------------------------------
Multiprint 
    move.l d7,-(a7)           ; See main text
    clr.l d7                  ; No linefeeds required

mp_saveregs
    movem.l d1-d3/d6-d7/a2,-(a7)  ; Save working registers + D7 again!
    move.w (a1)+,d6           ; Fetch counter value
    bra.s mp_next             ; Skip loop first time
       
mp_loop
    move.l a1,-(a7)           ; Save current string
    move.w ut_mtext,a2        ; Get  the vector
    jsr (a2)                  ; Print current string
    bne.s mp_oops             ; Something bad happened
    move.l (a7)+,a1           ; Start of current string
    adda.w (a1),a1            ; Add size word
    addq.l #3,a1              ; Prepare to make even
    move.l a1,d5
    bclr #0,d5                ; D5 now points at next string
    move.l d5,a1              ; Back into A1
       
mp_lf
    move.b d7,d1              ; Linefeed or zero       
    beq.s mp_next             ; Not printing linefeeds
    moveq #io_sbyte,d0        ; Print a byte
    moveq #timeout,d3
    trap #3                   ; Print linefeed
    tst.l d0
    bne.s mp_done             ; Something bad happened
       
mp_next
    dbf d6,mp_loop            ; Go around again
    clr.l d0                  ; No errors detected
    bra.s mp_done             ; Clean up on the way out
       
mp_oops
    adda.l #4,a7              ; Remove saved A1.L
     
mp_done
    movem.l (a7)+,d1-d3/d6-d7/a2  ; Restore working registers
    move.l (a7)+,d7           ; Restore original D7 again

mp_exit
    tst.l d0                  ; Set the Z flag as necessary
    rts
;--------------------------------------------------------------------

\end{lstlisting}

\subsection{Stacking D7 Twice? Why?}

When I originally wrote this code, I explicitly saved the entry value of register D7, by itself, in \opcode{multiprint\_lf} but not in \opcode{multiprint} where it was the linefeed indicator value that was stacked along with the other working registers. When the code was almost done, it popped the working registers off the stack and checked D7 for zero at \opcode{mp\_done}. If it was not zero, I popped D7 off the stack again - assuming that we had entered at \opcode{multiprint\_lf}. Can you see the ever so slightly insidious bug there?

What happens if I enter the code at \opcode{multiprint} with D7 already set to zero, when the utility was done, it would pop D7 off the stack, and check it and on finding it to be zero, would attempt to pop another D7 off the stack, assuming that we had entered at \opcode{multiprint\_lf}. D7 would be loaded with the \emph{calling code's return address} from the stack as opposed to its original value, and so the final \opcode{RTS} would cause a crash.

The solution is as per the code above, D7 gets stacked by both utility routines and will always be popped off at the end, twice. That helps keep the stack neat and tidy and avoids this particular intermittent bug/crash.


\subsection{Testing MultiPrint}
To test the utility code, all you need is something line the following which I've saved typing time and effort by setting up as yet another filter program which allows me to pass a channel number on the command line, and the output will go to that channel. Lazy? me? ;-)

\begin{lstlisting}[firstnumber=1,caption={Testing the Multiprint Utility}]
me	   equ -1                 ; This job
channel_id equ $02            ; Offset(A7) to input file id

start	   
       bra  start_2
       dc.l $00
       dc.w $4afb

name	   
       dc.w name_end-name-2
       dc.b 'MultiPrint Test'

name_end   equ	*

version    
       dc.w vers_end-version-2
       dc.b 'Version 1.00'

vers_end   equ *

str_table
       dc.w 4

s1     dc.w s1e-s1-2
       dc.b 'This is a demo of MultiPrint '
s1e    equ *
       ds.w 0

s2     dc.w s2e-s2-2
       dc.b 'which shows how easy it is to '
s2e    equ *
       ds.w 0

s3     dc.w s3e-s3-2
       dc.b 'print multiple strings in one easy manner. '
s3e    equ *
       ds.w 0

s4     dc.w s4e-s4-2
       dc.b 'Written by Norman Dunbar',$0a
s4e    equ *
       ds.w 0


start_2
       move.l channel_id(a7),a0  ; channel id
       lea str_table,a1       ; Table of strings
       bsr MultiPrint         ; Print with no linefeeds

       lea str_table,a1       ; Table of strings again
       bsr MultiPrint_lf      ; Print with linefeeds between

       moveq #0,d3            ; No error code
       moveq #mt_frjob,d0
       moveq #me,d1           ; This job is about to die
       trap #1
       
       in "ram1_MultiPrint_lib"

\end{lstlisting}

And finally, the ram1\_MultiPrint\_lib file will look like this. However, if you have changed the code layout above (for MultiPrint\_asm) then you may have to regenerate the lib file using the SYM\_bin utility.

\begin{lstlisting}[firstnumber=1,caption={The Multiprint Library File}]
MULTIPRINT_LF  EQU    *+$00000000
MULTIPRINT     EQU    *+$00000006

               lib "ram1_multiprint_bin"
\end{lstlisting}

You should execute the test harness as follows:

\begin{lstlisting}[frame=none,numbers=none,caption={Executing the Multiprint Test Harness}]
ex ram1_MultiPrint_test_bin, #1
\end{lstlisting}

And the output will be something like the following:

\begin{lstlisting}[frame=none,numbers=none,caption={Results of the Multiprint Test Harness}]
This is a demo of MultiPrint which shows how easy it is to print 
multiple strings in one easy manner. Written by Norman Dunbar
This is a demo of MultiPrint 
which shows how easy it is to 
print multiple strings in one easy manner. 
Written by Norman Dunbar

\end{lstlisting}

The first couple of lines shows the data printed ``as is'' without linefeeds. The remainder of the output shows each string printed with a separating linefeed.

Because I had my channel \#1 defined as a quite narrow window, the first line of output wrapped around onto the next line, in the normal manner of printing long strings.

Because there are now two linefeeds after the final string, we get a blank line after the final one. Or, we will when the next print to that channel takes place, it's possible that QDOSMSQ has the final linefeed as pending. I noticed that in testing occasionally.

\chapter{Hexdump Utility}

I'm a frequent user of the Linux/Unix \opcode{hexdump} utility in my real life, and I miss it on QDOSMSQ. I decided to put that right and as a continuation of the use of filter utilities in a previous issue, I decided to make this utility a filter too.

To execute the utility, you simply:

\begin{lstlisting}[frame=none,numbers=none,caption={Executing the Hexdump Utility}]
ex win1_hexdump_bin, source_file, dest_location
\end{lstlisting}

The source file should be obvious, it's the one you want to examine, and the dest\_location can be either a filename or a channel number.

So, without any further ado, here's the code. I'll explain it at the end, but it's fairly simple.

\subsection{Hexdump Listing}

\begin{lstlisting}[firstnumber=1,caption={Hexdump Utility}]
;--------------------------------------------------------------------
; HEXDUMP:
;
; A filter program using an input and output channel, passed on
; the stack for it's files.
; 
; EX hexdump_bin, binary_file, output_file
;
;--------------------------------------------------------------------
; 21/09/2015 NDunbar Created for QDOSMSQ Assembly Mailing List
;--------------------------------------------------------------------
; (c) Norman Dunbar, 2015. Permission granted for unlimited use
; or abuse, without attribution being required. Just enjoy!
;--------------------------------------------------------------------

me         equ -1                 ; This job
infinite   equ -1                 ; For timeouts
err_bp     equ -15                ; Bad parameter error
linefeed   equ $0A                ; Linefeed character
eof        equ -10                ; End of file
buff_size  equ $10                ; Maximum size of read buffer
out_size   equ 73                 ; Output string length
space      equ ' '                ; 1 space
dot        equ '.'                ; 1 dot
max_char   equ $C0                ; Highest printable ASCII character

source_id  equ $02                ; Offset(A7) to input file id
dest_id    equ $06                ; Offset(A7) to output file id
param_size equ $0A                ; Offset(A7) to command string size
param      equ $0C                ; Offset(A7) to command bytes

start      
       bra  Hexdump
       dc.l $00
       dc.w $4afb

name       
       dc.w name_end-name-2
       dc.b 'Hexdump'

name_end   equ  *

version    
       dc.w vers_end-version-2
       dc.b 'Version 1.00'

vers_end   equ *

in_buffer  
       ds.l 4                 ; 16 bytes read at a time

out_buffer 
       ds.l 20                ; 80 bytes max output

open_bracket equ out_buffer+54    ; Where '[' should be
close_bracket equ out_buffer+71   ; Where ']' should be

;--------------------------------------------------------------------
; Stack on entry:
;
; $0c(a7) = bytes of parameter + padding, if odd length. (Ignored)
; $0a(a7) = Parameter size word. (Ignored)
; $06(a7) = Output file channel id.
; $02(a7) = Source file channel id.
; $00(a7) = How many channels? Should be $02.
;--------------------------------------------------------------------
bad_parameter
       moveq #err_bp,d0       ; Guess!
       bra suicide            ; Die horribly

Hexdump
       cmpi.w #$02,(a7)       ; Two channels is a must
       bne.s bad_parameter    ; Oops

start_loop
       moveq #infinite,d3     ; Timeout - preserved throughout
       clr.l d7               ; Current location in file

read_loop
       move.l source_id(a7),a0  ; Input channel id
       lea in_buffer,a1        ; Where to read the data into
       moveq #buff_size,d2    ; Maximum size of the buffer
       moveq #io_fstrg,d0     ; Trap utility we want
       trap #3                ; Read a chunk of source file
       tst.l d0               ; Did it work?
       beq.s read_ok          ; Not EOF yet, carry on
       cmpi.l #eof,d0         ; EOF?
       bne error_exit         ; Something bad happened
       tst.w d1               ; Any remaining data?
       beq all_done           ; No, exit the main loop

read_ok
       lea in_buffer,a2       ; Source buffer
       lea out_buffer,a1      ; Output buffer
       moveq #79,d0           ; 80 bytes to clear

;--------------------------------------------------------------------
; Space fill the entire output buffer on each pass through the loop.
;--------------------------------------------------------------------
ob_clear
       move.b #space,(a1,d0.w) ; Space fill from the end back
       dbf d0,ob_clear        ; And do the rest
       moveq #0,d5            ; Extra linefeed counter

;--------------------------------------------------------------------
; Add the address to the buffer as 8 hex characters. Then 4 spaces.
;--------------------------------------------------------------------
hd_address
       move.l d7,d4           ; D4 is required here
       beq.s hd_continue      ; No extra linefeed at start
       cmpi.b #0,d7           ; On a 256 Byte boundary?
       bne.s hd_continue      ; Nope.
       move.b #linefeed,(a1)+ ; Yes, extra linefeed
       moveq #1,d5            ; Adjust counter

hd_continue
       ext.l d1               ; Curently only word sized
       add.l d1,d7            ; Update file offset counter
       bsr hex_l              ; Store address in buffer at A1
       adda.l #4,a1           ; Leave 4 spaces

;--------------------------------------------------------------------
; There might not always be 16 bytes to convert. Adjust the count to
; add groups of 4 bytes then two spaces to the output buffer, by 
; counting long words and then the remaining spare bytes.
;--------------------------------------------------------------------
hd_data
       move.l d1,d0           ; Byte counter (long sized)
       divu #4,d0             ; D0.Low = Long word count
;                             ; D0.High = Byte count remainder
       bra.s hdl_next         ; Skip first time

hdl_loop
       move.l (a2)+,d4        ; Get a long word
       bsr.s hex_l            ; Add hex to buffer
       adda.l #2,a1           ; Leave 2 spaces between groups

hdl_next
       dbf d0,hdl_loop        ; Do next long word            

       swap d0                ; D0.W = remaining bytes (0-3)
       bra.s hdb_next         ; Skip first byte

hdb_loop
       move.b (a2)+,d4        ; Get a byte
       bsr.s hex_b            ; Add to buffer

hdb_next
       dbf d0,hdb_loop        ; Do next byte

;--------------------------------------------------------------------
; Because we don't always get 16 bytes, we simply force A1 to the
; desired location in the output buffer.
;--------------------------------------------------------------------
hd_ascii
       lea open_bracket,a1    ; where to put the '['
       adda.w d5,a1           ; Adjust for extra linefeeds
       lea in_buffer,a2       ; Back to the start of data
       move.w d1,d0           ; Data counter   
       move.b #'[',(a1)+      ; Opening delimiter added

       bra hda_next           ; Skip first time

hda_loop
       move.b (a2)+,d2        ; Fetch byte of data
       cmpi.b #space,d2       ; We can print space or higher only
       bcs.s hda_dot          ; This character is not ok
       cmpi.b #max_char,d2    ; Reached the control characters?
       bcs.s hda_store        ; No, this one is fine

hda_dot
       moveq #dot,d2          ; Print a dot instead

hda_store
       move.b d2,(a1)+        ; Save in output buffer

hda_next
       dbf d0,hda_loop        ; And do the rest

       lea close_bracket,a1   ; Where to put the ']'
       adda.w d5,a1           ; Adjust for extra linefeeds
       move.b #']',(a1)+      ; Closing delimiter added
       move.b #linefeed,(a1)  ; And linefeed at the end

hd_print
       moveq #io_sstrg,d0     ; Trap call we want
       moveq #out_size,d2     ; How many bytes?
       add.w d5,d2            ; Adjust for extra linefeeds
       lea out_buffer,a1      ; Where our string is
       move.l dest_id(a7),a0  ; Output channel
       trap #3                ; Do it
       tst.l d0               ; Did it work?
       beq read_loop          ; Yes, continue

error_exit
       move.l d0,d3           ; Error code we want to return
       bra.s suicide          ; And die

all_done
       moveq #0,d3            ; No error code

suicide
       moveq #mt_frjob,d0
       moveq #me,d1           ; This job is about to die
       trap #1

;--------------------------------------------------------------------
; The hex conversion routines in QDOS are corrupt in some versions so
; these will work. The take a long, word, byte or nibble in D4 and
; write the hex byte(s) to a buffer pointed to by A1.
;
; The various routines here call a lower level one, then drop into
; the called code again to process the "other half" of the data to be
; converted.
;--------------------------------------------------------------------
hex_l
       swap d4                ; We do this in MS word order
       bsr.s hex_w            ; Do original high word
       swap d4                ; Get low word back

hex_w
       ror.w #8,d4            ; We do this in MS byte order
       bsr.s hex_b            ; Do original high byte
       rol.w #8,d4            ; Get low byte back

hex_b
       ror.b #4,d4            ; We do this in MS nibble order
       bsr.s hex_nibble       ; Do original high nibble
       rol.b #4,d4            ; Get original low niggle back

hex_nibble
       move.b d4,-(a7)        ; We need to save the byte
       andi.b #$0f,d4         ; Mask out low nibble
       addi.b #'0',d4         ; Assume digit 0-9
       cmpi.b #'9',d4         ; Digit?
       bls.s hex_store        ; Yes, digit
       addi.b #7,d4           ; Offset for an A-F character

hex_store
       move.b d4,(a1)+        ; Add to the buffer at A1.L
       move.b (a7)+,d4        ; Retrieve original byte
       rts
\end{lstlisting}

\subsection{Hexdump Code Explained}

As ever with my code, the first part is a load of bumff explaining briefly, sometimes, what the program should be doing. This utility is no different! Following on, we have a number of equates defined. The important ones here should be adequately commented - but we set up various offsets onto the A7 stack to extract the source file and destination channel ids and, not \emph{currently} used here, where we should find the command string, if passed.

Then there is the usual standard QDOS header for a job with the job name embedded and a couple of buffers. The input buffer is where we read the source file into, 16 bytes at a time. The output buffer is big enough to hold a printed output line of up to 80 characters. You may note that a program version has been defined, but is only for my own documentation, it is never display or used. Feel free to leave it out.

The next couple of equates define the locations in the output buffer where the '[' and ']' surrounding the ASCII representation of the hex codes will be.

Just before the main Hexdump code itself, we have the \opcode{bad\_parameter} code which is, as you might expect, used to handle bad parameters - these are when we get less than or more than two channels on the stack at execution time. The utility simply exits with an error code back to the caller. 

Be aware that you will not see this error code if you \opcode{EX} the utility, only if you call it with \opcode{EW} will errors be reported back to SuperBasic. This is normal.

\opcode{Hexdump} starts by checking the word on the stack to ensure that we only received two channel ids on the stack. If this is not the case, we exit via \opcode{bad\_parameter} as explained above. Assuming this is not the case, we preload D3 with an infinite timeout. This is preserved through all trap calls, so only needs to be done once.

We use D7 as the current offset counter, so we initialise it to zero, as we are still at the start of the source file.

\opcode{Read\_loop} is the start of the main loop. In here, we load the source file's channel id into A0 and read the next 16 bytes, maximum, into the input buffer. When we hit end of file, we need to ensure that the last few remaining bytes are converted to hex - if there was not exactly 16 bytes read when we hit EOF, they are still valid. We test D1 to be sure that we do have some data to process, if not, we are truly at EOF and we bale out of the utility passing a zero error code back to the caller.

If there was some other error in the read, ie, not EOF, then we simply bale out and return the error code to the caller. 

Assuming all went well, we enter the code at \opcode{read\_ok} where we set up A2 and A1 with the input and output buffer addresses respectively. As we want spaces in between each section of data in the output buffer, we fill all 80 bytes with spaces, prior to each conversion, at \opcode{ob\_clear}. D5 is cleared here as well, on each pass, as it counts the number of extra linefeeds that have been injected into the output buffer - zero or one - and is used to adjust various pointers and counts as necessary.

The code at \opcode{hd\_address} copies the current offset from D7 into D4 and if this is the start of the file - the offset is zero - skips over the next bit. Assuming that this is not the start of the file, we wish to insert an extra linefeed after every 256 bytes of the input file. This is easy to accomplish as we simply need to check the lowest byte of the offset. If it is zero, then we add a linefeed to the buffer and set D5 to 1 to show the extra byte. This happens at offsets \$0100, \$0200, \$0300 and so on.

Prior to updating D7 with the count of the bytes just read. For most of the file, this will be 16 but there may be less at EOF. As the offset in D7 is long sized - we could be dumping large files - we have to extent D1 from a word to a long prior to the addition. D4 is converted from an offset to 8 hex characters in a call to \opcode{hex\_l} which adds the converted characters to the output buffer and updates A1.

After the address has been added, we wish to have 4 spaces after it, so A1 is incremented by 4 to account for this. We are now ready to convert the data.

\opcode{Hd\_data} is where this happens. The bytes read is copied to D0 as a long word and then divided by 4 to get the number of long words read in. In most cases this will be 4, at least until we get to EOF. After the division, the low word of D0 holds the number of long words to convert and the high word holds the remaining bytes to convert afterwards. Each long word is converted by copying it to D4.L and calling out to the \opcode{hex\_l} code again to convert and add it to the buffer as 8 hex characters. Two spaces are then `added' by incrementing A1 accordingly.

After all the long words are converted, we process the remaining bytes by swapping D0 around so that the remaining bytes are in the low word, and we loop around those converting them one byte at a time at \opcode{hdb\_loop}.

After all the bytes are processed and added to the buffer, we need to add in the ASCII characters. Only printable ones will be considered - those between `space' and the down arrow character, inclusive. Anything less than a space or any of the control characters from \$C0 upwards are represented by a dot.

The first part of the code at \opcode{hd\_ascii} adds an opening bracket to the buffer, then the individual ASCII characters are added, all 16 (usually) of them, then a closing bracket is added to the buffer followed by a linefeed. If we injected an extra linefeed previously, then D5 is added to the offsets for the opening and closing brackets to ensure that they are inserted into the buffer at the correct location.

We then drop into \opcode{hd\_print} where we send the completed buffer, to the destination file or channel before looping around and back to \opcode{read\_loop} to do it all again. Once again, the counter in D2 which determines the size of the string to print has to be adjusted to account for any extra linefeeds, so D5 is added to D2 before the \opcode{TRAP \#3}.

In the unlikely event of an error during the conversion to hex, the code at \opcode{error\_exit} will be executed to copy the error code from D0 into D3 prior to returning to the caller. If there were no errors, then \opcode{all\_done} will cause a zero to be returned. The job then kills itself which will cleanly close the input and output files, flushing any buffers as appropriate.

\subsection{Hex Conversion}

As noted in the comments, certain versions of QDOS, prior to 1.03 I believe, have hex conversion routines in the ROM, but they are somewhat broken. To this end, I have supplied my own. To use them, D4 should contain the value to be converted and A1 should point to a location in a buffer, somewhere, for the results. After conversion, A1 is updated to the next free location in the buffer.

The following is a sample of the output from the utility when used to hexdump an earlier incarnation\footnote{A \emph{much} earlier version!} of itself.

\begin{lstlisting}[frame=none,numbers=none,caption={Example Hexdump Output}]
00000000    60000078  00000000  4AFB0007  48657864    [`..x....J...Hexd]
00000010    756D7000  61736D00  00000000  00000000    [ump.asm.........]
00000020    00000000  00000000  00000000  00000000    [................]
00000030    66EDE055  00010002  00000000  00000000    [f..U............]
00000040    00000000  00000000  00000000  00000000    [................]
00000050    00000000  00000000  00000000  00000000    [................]
00000060    00000000  00000000  00000000  00000000    [................]
00000070    00000000  70F16000  00C00C57  000266F4    [....p.`....W..f.]
00000080    76FF4287  206F0002  43FAFF8A  74107003    [v.B. o..C...t.p.]
00000090    4E434A80  67100C80  FFFFFFF6  66000094    [NCJ.g.......f...]
000000A0    4A416700  009245FA  FF6C43FA  FF78704F    [JAg...E..lC..xpO]
000000B0    13BC0020  000051C8  FFF82807  48C1DE81    [... ..Q...(.H...]
000000C0    617CD3FC  00000004  200180FC  0004600A    [a|...... .....`.]
000000D0    281A616A  D3FC0000  000251C8  FFF44840    [(.aj......Q...H@]
000000E0    6004181A  616451C8  FFFA43FA  FF6E45FA    [`...adQ...C..nE.]
000000F0    FF243001  12FC005B  60000014  141A0C02    [.$0....[`.......]

00000100    00206506  0C0200C0  6502742E  12C251C8    [. e.....e.t...Q.]
00000110    FFEC43FA  FF5712FC  005D12BC  000A7007    [..C..W...]....p.]
00000120    744943FA  FF00206F  00064E43  4A806700    [tIC... o..NCJ.g.]
00000130    FF542600  60027600  700572FF  4E414844    [.T&.`.v.p.r.NAHD]
00000140    61024844  E05C6102  E15CE81C  6102E91C    [a.HD.\a..\..a...]
00000150    1F040204  000F0604  00300C04  00396304    [.........0...9c.]
00000160    06040007  12C4181F  4E75                  [........Nu      ]
\end{lstlisting}


\chapter{Jump Tables}

Imagine that your next great programming wonder is not based on the Pointer Environment, but does display a menu to the user with a number of options\footnote{It wouldn't be much of a menu otherwise, would it? :-)}. Each option can be selected by a single key press, and your application code has to choose a piece of code, a subroutine, to handle the user's choice.

You could do something like the following, where we assume that only the 10 digits are allowed and that D0.B holds the keypress character from the menu.

\begin{lstlisting}[firstnumber=1,caption={Processing User Options - First Attempt}]
;--------------------------------------------------------------------
main_loop
       bsr display_menu       ; CLS and display the menu
       bsr get_menu_option    ; Wait for a menu choice
       
got_menu_option
       cmpi.b #'0',d0         ; Zero or above?
       bcs bad_option         ; Oops
       cmpi.b #'9',d0         ; Nine or below?
       bcc bad_option         ; Oops
       
got_good_option
       cmpi.b #'0',d0
       beq option_0           ; Process option '0'
       cmpi.b #'1',d0
       beq option_1           ; Process option '1'
       ...
       ...
       cmpi.b #'8',d0
       beq option_8           ; Process option '8'
       cmpi.b #'9',d0         ; Not strictly required, but safe
       beq option_9           ; Process option '9'

option_return     
       ; do some post routine clean up here      
       ...
       ...
       bra main_loop          ; Ready for the next option

option_0
       ; Process option zero here.
       ...
       bra option_return      ; Back to the main loop
       
option_1
       ; Process option one here.
       ...
       bra option_return      ; Back to the main loop
       ...
       ...
;--------------------------------------------------------------------
\end{lstlisting}

Ignoring the fact that there are numerous helper routines called, but not shown in the above example, then we can see that the above is quite simple to read and is fine for a small number of options. However, note that none of the option handling subroutines can use an \opcode{RTS} instruction to exit, as the call to the subroutine was by way of a \opcode{BEQ} instruction. They must therefore execute a \opcode{bra option\_return} to get back into the clean up code and back to the main loop.

We could improve matters slightly and use the \opcode{PEA} here to set up a pseudo subroutine call, by pushing the \opcode{common\_return} address onto the stack prior to calling any of the subroutines, as follows.

\begin{lstlisting}[firstnumber=1,caption={Processing User Options - Improved First Attempt}]
;--------------------------------------------------------------------
main_loop
       bsr display_menu       ; CLS and display the menu
       bsr get_menu_option    ; Wait for a menu choice
       
got_menu_option
       cmpi.b #'0',d0         ; Zero or above?
       bcs bad_option         ; Oops
       cmpi.b #'9',d0         ; Nine or below?
       bcc bad_option         ; Oops
       
got_good_option
       pea option_return      ; Stack a "return" address
       
       cmpi.b #'0',d0
       beq option_0           ; Process option '0'
       ...
       ...
       cmpi.b #'9',d0         ; Not strictly required, but safe
       beq option_9           ; Process option '9'

option_return           
       ; do some post routine clean up here      
       ...
       ...
       bra main_loop          ; Ready for the next option

option_0
       ; Process option zero here.
       ...
       rts                    ; Back to option_return
       
option_1
       ; Process option one here.
       ...
       rts                    ; Back to option_return
       
       ...
       ...
;--------------------------------------------------------------------
\end{lstlisting}

This version is a lot better, while we are still calling the subroutines with a \opcode{BEQ} instruction, we have fiddled the stack by pushing a common return address onto it when we know we have a valid menu option. When each individual subroutine executes the \opcode{RTS} at the end, it will pop the address of \opcode{option\_return} and continue executing from there.

We could, if we wished to use the actual \opcode{BSR} instruction, perhaps to avoid confusion, code something like the following.

\begin{lstlisting}[firstnumber=1,caption={Processing User Options - Another Improved First Attempt}]
;--------------------------------------------------------------------
main_loop
       bsr display_menu       ; CLS and display the menu
       bsr get_menu_option    ; Wait for a menu choice
       
got_menu_option
       cmpi.b #'0',d0         ; Zero or above?
       bcs bad_option         ; Oops
       cmpi.b #'9',d0         ; Nine or below?
       bcc bad_option         ; Oops
       
got_good_option
       cmpi.b #'0',d0
       bne.s ggo_try_1        ; Not zero
       bsr option_0           ; Process option '0'
       bra option_return      ; Do cleanup

ggo_try_1
       cmpi.b #'1',d0
       bne.s ggo_try_2        ; Not '1'
       bsr option_1           ; Process option '1'
       bra option_return      ; Do cleanup

       ...
       ...
ggo_try_8
       cmpi.b #'8',d0
       bne.s ggo_try_9        ; Not '8'
       bsr option_8           ; Process option '8'
       bra option_return      ; Do cleanup

ggo_try_9
       cmpi.b #'9',d0         ; Not strictly required, but safe
       bne.s option_return    ; Not '9'
       bsr option_9           ; Process option '9'
       bra option_return      ; Do cleanup

option_return           
       ; do some post routine clean up here      
       ...
       ...
       bra main_loop          ; Ready for the next option

option_0
       ; Process option zero here.
       ...
       rts
       
option_1
       ; Process option one here.
       ...
       rts

       ...
       ...
;--------------------------------------------------------------------
\end{lstlisting}

So, in this version, we are using the \opcode{BSR} instruction that we wanted to, but now we've had to invert all the flag checks after the \opcode{cmpi.b \#whatever,d0} and add in numerous new labels and branches, plus, after a successful return from the subroutine, we need an explicit branch to the clean up code at the bottom of the loop. It's all getting rather messy now.

You can imagine that as we add more and more menu options, that adding in new subroutines etc could get a bit frantic, especially trying to remember to do all the branches etc. In addition, there's much more typing, and, if you type like I do, too much room for errors!\footnote{I've been in the IT business since around 1982, I \emph{still} cannot touch type, I have to look at the keyboard to see where the next key I want is hiding!}

Jump tables are easily set up, and can make life so much easier, with a lot less typing, although, it could be said that they are slightly less easily understood\footnote{At least until you begin to understand exactly how useful they really are!}.

\begin{lstlisting}[firstnumber=1,caption={Processing User Options - Jump Tables}]
;--------------------------------------------------------------------  
JumpTable
       dc.w option_0-JumpTable
       dc.w option_1-JumpTable
       dc.w option_2-JumpTable
       dc.w option_3-JumpTable
       dc.w option_4-JumpTable
       dc.w option_5-JumpTable
       dc.w option_6-JumpTable
       dc.w option_7-JumpTable
       dc.w option_8-JumpTable
       dc.w option_9-JumpTable
       
main_loop
       bsr display_menu       ; CLS and display the menu
       bsr get_menu_option    ; Wait for a menu choice
       
got_menu_option
       cmpi.b #'0',d0         ; Zero or above?
       bcs bad_option         ; Oops
       cmpi.b #'9',d0         ; Nine or below?
       bcc bad_option         ; Oops
       
got_good_option
       subq.b #'0',d0         ; D0.B = 0 to 9 as a number
       ext.w d0               ; Now extend to a word
       lsl.w #1,d0            ; Convert to a table offset
       lea JumpTable,a2       ; Where the jump table lives
       jsr (a2,d0.w)          ; Jump to the correct subroutine

option_return           
       ; do some post routine clean up here      
       ...
       ...
       bra main_loop          ; Ready for the next option
       
option_0
       ; Process option zero here.
       ...
       rts
       
option_1
       ; Process option one here.
       ...
       rts

       ...
       ...
;--------------------------------------------------------------------
\end{lstlisting}

Each entry in the table surprisingly names \opcode{JumpTable} is a word sized \emph{signed} offset to the desired routine, from the start of the table itself. This allows for subroutines that are located prior to, or after, the jump table being defined. Negative offsets are to subroutines defined before the table, and positive offsets are to subroutines defined after the jump table. Simple? 

You can see how much less code there is at the label \opcode{got\_good\_option}. At that point all we have to do is convert D0.B from a byte, containing one of the characters `0' through `9', into a word containing the numeric value zero to nine, as opposed to the character `0' to `9', then double it as each entry in the table takes two bytes. The offset to the \opcode{option\_0} subroutine is at \opcode{JumpTable} + 0, while that for the \opcode{option\_1} subroutine is at \opcode{JumpTable} + 2 and so on.

Obviously, the code at \opcode{main\_loop} is executed without passing through the preceding jump table, or who knows what might happen! Jump tables are data, not code.

The \opcode{jsr (a2,d0.w)} takes care of calling the correct routine, as A2 is pre-loaded with the address of \opcode{JumpTable}. On return, we drop into the clean up code and pass back to the main loop start again. Remember, D0.W will be sign extended to a long word prior to adding it to A2.L.

Adding new options is a simple matter of inserting or appending a new entry to the jump table \emph{in the correct place}, and making sure that D0.W is set equal to the offset in the jump table, so that when we execute the \opcode{jsr (a2,d0.w)} instruction, we get the correct subroutine address.

\subsection{What About Missing Options}
So far so good, our table holds one subroutine offset for each menu option from `0' to `9', which gets translated to a value between 0 and 9, and subsequently, into an offset into the table of offset words\footnote{Ugh! Too many offsets in that sentence!}. What do we do if, for example, option 5 is not actually allowed? We have a couple of choices:

\begin{itemize}
\item Filter out the illegal option(s) when checking for a valid choice.
\item Use a `do nothing' entry for the invalid choice(s) in the table.
\item Use a zero offset in the table,test for it in the  and don't jump if that is found.
\end{itemize}

The first option is obviously the best as it gives you the opportunity to advise the user of their error when they try to make an invalid choice. The last option would require a slight change to the code at \opcode{got\_good\_option}, as follows:

\begin{lstlisting}[firstnumber=1,caption={Processing User Options - Jump Tables}]
got_good_option
       subq.b #'0',d0         ; D0.B = 0 to 9 as a number
       ext.w d0               ; Now extend to a word
       lsl.w #1,d0            ; Convert to a table offset
       lea JumpTable,a2       ; Where the jump table lives
       tst.w (a2,d0.w)        ; Valid offset?
       beq.s no_jump          ; No, do nothing
       jsr (a2,d0.w)          ; Jump to the correct subroutine
\end{lstlisting}

The code at label \opcode{no\_jump} would do whatever is required prior to the next pass through the main loop.

\chapter{Using the MC68020}
As mentioned in the last issue, I am planning on upgrading the eComic to use the 68020 instructions available in \program{QPC}QPC and in George's \program{Gwass}Gwass assembler. This currently means that unless you have a Q40 or Q60 to hand, you will need to run the programs and assembler on \program{QPC}QPC. Is this a problem I wonder?

\section{Overview}
Here are a few brief details of what a proper 68020 has to offer:

\begin{itemize}
\item Full support for 32 bit operations.
\item A full 32 bit external data bus which can also cope with 16 or 8 bit peripherals.
\item 32 bit offsets in branch instructions.
\item 32 bit displacements in indexed addressing modes.
\item Two new addressing modes are provided, which allow indexed address with \emph{two} levels of indirection. 
\item Word and Long memory accessing need no longer be on an even address.
\item New bit field instructions.
\item Instructions to convert between character and decimal numbers.
\item And lots, lots more!
\end{itemize}


\section{Addressing Modes}
The addressing modes of the 68008 are mostly familiar, or should be by now, however, here is a reminder of those modes, plus the new modes available in the 68020. The mode number given is that coded into the mode bits of the effective address in the various instructions. (But you don't really \emph{need} to know this!) 

In the following descriptions, I've taken the wording as is from the Motorola Programmers' Manual - hence the strangeness of some of the wording. The examples, however, are mine.

\subsection{Data Register Direct}
Mode zero. The effective address specifies the data register that the contains the operand. For example \opcode{move.w \#1,d0} the destination address is Data Register Direct.

\subsection{Address Register Direct}
Mode 1. The effective address specifies the address register that the contains the operand. For example \opcode{move.l \$c0ffee,a3}. The destination effective address is \register{A3} and is indeed Address Register Direct.

\subsection{Address Register Indirect}
Mode 2. The effective address specifies the address register that the contains the operand in memory. For example \opcode{move.w (a3),d7} where \register{A3} holds the address where the operand, a word of data, is to be found.

\subsection{Address Register Indirect with Post-Increment}
Mode 3. In the address register indirect with postincrement mode, the operand is in memory. The effective address field specifies the address register containing the address of the operand in memory. 

After the operand address is used, it is incremented by one, two, or four
depending on the size of the operand: byte, word, or long word, respectively. 

For example \opcode{move.w (a0)+,d0} will read the word value from the address held in \register{A0} into \register{D0} and then will add two - the size of a word - to A3.

If the address register is \register{a7}, the stack pointer, then byte sized operations cause \register{a7} to be incremented by 2, rather than by 1. For example \opcode{move.b d0, (a7)+} will cause \register{A7} to be incremented by two to keep it even.

The address register in question retains the new incremented value after the instruction.

\subsection{Address Register Indirect with Pre-Decrement}
Mode 4. In the address register indirect with predecrement mode, the operand is in memory. The effective address field specifies the address register containing the address of the operand in memory. 

Before the operand address is used, it is decremented by one, two, or four depending on the operand size: byte, word, or long word, respectively. 

For example \opcode{move.l -(a0),d0} causes \register{A0} to be decremented by 4 and the long word found at the new address will be moved into \register{D0}.

If the register is \register{A7}, the stack pointer, then byte sized operations cause \register{A7} to be incremented by 2, rather than by 1. For example \opcode{move.b d0,-(a7)}.

The address register in question retains the new decremented value after the instruction.

\subsection{Address Register Indirect with Displacement}
Mode 5. In the address register indirect with displacement mode, the operand is in memory. 

The sum of the address in the address register, which the effective address specifies, plus the sign-extended 16-bit displacement integer in the extension word is the operand’s address in memory. 

Displacements are always sign-extended to 32 bits prior to being used in effective address calculations.

For example \opcode{move.l d0,\$10(a1)} will sign-extend the 16 bit displacement word ($10_{hex}$) to a full 32 bit signed value, add it to the address held currently in \register{A1} - without affecting the actual address held in the register - and the long value in \register{D0} will then be stored there.

The displacement can of course, be negative, \opcode{move.l d0,-\$14(a1)}.

The displacement word is 16 bits, however, it will always be sign extended to 32 bits prior to the addition to the address register.

The address register in question retains its \emph{current} value after the instruction - it is not adjusted in any way.


\subsection{Address Register Indirect with Index (8 bit Displacement)}\label{sub-ARegII8}
Mode 6. This addressing mode requires one extension word that contains an index register indicator and an 8-bit displacement. The index register indicator includes size and scale information.

In this mode, the operand is in memory. The operand’s address is the sum of the address
register’s contents; the sign-extended displacement value in the extension word’s low-order eight bits; and the index register’s sign-extended contents (possibly scaled). 

The user \emph{must} specify the address register, the displacement, and the index register in this mode - none of these are optional, only the scaling factor is optional and will default to 1 if omitted.

For example \opcode{move.w 4(a6,d7.W),d3}. In this example, the 8 bit displacement value, 4, is sign extended to 32 bits and added to the address held in \register{A6}. The value in \register{D7} is also sign extended to 32 bits and added to the above calculation. The word value at this calculated address is copied into \register{D3}.

The calculated address is not stored anywhere, it is used and discarded. The value in the address register, \register{A6} in this case, is not affected.

The index register, \register{D7} may, optionally, have its value scaled - which the example code shown below attempts to explain.

It seems,according to the Programmers' manual, that we \emph{should} be writing the above example as \opcode{move.w (4,a6,d7.W),d3} instead. Luckily \program{Gwass}GWASS is happy with the old style\footnote{My preferred style!} as well as the new.

As mentioned, both the 8 bit displacement and the index register, if word sized, will be sign extended to 32 bits before being used in effective address calculations.

As for the scaling mode mentioned above, do you remember last issue's jump tables article? Well, here's a reminder\footnote{Corrected as per George's comments!}:

\begin{lstlisting}[firstnumber=1,caption={Jump Table - Old Style}]
got_good_option
       subq.b #'0',d0         ; D0.B = 0 to 9 as a number
       ext.w d0               ; Now extend to a word
       lsl.w #1,d0            ; Convert to a table offset
       lea JumpTable,a2       ; Where the jump table lives
       move.w 0(a2,d0.w),d0   ; Fetch the offset word
       jsr (a2,d0.w)          ; Jump to the correct subroutine
\end{lstlisting}       
       
Now, with the 68020 and scaling, there's no need to do the separate doubling of the table's index (\opcode{lsl.w \#1,d0}) to calculate the correct offset into the table as the scaling does this automatically and \emph{without changing \register{d0}}. The above extract would be as written as follows:

\begin{lstlisting}[firstnumber=1,caption={Jump Table - New Style}]
got_good_option
       subq.b #'0',d0         ; D0.B = 0 to 9 as a number
       lea JumpTable,a2       ; Where the jump table lives
       move.w 0(a2,d0.w*2),d0 ; Fetch the offset word
       jsr (a2,d0.w)          ; Jump to the correct subroutine
\end{lstlisting}       
       
You will notice that I have specified the displacement (0) and both the address (\register{A2}) and index register (\register{D0.W}) as required.


\subsection{Address Register Indirect with Index (Base Displacement)}\label{sub-ARegIIbd}
Also mode 6. This addressing mode requires an index register indicator and an optional 16- or 32-bit sign-extended base displacement. The index register indicator includes size and scaling information. The operand is in memory. 

The operand’s address is the sum of the contents of the address register, the base displacement, signed extended if necessary, and the scaled contents of the sign-extended index register.

In this mode, the address register, the index register, and the displacement are \emph{all optional}.

The effective address is zero if there is no specification. This mode can provide a data register indirect address when there is no specific address register and the index register is a data register.

The example for this addressing mode is similar to the one above, however you don't need to specify all of the fields and scaling. For example we can change our addressing mode from \opcode{move.w 0(a2,d0.w*2),d0} to \opcode{move.w (a2,d0.w*2),d0} where the displacement is optional.

As mentioned, this mode can give you a pseudo \emph{Data register Indirect} addressing mode, simply by leaving off most of the optional fields. For example, under the 68020, the following is valid \opcode{move.w (d0.l),d0} - assuming that \register{D0.L} contains a valid `address' of course.

\subsection{Memory Indirect Postindexed}\label{sub-MIPostI}
Also mode 6. In this mode, both the operand and its address are in memory. The processor calculates an intermediate indirect memory address using a base address register and base
displacement. 

The processor accesses a long word at this address and adds the index operand (Xn.SIZE*SCALE) and the outer displacement to yield the effective address. 

Both displacements and the index register contents are sign-extended to 32 bits.

In the syntax for this mode, square brackets [] enclose the values used to calculate the intermediate memory address.

All four user-specified values are optional. 

Both the base and outer displacements may be null, word, or long word. When omitting a displacement or suppressing an element, its value is zero in the effective address calculation.

For example \opcode{move.l ([8,a6],d4.w*4,96),d0} will calculate a temporary address in memory by adding the sign-extended base displacement (8) and the address register (\register{A6}). This address will contain a long word which is read, added to the sign-extended index register (\register{D4.W*4}), plus the outer displacement (96). Phew!

The immediate question that comes to my mind is ``why?'' However, there must have been a reason for this addressing mode to be built in silicon.

\subsection{Memory Indirect Preindexed}\label{sub-MIPreI}
Mode 6 again. In this mode, both the operand and its address are in memory. The processor calculates an intermediate indirect memory address using a base address register, a base displacement, and the index operand (Xn.SIZE*SCALE). 

The processor accesses a long word at this address and then adds the outer displacement to yield the effective address. 

Both displacements and the index register contents are sign-extended to 32 bits.

In the syntax for this mode, brackets enclose the values used to calculate the intermediate memory address. 

All four user-specified values are optional. 

Both the base and outer displacements may be null, word, or long word. When omitting a displacement or suppressing an element, its value is zero in the effective address calculation.

For example \opcode{move.l ([18,a5,d4.w*4],200),d0} will calculate a temporary address in memory by adding the sign-extended base displacement (18), the address register (\register{A5}) and the sign-extended index register (\register{D4.W*4}). The long word at that address will then be read and added to the outer displacement (200) and whatever long word is found at that address will be copied into \register{D0}. Phew!

The immediate question that again comes to my mind is ``why?''

\subsection{Absolute Short}
Mode 7 Submode 0. In this addressing mode, the operand is in memory, and the address of the operand is in the extension word. The 16-bit address is sign-extended to 32 bits before it is used.

For example \opcode{move.w \$1234,d4} takes 2 words of memory. The first defines the opcode and the second word defines the short address. The second word is read, sign-extended to 32 bits and the word, in this example, at that address is copied into \register{D4}.

Note that addresses between \$0000 and \$7fff sign-extend to the same values, but addresses from \$8000 to \$ffff sign-extend to actual addresses of \$ffff8000 to \$ffffffff. So, effectively, you can only use this addressing mode on the lowest 32KB of memory and, if you have enough RAM, the upper 32KB of memory.

\subsection{Absolute Long}
Mode 7 Submode 1. In this addressing mode, the operand is in memory, and the operand’s address occupies the two extension words following the instruction word in memory. 

The first extension word contains the high-order part of the address; the second contains the low-order part of the address.

For example \opcode{move.b \$12345678,d4} takes 3 words of memory. The first defines the opcode, the second word defines the high half of the address \$1234 and, finally, the third word defines the low half of the address \$5678. The two words are read, and the byte, in this example, at that address is copied into \register{D4}.

\subsection{Program Counter Indirect with Displacement}
Mode 7 Submode 2. In this mode, the operand is in memory. The address of the operand is the sum of the address in the program counter (PC) and the sign-extended 16-bit displacement integer in the extension word. The `(PC)' part of the opcode can be left off as it is optional.

The value in the PC is the address of the extension word defining the offset. 

This is a program reference allowed only for reads.

For example, \opcode{lea jumptable(pc),a2} will set \register{A2} to the position independent location of the label \opcode{jumptable} no matter which address in RAM the code is running at. In memory, there are two words. The first defines the opcode, the second, which is where the Program Counter is pointing, is the displacement to the given label from the current address of the PC.

The example could also have been written as \opcode{lea jumptable,a2}

\subsection{Program Counter Indirect with Index (8-Bit Displacement)}
Mode 7 Submode 3. This mode is similar to the mode described in \emph{\nameref{sub-ARegII8}} on page~\pageref{sub-ARegII8} , except the PC is the base register. 

The operand is in memory.

The operand’s address is the sum of the address in the PC, the sign-extended displacement
integer in the extension word’s lower eight bits, and the sized, scaled, and sign-extended index operand. 

The value in the PC is the address of the extension word. 

This is a program reference allowed only for reads. 

The user \emph{must} include the displacement, the PC, and the index register when specifying this addressing mode.

For example \opcode{move.w jumptable(pc,d0.w*2),d0} could have been used in our jump table example above as it does not require the use of a base register to access the table to fetch the offset.

\subsection{Program Counter Indirect with Index (Base Displacement)}
Mode 7 Submode 3 again. This mode is similar to the mode described in \emph{\nameref{sub-ARegIIbd}} on page~\pageref{sub-ARegIIbd}, except the PC is used as the base register. 

It requires an index register indicator and an optional 16 or 32 bit sign-extended base displacement. 

The operand is in memory. 

The operand’s address is the sum of the contents of the PC, the base displacement, and the scaled contents of the sign-extended index register. 

The value of the PC is the address of the first extension word.

This is a program reference allowed only for reads.

For example \opcode{lea jumptable(PC,d0.w*2),a3} will work out the address of the \register{D0}th word in the table at label \opcode{jumptable} and copy it into \register{A3}.

In this mode, the PC, the displacement, and the index register are optional. The user must supply the assembler notation ZPC (a zero value PC) to show that the PC is not used. This allows the user to access the program space without using the PC in calculating the effective address. 

The user can access the program space with a data register indirect access by placing ZPC in the instruction and specifying a data register as the index register.

I have to admit that I'm not convinced that a PC or zero is going to be useful, certainly not in program independent code.

\subsection{Program Counter Memory Indirect Postindexed Mode}
Mode 7 Submode 3 also. This mode is similar to the mode described in \emph{\nameref{sub-MIPostI}} on page~\pageref{sub-MIPostI}, but the PC is the base register. 

Both the operand and operand address are in memory. 

The processor calculates an intermediate indirect memory address by adding a base
displacement to the PC contents. The processor accesses a long word at that address and
adds the scaled contents of the index register and the optional outer displacement to yield the effective address. 

The value of the PC used in the calculation is the address of the first
extension word. 

This is a program reference allowed only for reads.

In the syntax for this mode, brackets enclose the values used to calculate the intermediate memory address. All four user-specified values are optional. 

The user must supply the assembler notation ZPC (a zero value PC) to show the PC is not used. This allows the user to access the program space without using the PC in calculating the effective address. 

Both the base and outer displacements may be null, word, or long word. When omitting a
displacement or suppressing an element, its value is zero in the effective address
calculation.

For an example, see \emph{\nameref{sub-MIPostI}} on page~\pageref{sub-MIPostI} and replace the address register with `PC'.

\subsection{Program Counter Memory Indirect Preindexed Mode}
Mode 7 Submode 3 also again! This mode is similar to the mode described in \emph{\nameref{sub-MIPreI}} on \pageref{sub-MIPreI}, but the PC is the base register. 

Both the operand and operand address are in memory. 

The processor calculates an intermediate indirect memory address by adding the PC contents, a base displacement, and the scaled contents of an index register. The processor accesses a long word at immediate indirect memory address and adds the optional outer displacement to yield the effective address. 

The value of the PC is the address of the first extension word. 

This is a program reference allowed only for reads.

In the syntax for this mode, brackets enclose the values used to calculate the intermediate memory address. All four user-specified values are optional. The user must supply the assembler notation ZPC showing that the PC is not used. 

This allows the user to access the program space without using the PC in calculating the effective address. 

Both the base and outer displacements may be null, word, or long word. When omitting a displacement or suppressing an element, its value is zero in the effective address calculation.

For an example, see \emph{\nameref{sub-MIPreI}} on page~\pageref{sub-MIPreI} above and replace the address register with `PC'.


\subsection{Immediate Data}
Mode 7 Submode 4. In this addressing mode, the operand is in one or two extension words. 

For example, \opcode{move.l \#100,d0}. After this instruction has executed, \register{D0} will contain the value $100_{decimal}$ in all 32 bits.

That's it for the complete set of addressing modes. Next time, our exploration of the 68020 instructions will take a good look at the various Bit Field Instructions.






%----------------------------------------------------------------------------------------
% And the cover? What's nthat all about then?
%----------------------------------------------------------------------------------------
\chapter{Image Credits}
The front cover image on this ePeriodical is taken from the book \emph{Kunstformen der Natur} by German biologist Ernst Haeckel. The book was published between 1899 and 1904. The image used is of various \emph{Polycystines} which are a specific kind of micro-fossil.

I have also cropped the image for use on each chapter heading page.

You can read about Polycystines on \href{https://en.wikipedia.org/wiki/Polycystine}{Wikipedia} and there is a brief overview of the above book, also on \href{https://en.wikipedia.org/wiki/Kunstformen_der_Natur}{Wikipedia},
which shows a number of other images taken from the book. (Some of which I considered before choosing the current one!)

Polycystines have absolutely nothing to do with the QL or computing in general - in fact, I suspect they died out before electricity was invented - but I liked the image, and decided that it would make a good cover for the book and a decent enough chapter heading image too.

Not that I am suggesting, \emph{in any way whatsoever}, that we QL fans are ancient.

%----------------------------------------------------------------------------------------
%	BIBLIOGRAPHY - NOT REQUIRED
%----------------------------------------------------------------------------------------

%\chapter*{Bibliography}
%\addcontentsline{toc}{chapter}{\textcolor{ocre}{Bibliography}}
%\section*{Books}
%\addcontentsline{toc}{section}{Books}
%\printbibliography[heading=bibempty,type=book]
%\section*{Articles}
%\addcontentsline{toc}{section}{Articles}
%\printbibliography[heading=bibempty,type=article]

%----------------------------------------------------------------------------------------
%	INDEX
%
% To make an index in MikeTex:
%
% Compile main file with pdfLaTeX.
% Compile main file with makeIndex.
% Compile main file with pdfLaTeX again.
%
% This is useful, in the preamble:
%
% \DeclareRobustCommand{\ship}[1]{\textit{#1}\index{Steam ships!#1}}
% \DeclareRobustCommand{\AUports}[1]{\texttt{#1}\index{Austalia ports!#1}}
%
% Then just use \ship{ship_name} to index the ship name, under "ships" and
% to italicise it, as per Oxford Style Guide for shihp names.
% Australian Ports gets TeleType font in the main text.
%----------------------------------------------------------------------------------------

%\cleardoublepage
%\phantomsection
%\setlength{\columnsep}{0.75cm}
%\addcontentsline{toc}{chapter}{\textcolor{ocre}{Index}}
%\printindex

%----------------------------------------------------------------------------------------

\end{document}
