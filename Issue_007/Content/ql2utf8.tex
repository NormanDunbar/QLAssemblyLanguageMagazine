\chapter{Ql2utf8 Utility}

This utility is what I would need to use when I've saved a file on
the QL, or in QPC, and I need to transfer it down to the Linux box
for some processing - say, for example, to get the finished and tested
source code into an article like this one!

The utility is an example of a QL program which are collectively becoming
known as a ``YAF''.\footnote{Yet Another Filter!}

The utility reads a QL created text file, where the content is any
of the QL character set up to but not above, character 191 (\$BF)
which is the down arrow. Anything above that is a control character
and is unprintable - undefined results may occur if any are present
in the QL file.

It is executed in the usual manner:

\begin{lstlisting}[tabsize=4]
ex ram1_ql2utf8_bin, ram1_ql_txt, ram1_utf8_txt
\end{lstlisting}

The input file, ram1\_ql\_txt will be read in, and each byte converted
to the appropriate UTF8 byte sequence, and written out to the ram1\_utf8\_txt
file. The latter file will be used on my Linux box, but Windows and
MacOS users can also take advantage.

Right, enough waffle, on with the code.

\section{The Code}

As ever, my code starts with an introductory header and some equates.
This utility is no different as you can see below.

\begin{lstlisting}
;--------------------------------------------------------------------
; QL2UTF8:
;
; This filter converts QL text files to UTF8 for use on Linux, Mac or
; Windows where most modern editors etc, default to UTF8.
;
;
; EX ql2utf8_bin, input_file, output_file_or_channel
;
;--------------------------------------------------------------------
; 26/09/2019 NDunbar Created for QDOSMSQ Assembly Mailing List
;--------------------------------------------------------------------
; (c) Norman Dunbar, 2019. Permission granted for unlimited use
; or abuse, without attribution being required. Just enjoy!
;--------------------------------------------------------------------

; How many channels do I want?
numchans    equ     2       ; How many channels required?


; Stack stuff.
sourceId    equ     $02     ; Offset(A7) to input file id
destId      equ     $06     ; Offset(A7) to output file id

; Other Variables
pound       equ     96      ; UK Pound sign.
copyright   equ     127     ; (c) sign.
grave       equ     159     ; Backtick/Grave accent.
euro        equ     181     ; Euro symbol
err_bp      equ     -15
err_eof     equ     -10
me          equ     -1
timeout     equ     -1
\end{lstlisting}

The main entry point for the program is next. This section of code
contains the usual QDOS Job header and a few checks to ensure that
we only get a pair of channel IDs on the stack. If the user decided
to pass over a command string as well, it would be ignored.

\begin{lstlisting}
;====================================================================
; Here begins the code.
;--------------------------------------------------------------------
; Stack on entry:
;
; $06(a7) = Output file channel id.
; $02(a7) = Source file channel id.
; $00(a7) = How many channels? Should be $02.
;====================================================================
start       bra.s   checkStack

            dc.l    $00
            dc.w    $4afb
name        dc.w    name_end-name-2
            dc.b    'QL2UTF8'
name_end    equ     *

version     dc.w    vers_end-version-2
            dc.b    'Version 1.00'
vers_end    equ     *


bad_parameter
    moveq   #err_bp,d0      ; Guess!
    bra     errorExit       ; Die horribly


;--------------------------------------------------------------------
; Check the stack on entry. We only require NUMCHAN channels - any
; thing other than NUMCHANS will result in a BAD PARAMETER error on
; exit from EW (but not from EX).
;--------------------------------------------------------------------
checkStack
    cmpi.w  #numchans,(a7)  ; Two channels is a must
    bne.s   bad_parameter   ; Oops
\end{lstlisting}

Next up is some initialisation. In this short section of code, a couple
of registers are set to values which will be used throughout the entire
utility.

\begin{lstlisting}
;--------------------------------------------------------------------
; Initialise a couple of registers that will keep their values all
; through the rest of the code.
;--------------------------------------------------------------------
ql2utf8
    lea     utf8,a2         ; Preserved throughout
    moveq   #timeout,d3     ; Timeout, also Preserved

\end{lstlisting}

And now we have the top of the main loop for the program. We start
here by initialising the various registers to be able to read a single
byte from the input channel. The ID for that file is on the stack
at offset 2 from the current value in register A7.

Once a byte has been read we check the error code in D0, and if it
shows no errors, we can get on with the translation. If D0 is showing
an error, and it happens to be End Of File, we bale out of the program
and return success to SuperBASIC, Other errors will return the appropriate
error code to SuperBASIC, but that will only be seen if the utility
was executed with EXEC\_W or EW, or equivalent.

\begin{lstlisting}
;--------------------------------------------------------------------
; The main loop starts here. Read a single byte, check for EOF etc.
;--------------------------------------------------------------------
readLoop
    moveq   #io_fbyte,d0    ; Fetch one byte
    move.l  sourceID(a7),a0  ; Channel to readLoop
    trap    #3              ; Do input
    tst.l   d0              ; OK?
    beq.s   testBit7        ; Yes
    cmpi.l  #ERR_EOF,d0     ; All done?
    beq     allDone         ; Yes.
    bra     errorExit       ; Oops!
\end{lstlisting}

The first check is to test it bit 7 of the character just read, is
set or not. It it is set then the chances are that it is a multi-byte
character. If it is clear, then we continue processing.

\begin{lstlisting}
testBit7
    btst    #7,d1           ; Bit 7 set?
    bne.s   twoBytes        ; Multi Byte character if so
\end{lstlisting}

Right then, at this point the top bit must be clear, so we are looking
at a single byte character, or are we? The QL has a few little exceptions
to the rule as it uses different character codes to standard (if there
is such a thing) ASCII.

The first exception is the UK Pound sign, which is a two byte character
in UTF8. The code below checks and processes a Pound sign, if one
is found. After writing out the UTF8 codes, it loops back to the start
of the main loop, ready for the next character.

\begin{lstlisting}
;--------------------------------------------------------------------
; The UK Pound and copyright signs are exceptions to the "bytes 
; less than $80 are the same in UTF8 as they are in ASCII" rule as 
; Sir Clive didn't follow ASCII 100%. Both characters are multi-byte
; in UTF8.
;--------------------------------------------------------------------
testPound
    btst    #7,d1           ; Potential multi-byte character? 
    bne.s   twoBytes        ; Yes
    cmpi.b  #pound,d1       ; Got a UK Pound sign?
    bne.s   testCopyright   ; No.

gotPound
    move.b  #$c2,d1         ; Pound is $C2A3 in UTF8.
    bsr.s   writeByte       ; Write first byte
    move.b  #$a3,d1
    bsr.s   writeByte       ; Write second byte
    bra.s   readLoop
\end{lstlisting}

The next exception is the copyright symbol. It too is a multi byte
character in UTF8 so the code below checks for it and deals with it
appropriately.

\begin{lstlisting}
;--------------------------------------------------------------------
; Here we repeat the same check as above, in case we have the
; copyright sign.
;--------------------------------------------------------------------
testCopyright
    cmpi.b  #copyright,d1   ; Got a copyright sign?
    bne.s   oneByte         ; No.

gotCopyright
    move.b  #$c2,d1         ; Copyright is $C2A9 in UTF8.
    bsr.s   writeByte       ; Write first byte
    move.b  #$a9,d1
    bsr.s   writeByte       ; Write second byte
    bra.s   readLoop

\end{lstlisting}

That's all the QL characters that are exceptions to the ``ASCII characters
below code 128 are single byte in UTF8'' rule. The remaining QL characters
less than code 128 are dealt with by simply calling the routine to
write a single byte and then heading back to the top of the main loop.
Job done.

\begin{lstlisting}
;--------------------------------------------------------------------
; All other ASCII characters, below $80, are single byte in UTF8 and
; are the same code as in ASCII.
;--------------------------------------------------------------------
oneByte     
    bsr.s   writeByte       ; Single byte required in UTF8
    bra.s   readLoop
\end{lstlisting}

Speaking of writing a single byte, the following code does exactly
that. It fetches the channel ID for the output channel from the stack.
Normally, this would be at offset ``destId'' on from A7, but as
this code is always called as a subroutine, there is an extra 4 bytes
on the stack for the calling code's return address, so that has to
be considered.

All the following snippet has to do is set up the registers to enable
the trap call, IO\_SBYTE, to be called. D3, the timeout, is already
set to -1, and will be preserved on return, as will D2, which is being
used elsewhere in the code to safely hold a value during processing.

\begin{lstlisting}
;--------------------------------------------------------------------
; A small but perfectly formed subroutine to send the byte in D1 to
; the output channel.
; BEWARE: This is called with an extra 4 bytes on the stack!
;--------------------------------------------------------------------
writeByte
    moveq   #io_sbyte,d0    ; Send one byte
    move.l  4+destId(a7),a0  ; Output channel id
    trap    #3
    tst.l   d0              ; OK?
    bne.s   errorExit       ; Oops!
    rts
\end{lstlisting}

As mentioned above, we have processed all the QL characters that are
a single byte in UTF8, so now we need to think about those characters
with codes above 127, the majority of these are accented characters
and as the QL doesn't cover all the ``standard'' ones, there is
some ``furkling about''\footnote{That would be a technical term!}
to be done.

The QL wouldn't be the QL we know and love if there were not a couple
of exceptions to the rule that ``ASCII characters above code 128
are always multi-byte''. The grave (no, not somewhere you bury people,
the accent much loved by the French I believe) aka the backtick (at
least on Unix, Linux etc) is actually a single byte character in UTF8,
so that is dealt with first.

We arrive at the following code whenever a character is read in which
has the top bit, bit 7, set.

The code begins by checking for and processing a grave character.

\begin{lstlisting}
;--------------------------------------------------------------------
; ASCII codes from $80 upwards require multiple bytes in UTF8. In the
; case of the QL, these are mostly 2 bytes long. I could use IO_SSTRG
; here, I know. 
; However, as ever, there are exceptions. The grave accent (backtick)
; is a single byte on output, while the 4 arrow keys are three bytes.
; The bytes to be sent are read from a table because, again, the QL
; is not using the full set of accented characters - so there is
; mucking about to be done.
;--------------------------------------------------------------------
twoBytes
    cmpi.b  #grave,d1       ; Backtick/Grave accent?
    bne.s   testEuro        ; No.

;--------------------------------------------------------------------
; We are dealing with a backtick character (aka Grave accent)?
;--------------------------------------------------------------------
gotGrave
    move.b  #pound,d1       ; Grave in = pound out!
    bsr.s   writeByte       ; Single byte required
    bra     readLoop        ; Do the rest
\end{lstlisting}

From here on in we should be dealing with all the two byte characters
for UTF8, however, those exceptions are popping up again. The first
is the Euro symbol. This is missing from the original 128Kb QLs of
old, as the Euro didn't even exist when they were conceived, however,
in SMSQ, they have been allocated character 181 - which, when you
look at it in Pennel or similar, is a seriously weird character which
I've never seen used, so I think the SMSQ authors chose well!

In UTF8 the Euro needs three characters, \$E282AC, so the following
section of code does the necessary checking and handling of a Euro
character.

\begin{lstlisting}
;--------------------------------------------------------------------
; Here we repeat the same check as above, in case we have the
; Euro sign.
;--------------------------------------------------------------------
testEuro
    cmpi.b  #euro,d1        ; Got a Euro sign?
    bne.s   testArrows      ; No.

gotEuro
    move.b  #$e2,d1         ; Euro is $E282AC in UTF8.
    bsr.s   writeByte       ; Write first byte
    move.b  #$82,d1
    bsr.s   writeByte       ; Write second byte
    move.b  #$ac,d1
    bsr.s   writeByte       ; Write third byte
    bra.s   readLoop
\end{lstlisting}

Finally, in our exception handling code, the 4 arrow keys. These too
are three bytes long in UTF8, \$E2869x, where the 'x' nibble is 0,
1, 2 or 3 depending on the arrow's direction. Just to be awkward,
the QL's arrow order is different to UTF8 - on the QL the ascending
character codes are for the Left, Right, Up, Down arrows, but in UTF8
they are ordered Left, Up, Right, Down. 

The code snippet below handles the arrow keys.

\begin{lstlisting}
;--------------------------------------------------------------------
; The arrows are $BC, $BD, $BE and $BF (left, right, up, down). These
; are three bytes in UTF8, $E2 $86 $9x where 'x' is 0, 2, 1 or 3.
;--------------------------------------------------------------------
testArrows
    move.b  d1,d2           ; Copy character code
    subi.b  #$bc,d2         ; Anything lower = C set
    bcs.s   notArrows       ; And is not an arrow
    subi.b  #4,d2           ; Arrows = 0-3. C clear is bad
    bcc.s   notArrows       ; Still not an arrow.

gotArrows
    subi.b  #$bc,d1         ; D1 = 0 to 3 
    lea     arrows,a3       ; Arrow table
    move.b  d1,d2           ; Save index into table
    ext.w   d2              ; Need word not byte

    move.b  #$e2,d1         ; First byte
    bsr.s   writeByte
    move.b  #$86,d1         ; Second byte
    bsr.s   writeByte
    move.b  0(a3,d2.w),d1   ; Third byte
    bsr.s   writeByte     
    bra     readLoop        ; Go around again.

\end{lstlisting}

The arrow key's third byte is located in the following tiny table
which has the correct third byte for the appropriate arrow's code
on the QL.

\begin{lstlisting}
;--------------------------------------------------------------------
; We need this as arrows in the QL are Left, Right, Up, Down but in
; UTF8 they are Left, Up, Right, Down. Sigh.
;--------------------------------------------------------------------
arrows
    dc.b    $90,$92,$91,$93  ; Awkward byte order!
\end{lstlisting}

That is now, all the two byte exceptions catered for. The remainder
of the higher ASCII characters are all two bytes in size. Obviously,
being the QL, these are not in the same order as the originating ASCII
codes would be, had Sir Clive done the decent thing and used a standard
ASCII code page! Instead he chose to omit some characters and rearrange
the others into a non-standard order.\footnote{Ok, fair play, there probably wasn't a standard ASCII code page he
could use back then. }

The following code simply copies the character code from D1 to D2
and then manipulates D2 to go from an index into the table, to an
offset into the table where a pair of bytes can be found that represent
the UTF8 code for the current character. 

As we are dealing with character codes from 128 (\$80) onwards, we
start by subtracting \$80 from the character code. This gives the
correct index into the table. As each entry in the table is two bytes,
we double the index to get the correct offset, then pick up the two
bytes there and send them on their way to the output file, before
heading back to the start of the main loop.

\begin{lstlisting}[tabsize=4]
;--------------------------------------------------------------------
; Now we are certain, everything is two bytes. Read them from the
; table and write them out.
;--------------------------------------------------------------------
notArrows
    move.b  d1,d2           ; D2 = byte just read
    subi.b  #$80,d2         ; Adjust for table index
    ext.w   d2              ; Word size needed
    lsl.w   #1,d2           ; Double D2 for Offset
    move.b  0(a2,d2.w),d1   ; First byte
    bsr.s   writeByte       ; Send it output
    addq.b  #1,d2
    move.b  0(a2,d2.w),d1   ; Second byte
    bsr.s   writeByte       ; Send it out too
    bra     readLoop        ; Go around.
\end{lstlisting}

The code below is the usual tidy up and bale out code. It doesn't
require much explanation as you will have seen it before, many times.

\begin{lstlisting}
;--------------------------------------------------------------------
; No errors, exit quietly back to SuperBASIC.
;--------------------------------------------------------------------
allDone
    moveq   #0,d0

;--------------------------------------------------------------------
; We have hit an error so we copy the code to D3 then exit via a
; forcible removal of this job. EXEC_W/EW will display the error in
; SuperBASIC, but EXEC/EX will not.
;--------------------------------------------------------------------
errorExit
    move.l  d0,d3           ; Error code we want to return

;--------------------------------------------------------------------
; Kill myself when an error was detected, or at EOF.
;--------------------------------------------------------------------
suicide
    moveq   #mt_frjob,d0    ; This job will die soon
    moveq   #me,d1
    trap    #1
\end{lstlisting}

Finally, the table of two byte values for the multi-byte characters.
Those which have a word of \$0000 are exceptions, dealt with elsewhere.
And finally, the table only goes as far as character 191 (\$BF) as
everything that follows is unprintable and unlikely to ever get into
a QL text file. This basically means that if you do manage to do this,
the output will be ``undefined'' - as they say!

\begin{lstlisting}
;--------------------------------------------------------------------
; The following table contains the two byte sequences required for 
; QL characters above $80. These are all 2 bytes in UTF8, so quite a 
; simple case. (Not when converting UTF8 to QL though!)
;--------------------------------------------------------------------
utf8
    dc.w    $c3a4           ; a umlaut
    dc.w    $c3a3           ; a tilde
    dc.w    $c3a2           ; a circumflex
    dc.w    $c3a9           ; e acute
    dc.w    $c3b6           ; o umlaut
    dc.w    $c3b5           ; o tilde
    dc.w    $c3b8           ; o slash
    dc.w    $c3bc           ; u umlaut
    dc.w    $c3a7           ; c cedilla
    dc.w    $c3b1           ; n tilde
    dc.w    $c3a6           ; ae ligature
    dc.w    $c593           ; oe ligature
    dc.w    $c3a1           ; a acute
    dc.w    $c3a0           ; a grave
    dc.w    $c3a2           ; a circumflex
    dc.w    $c3ab           ; e umlaut
    dc.w    $c3a8           ; e grave
    dc.w    $c3aa           ; e circumflex
    dc.w    $c3af           ; i umlaut
    dc.w    $c3ad           ; i acute
    dc.w    $c3ac           ; i grave
    dc.w    $c3ae           ; i circumflex
    dc.w    $c3b3           ; o acute
    dc.w    $c3b2           ; o grave
    dc.w    $c3b4           ; o circumflex
    dc.w    $c3ba           ; u acute
    dc.w    $c3b9           ; u grave
    dc.w    $c3bb           ; u circumflex
    dc.w    $ceb2           ; B as in ss (German)
    dc.w    $c2a2           ; Cent
    dc.w    $c2a5           ; Yen
    dc.w    $0000           ; Grave accent - single byte
    dc.w    $c384           ; A umlaut
    dc.w    $c383           ; A tilde
    dc.w    $c385           ; A circle
    dc.w    $c389           ; E acute
    dc.w    $c396           ; O umlaut
    dc.w    $c395           ; O tilde
    dc.w    $c398           ; O slash
    dc.w    $c39c           ; U umlaut
    dc.w    $c387           ; C cedilla
    dc.w    $c391           ; N tilde
    dc.w    $c386           ; AE ligature
    dc.w    $c592           ; OE ligature
    dc.w    $ceb1           ; alpha
    dc.w    $ceb4           ; delta
    dc.w    $ceb8           ; theta
    dc.w    $cebb           ; lambda
    dc.w    $c2b5           ; micro (mu?)
    dc.w    $cf80           ; PI
    dc.w    $cf95           ; o pipe
    dc.w    $c2a1           ; ! upside down
    dc.w    $c2bf           ; ? upside down
    dc.w    $0000           ; Euro
    dc.w    $c2a7           ; Section mark
    dc.w    $c2a4           ; Currency symbol
    dc.w    $c2ab           ; <<
    dc.w    $c2bb           ; >>
    dc.w    $c2ba           ; Degree
    dc.w    $c3b7           ; Divide
\end{lstlisting}

