\chapter{Feedback on Issue 4}
\section{ASMReformat Comments}
Not long after I published \href{http://qdosmsq.dunbar-it.co.uk/downloads/AssemblyLanguage/Issue\_004/Assembly\_Language\_004.pdf}{Issue 4} I received some feedback from Wolfgang Lenerz regarding the code for the \program{ASMReformat}ASMReformat program listing, well, to be precise, the listing as published \emph{and} the output when run against a source file.

\textbf{WL:} Four traps aren't defined in the code (\opcode{io\_fline}, \opcode{io\_sbyte}, \opcode{io\_sstrg}, \opcode{mt\_frjob}) you might want to include the \opcode{equ} for these.

\textbf{ND:} I did wonder about this when I was writing the utility. \program{GWASS}GWASS includes a definitions file automatically so when I had those equates in the source, I got errors that they were duplicated, so I had to remove them. I was sure that \program{QMAC}QMAC did the same thing - but I haven't had to use that assembler since I started writing in QL Today all those years ago.

For anyone who needs them, here are the QDOS versions, from Pennell\footnote{\emph{The Sinclair QDOS Companion} A Guide to the QL Operating System.  \copyright{}~Andrew Pennell, 1985}:

\begin{verbatim}
io_fline    equ 2
io_sbyte    equ 5
io_sstrg    equ 7
mt_frjob    equ 5
\end{verbatim}


\textbf{WL:} I don't know about \program{GWasl}Gwasl, but \program{Qmac}Qmac doesn't allow, for the \opcode{equ} directive, the label and the content to be on different lines:

\begin{verbatim}
label equ something
\end{verbatim}

is ok

\begin{verbatim}
label 
      equ something
\end{verbatim}

is not ok....

\textbf{ND:} Hmm. I wasn't aware of this, but as mentioned above, it's been a long time! Perhaps, in a forthcoming issue, I could amend the utility to accept a parameter string that prevents this line split occurring. I'll look into it and see what I can come up with.


\textbf{WL:} Finally, for Qmac, you need simple "section x" at the start. 

\textbf{ND:} This is true, and I \emph{think} you also need an \opcode{END} at the end as well, if I remember correctly?

Anyway, I was assuming when I wrote the code that you would be passing your own code through the utility to get proper/my/standard/whatever formatting. so I would assume that you already have a \opcode{SECTION x} at the start (and an \opcode{END} at the end?) so they should be already there. Unless, of course, you mean that it doesn't work for those lines in a source file?

I've tested it on my own DJToolkit source, which was originally QMAC based, and it works fine.

