\chapter{Preface}
\section{Feedback}\label{section: feedback}
Please send all feedback to \url{assembly@dunbar-it.co.uk}. You may also send articles to
this address, however, please note that anything sent to this email address may be used in a future issue of the eMagazine. Please mark your email clearly if you do not wish this to happen.

This eMagazine is created in \LaTeX  source format, aka plain text with a few formatting commands thrown in for good measure, so I can cope with almost any format you might want to send me. As long as I can get plain text out of it, I can convert it to a suitable source format with reasonable ease. 

I use a Linux system to generate this eMagazine so I can read most, if not all, Word or MS Office documents, Quill, Plain text, email etc formats. Text87 might be a problem though!

\section{Subscribing to The Mailing List}
This eMagazine is available by watching for updates on the ql-users email list, \url{ql-users@q-v-d.com} of by subscribing to the QL Forum at \url{https://qlforum.co.uk} and watching out for new messages from me about available issues.

\section{Contacting The Mailing List}
I'm rather hoping that this mailing list will not be a one-way affair, like QL Today appeared to be. I'm very open to suggestions, opinions, articles etc from my readers, otherwise how do I know what I'm doing is right or wrong?

Sadly, George Gwilt has passed away and will no longer continue to keep me correct on matters where I get stuff completely wrong, as before with \emph{QL Today}. I know George did ask, way back when I proposed this eMagazine, if there would be a contact address, so I've set up an email address for the list, so that you can make comments etc as you wish. The email address is:

\url{assembly@qdosmsq.dunbar-it.co.uk}

Any emails sent there will find me. Please note, anything sent to that email address will be considered for publication, so I would appreciate your name at the very least if you intend to send something. If you do not wish your email to be considered for publication, please mark it clearly as such, thanks. I look forward to hearing from you all, from time to time.

If you do have an article to contribute, I'll happily accept it in almost any format - email, text, Word, Libre/Open Office odt, Quill, PC Quill, etc etc. Ideally, a \LaTeX source document is the best format, because I
can simply include those directly, but I doubt I'll be getting many of those! But not to worry, if you have something, I'll hopefully manage to include it.
