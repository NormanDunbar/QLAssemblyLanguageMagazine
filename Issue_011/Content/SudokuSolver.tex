
\chapter{Sudoku Solver}

This is the most brutal program I think I've ever written. It's a
Sudoku puzzle ``solver'' in as much as it will, eventually, find
a solution if a puzzle can be solved, but it does it by brute force
and complete ignorance of how a puzzle should be solved -- using
logic and deduction (induction?) -- and, it's recursive!

To put it mildly, it starts with a grid from which some entries are
missing, it looks for the one closest to the top left corner, and
tries the digits 1 to 9 in that location. If the digit can be put
there -- it's not already in the same row, column or small box --
then it effectively has a new grid to solve, and calls itself recursively,
to solve the new board. If it successfully places another digit, then
it recurses again and so on. Eventually though, the board will be
solved, or a digit will not be able to be placed. 

In the latter case, the recursive call will return -- \emph{backtrack}
-- to the calling code, reset it's variables, and try again with
a different starting digit. This will progress until the puzzle is
solved or every possible combination of digits for every possible
blank space, has been tried.

Now, you might think this will need a lot of stack space for the recursive
calls, but it needs $4+(20*depth)$ bytes of stack. By default, the
solver has a 4,000 byte data space, and so far, I have yet to blow
that out with any puzzles. That's enough for 199.8 recursive calls.

Did I mention, the code assembles to 1,028 bytes? And that includes
a demo puzzle and the ability to load games from disc?

The idea for this ``utility'' came from a \emph{You Tube} video\footnote{No, I don't have the URL I'm afraid, I didn't actually watch it --
I just saw the description.} that was recommended to me, for some reason, by You Tube's so called
AI, which seemed to think that I was interested in a Java Programming
series. I'm not and I wasn't, but one of the offered videos was a
Sudoku Solver in Java. So I stole the idea and wrote it in Assembly!

\section{Quick Explanation}

Picture a grid with three blank cells and the following explanation
extremely simplified. The solution will progress as follows:
\begin{enumerate}
\item Try a `1' in the first cell, %
\begin{tabular}{|c|c|c|}
\hline 
1 &  & \tabularnewline
\hline 
\end{tabular}, if it is legal move, recurse to solve the grid with 2 blank cells;
\item A `1' in the new first cell will obviously be illegal, from the
previous step, so try a `2', %
\begin{tabular}{|c|c|c|}
\hline 
1 & 2 & \tabularnewline
\hline 
\end{tabular}. If legal, recurse to solve the grid with a single blank cell;
\item Both `1' and `2' are illegal here, so all of the digits `3'
through `9' will be tried until one is legal. If so, the grid is
solved, %
\begin{tabular}{|c|c|c|}
\hline 
1 & 2 & 3\tabularnewline
\hline 
\end{tabular}, and the code returns through all of the recursive calls to the top
level, and finishes;
\item If none of the digits `3' through `9' are legal, %
\begin{tabular}{|c|c|c|}
\hline 
1 & 2 & X\tabularnewline
\hline 
\end{tabular}, then the grid is unsolved, so we return to the grid with two blank
cells and try the remaining numbers -- `3' through `9'. Starting
with `3', %
\begin{tabular}{|c|c|c|}
\hline 
1 & 3 & \tabularnewline
\hline 
\end{tabular}, and another recursive call to see if the grid with one blank cell
can now be solved;
\item If we find that we didn't get a solution to the two blank cells grid,
we return back to the grid with three blanks, and try the next number
that legally fits, %
\begin{tabular}{|c|c|c|}
\hline 
2 &  & \tabularnewline
\hline 
\end{tabular}, and recurse again to solve the new grid with two and then one blank
cells;
\item This process repeats until we end up with either a solution, we fitted
three legal digits in the three blank cells; or no solution -- the
grid is unsolvable.
\end{enumerate}

\section{The Solver}

To use the solver, simply EX/EXEC it and you will be presented with
a menu offering three choices:
\begin{itemize}
\item `Q' or `q' to quit;
\item ENTER to run the built in demonstration puzzle;
\item `L' or `l' to load a puzzle. You will be further prompted for
the file name which should contain 81 bytes of grid data -- the puzzle
only caters for 9 by 9 grids at present -- which can be a stream
of 81 bytes or separated into 9 lines of 9 bytes using linefeeds.
Digits are represented by themselves, obviously, and blanks are represented
by whichever character you wish to use. Linefeeds are ignored but
anything else that isn't a digit is deemed to be a space.
\item After a puzzle is solved, or not, you are required to press ENTER
to quit.
\end{itemize}
The load feature only accesses the first 81 bytes of data, ignoring
linefeeds, to obtain the grid data. You can fill the file with anything
you like after the first 81 bytes (or 90 if linefeeds are present).

The code repository for this issue contains a couple of example files
to load -- \texttt{Sudoku\_easy}, \texttt{Sudoku\_hard} and \texttt{Sudoku\_expert}
-- but don't be misled, those are simply the puzzle levels from the
Sudoku app on my tablet that I stole the grids from! In practice,
the easy version takes longer for the brute force algorithm to solve
than it does for the other two.

The code is written in plain 68008 instructions, there are no 68020
etc options used, so it will run on a bare bones QL. I have no idea
how efficiently it will run though. My QPC setup -- Linux under Wine,
64 bit dual core -- takes a few seconds to solve the demo and the
hard and expert loadable files, but longer of seconds to solve the
easy loadable puzzle. 

On with the code.

\section{The Solver Code}

The code is supplied in a format that makes it easy to assemble with
my preferred assembler, GWASS, but with a couple of very minor changes,
it can be assembled also by QMAC. The changes are just to remove two
comment markers, a semicolon form the job header and the end of the
file. All will be explained below.

\subsection{Equates}

At the start of the file, \texttt{SudokuSolver\_asm}, we begin with
a list of equates. Listing \ref{lis:equates} shows them all in gory
detail.

\lstinputlisting[linerange={1-36},caption={Equates},label={lis:equates}]{/home/norman/SourceCode/Assembly eMagazine/Issue_011/Code/Sudoku/SudokuSolver.asm}

Most of these are fairly constant, but you might wish to adjust \texttt{CON\_X}
and/or \texttt{CON\_Y} if you would like a different placement on
the screen.

\subsection{Job Header}

The standard job header is next and Listing \ref{lis:job-header}
shows it. The code is laid out in such a way that the ``end'' part
of the job name is followed by an alignment byte, if necessary, to
force the following address to be word aligned -- even -- if the
job name happens to be an odd number of characters. 

\lstinputlisting[linerange={45-58},caption={Job Header},label={lis:job-header}]{/home/norman/SourceCode/Assembly eMagazine/Issue_011/Code/Sudoku/SudokuSolver.asm}

The \texttt{equ {*}} is necessary on the same line as the label for
QMAC to be able to assemble the code without errors and is also the
reason why the job name is in single quotes -- QMAC doesn't appear
to like double quotes\footnote{Although a thought strikes me. I wonder if the double quotes I use
in Linux, where I type the source code, is a different character to
the QL double quotes? I must find out.} and throws errors.

You will notice a comment line, \texttt{section code}, near the start?
This is needed to be uncommented for QMAC. Delete the semicolon at
the start of the line if you are using the QMAC assembler.

\subsection{Main Control Code}

Following the job header, we have the main controlling code for the
application. Listing \ref{lis:main-control-code} has the details.

We begin by calling \texttt{openScreen} to open the screen console
we will be using, then printing the job name and the menu. On return
from the menu, which will only occur if the user chose not to quit,
we will start to solve whichever board we now have in memory. The
code at \texttt{solveBoard} does this, and calls itself recursively
as necessary in order to solve the puzzle.

\lstinputlisting[linerange={67-100},caption={Main Control Code},label={lis:main-control-code}]{/home/norman/SourceCode/Assembly eMagazine/Issue_011/Code/Sudoku/SudokuSolver.asm}

On return from the attempt to solve the board, we display a message
regarding the success or failure of the attempt, then wait for the
user to press ENTER and quit the program. The Z flag will be clear
if the board cannot be solved, or if the user pressed ESC. Z will
be set if the board was solved.

It would be a simple task to perhaps, make the code loop around and
display the menu again, allowing the user to choose to load and solve
another board? I leave this as homework and look forward to seeing
your results.

\subsection{Printing the Board}

It's no good solving, or not, the puzzle but never displaying the
results. Listing \ref{lis:print-board} is the \texttt{printBoard}
code which does exactly that, it prints out the board converting any
zeros in the board data into spaces and printing the vertical and
horizontal separators to attempt to make the output resemble a Sudoku
grid. All registers are preserved except \texttt{D0} and on entry,
the code expects register \texttt{A3} to be holding the address of
the start of the board data. The board is printed on every call to
\texttt{solveBoard}.

The code should be easy to follow. It starts by setting the cursor
position to row 2, column 0 on the screen, and using \texttt{D6} as
a loop counter for the 81 bytes of data to be processed. Each byte
is loaded into \texttt{D1} and if not a zero, must be a digit. The
binary value in \texttt{D1} is then converted to an ASCII digit by
adding the code for the ASCII character '0' and is printed using the
handy subroutine \texttt{printeByte}.

If the value in \texttt{D1} was a zero, we set \texttt{D1} to the
value of an ASCII space minus an ASCII zero. \$20 - \$30 in other
words and drop in to convert back to an ASCII space, before printing
it.

\lstinputlisting[linerange={114-168},caption={Printing the Puzzle Grid},label={lis:print-board}]{/home/norman/SourceCode/Assembly eMagazine/Issue_011/Code/Sudoku/SudokuSolver.asm}

After printing a digit or space, we check to see if we need a linefeed
after every 9th character printed. We copy \texttt{D6}, our cell counter,
to \texttt{D1} and divide by 9 and if the remainder is 8 we print
a linefeed. We are counting from 0 to 80, so need a linefeed after
the counter reached 8, 17, 26 and so on. After printing the linefeed
we don't bother checking for vertical separators as they never occur
at the end of a line, but we do need to consider the horizontal separators
every 3 rows. These occur when the cell counter is at value 26 or
53, and if so, we print the separator.

If a linefeed is not required, a vertical separator might be needed.
The cell counter is again copied to D1 but this time divided by 3.
If the remainder is 2, then we print a vertical separator otherwise,
we don't print anything else.

At \texttt{pbLoopEnd}, we increment the cell counter and if not finished,
skip back to the start of the loop to print the next cell's data.
If we are finished, we restore the registers we needed to preserve,
and return to wherever we were called from.

\subsection{Solving the Board}

Now we are ready to solve the board, and I use the term \emph{solve}
in its vaguest sense! The algorithm is as shown in listing \ref{lis:Solving-the-puzzle-pseudo-code},
which is pseudo code for what I figured out would probably work!

\begin{lstlisting}[caption={Solving the Puzzle Pseudo Code},label={lis:Solving-the-puzzle-pseudo-code},language={[Visual]Basic},showstringspaces=false,tabsize=4]
Def FN solveBoard
   local r,c,tryMe
;
    for (r = 0; r < gridSize; r++) {
        for (c = 0; c < gridSize; c++) {
            if (board[r][c] = 0) {
                for (tryMe = 1; tryMe <= gridSize; tryMe++) {
                    if (isValidHere() {
                        board[r][c] = tryMe;

                        if (solveBoard(board)) {
                            return true;
                        } else {
                            board[r][c] = 0;
                        }
                    } end if validHere
                } end for tryMe
                return false;
            } end if board[r][c] = 0
        } end for c
    } end for r
    return true

\end{lstlisting}

We scan each row, and then scan each column within that row. If the
board data holds a zero for this cell, then we have a blank cell for
which we need a value. We iterate through each digit from 1 to 9 and
check if it can be legally placed in this cell. If so, it is tried
and a recursive call made to solve the new board with one cell less
to find a value for.

If we have tried all the digits, and none fit, then the previous values
attempted for one or more cells is not correct, so we return \texttt{false}
to indicate failure, having set the cell data back to a zero.

The only way that we can return from this code having solved the board
is if we have iterated through all rows and all columns and managed
to legally place a digit there that caused all further recursive calls
to also find a legal digit. The puzzle is solved, so we return a \texttt{true}
value to indicate solved.

On return from a recursive call, if a \texttt{true} was detected,
we also return \texttt{true} back to our caller, unwinding the stack
as we go until we reach the top level.

How hard could it be to convert that into Assembly? Read on! Listing
\ref{lis:solve-board} shows the code I ended up with, after much
wailing an gnashing of teeth. I've added in a scan for the ESC key
so that an attempt to solve a puzzle can be aborted if something has
gone wrong and it's taking forever to find a solution.

If the scan returns with Z clear, then the ESC key was not pressed,
so we can continue, otherwise we simply return to the caller with
Z set to indicate a failure to solve the puzzle.

The code proper begins by saving our ``local'' variables, or registers
in this case. If we are called recursively, which we will be, then
we need to save the current values for row, column, and the digit
we are trying to fit in. We also save the offset into the board data,
register \texttt{D4}, as we need that to be restored before we return
back from this recursive call.

\texttt{D5} is the row counter, \texttt{D6} is the column counter
and \texttt{D7} is the value we are attempting to insert into the
current cell. \texttt{D4}, as mentioned, has the cell offset and is
used as a way of not having to calculate the offset each time for
each new row and column in the grid.

One problem, which I might look into, is the fact that every recursive
call begins looking for blank cells from the \emph{start} of the board.
This is inefficient as surely, we should be looking one cell onwards
from the cell we have just fitted a value into? More homework perhaps?
Anyway, the offset, row and column registers are initialised and we
loop along looking for a blank cell.

When we find one, \texttt{D7} is initialised to 1 and we start looping
around the values 1 to 9, checking each to see if it can legally be
placed in the cell. A legal placement is when the value under test
is not in the current row, as per \texttt{D5}, or column as per \texttt{D6}
or box, as per \texttt{D5},\texttt{D6}. If the value in \texttt{D7}
is legal, it is placed in the puzzle data at offset \texttt{D4}, the
board printed with the new value, and a recursive call made to \texttt{solveBoard}
to try and solve this new grid.

If we return with Z set, the board has been solved and we return to
our caller with Z set, otherwise, we reset the cell's data to zero
and try another number at \texttt{sbNextNumber}.

\lstinputlisting[linerange={186-245},caption={Solving the Puzzle Grid},label={lis:solve-board}]{/home/norman/SourceCode/Assembly eMagazine/Issue_011/Code/Sudoku/SudokuSolver.asm}

If we have run out of numbers to try, then we drop into \texttt{sbUnsolved}
where we give \texttt{D0} a value of 1, which clears the Z flag, indicating
a failure to solve the board, then we return to our caller via the
register restoration code at \texttt{sbReturn}.

\texttt{SbNextColumn} and \texttt{sbNextRow} simply increment the
column and row counters and skip back to the start of their respective
loops to allow the whole grid to be scanned for missing values and
attempts made to fill them.

If we manage to exit from the last row iteration, then the grid has
been solved. \texttt{D0} is set to zero, which sets the Z flag to
indicate a solution has been found, and we drop into \texttt{sbReturn}
to return to the caller.

\subsection{Is Digit in this Row?}

The code in Listing \ref{lis:in-row} is used to determine if the
digit in \texttt{D7} is legally able to be placed in the row in \texttt{D5}.
If so, we return with the Z flag set to indicate that the digit is
already present in the row. If Z is clear, then this is a valid placement
for the digit in the row. It might still be found in the column or
the box, but the row is fine.

\lstinputlisting[linerange={258-276},caption={Checking if a digit is in a row},label={lis:in-row}]{/home/norman/SourceCode/Assembly eMagazine/Issue_011/Code/Sudoku/SudokuSolver.asm}

The code saves the working registers -- all registers are preserved
except for \texttt{D0} -- and simply multiplies the row by 9 to get
the offset to the start of the row in \texttt{D5} which is then added
to \texttt{A3} to get the offset into the puzzle data. Yes, I \emph{could}
have used \texttt{MULU}, but I chose not to!

\texttt{D0} is then used as a counter of the number of columns to
be checked, and the loop executes 9 times. If the digit in \texttt{D7}
is found in the row, we will drop out of the bottom of the DBEQ loop
(decrement and branch \emph{unless}\footnote{Or, if you prefer, decrement and branch \emph{until} equal}
equal) with the Z flag set.

If we exit the loop with Z clear, we ran out of columns to check and
the digit is not present in the row. The working registers are restored
before we return to the caller.

\subsection{Is Digit in this Column?}

The code in Listing \ref{lis:in-column} is used to determine if the
digit in \texttt{D7} is legally able to be placed in the column in
\texttt{D6}. If so, we return with the Z flag set to indicate that
the digit is already present in the row. If Z is clear, then this
is a valid placement for the digit in the column. It might still be
found in the box, but the row and column, if we get this far, are
fine.

\lstinputlisting[linerange={288-305},caption={Checking if a digit is in a column},label={lis:in-column}]{/home/norman/SourceCode/Assembly eMagazine/Issue_011/Code/Sudoku/SudokuSolver.asm}

The code saves the working registers -- all registers are preserved
except for \texttt{D0} -- and simply adds the column number in \texttt{D6}
to the start of the puzzle data in \texttt{A3}, to get the offset
to the first cell of the column number in \texttt{D6}. 

\texttt{D0} is then used as a counter of the number of rows in this
column to be checked, and the loop executes 9 times. If the digit
in \texttt{D7} is found in the column, we will drop out of the bottom
of the DBEQ loop with the Z flag set.

If we exit the loop with Z clear, we ran out of rows to check and
the digit is not present in the column. The working registers are
restored before we return to the caller.

\subsection{Is Digit in this Box?}

This one is slightly trickier then the previous two checks. We need
to check if the digit in \texttt{D7} is present in the small, 3 by
3 box, containing row \texttt{D5} and column \texttt{D6}. Listing
\ref{lis:in-box} shows how it is done.

The code begins by saving the working registers as before. We preserve
all registers except \texttt{D0} in this code and return with the
Z flag set to indicate that the number is in the box, or clear if
not. \texttt{D0} is cleared as we need to work with all 32 bits for
the \texttt{DIVU}, and then the row number is copied over. The remainder
of dividing the row number by 3 is then subtracted from the row number
to get the top row number in this box. So we have $TopRow=Row-(Row\ Mod\ 3)$.

\texttt{D5}, the row number, is again copied to \texttt{D0} and multiplied
by 9 to get the offset to add to \texttt{A3} for the start of the
puzzle data for the row in \texttt{D5}.

\lstinputlisting[linerange={317-352},caption={Checking if a digit is in a box},label={lis:in-box}]{/home/norman/SourceCode/Assembly eMagazine/Issue_011/Code/Sudoku/SudokuSolver.asm}

In a similar vein, we need to get the starting column for the column
number in \texttt{D6}. The same division is applied and the remainder
subtracted from the column number to get the starting column for the
box. Again, we have $TopColumn=Column-(Column\ Mod\ 3)$.

This value is added to \texttt{A3} to get the correct offset to the
starting cell in the 3 by 3 box. \texttt{D5} and \texttt{D6} are now
redundant, but are used to count rows and columns -- and for some
reason, I appear to have mixed them up, but it makes no difference,
my apologies for any confusion caused --- and a double \texttt{DBRA}
loop entered to check each of the 9 cells in the box. As soon as the
digit is found, we skip to \texttt{ibDone} to restore the working
registers and return with the Z flag set. If we exit both loops with
no sign of the digit, we return with the Z flag cleared.

\subsection{Subroutines}

All the code so far described constitutes the meat of the code to
solve a puzzle. There are, a number of helpful subroutines that are
called from one or more places, and those are described in this section.

\subsubsection{Print a Byte}

Listing \ref{lis:sub-print-byte} is a small routine to print the
byte in \texttt{D1} to the channel in \texttt{A0}. All registers are
preserved and the Z flag is set if all went well, or cleared if errors
were detected. 

\lstinputlisting[linerange={364-371},caption={Subroutine: Print a byte},label={lis:sub-print-byte}]{/home/norman/SourceCode/Assembly eMagazine/Issue_011/Code/Sudoku/SudokuSolver.asm}

After preserving the working registers, the \texttt{IOB.SBYT} trap
call is used to send \texttt{D1.B} to channel \texttt{A0}. This requires
a timeout in register \texttt{D3.W} and we use -1 as the timeout to
indicate that the trap call can take as long as it wants. 

\subsubsection{Position the Cursor}

In order to preserve the heading for the console channel, where the
job name is printed, we set the cursor to row 2 column 0, and the
clear from there to the end of the console. Positioning the cursor
is done using the subroutine at label \texttt{at}. Listing \ref{lis:sub-at}
shows the code.

\lstinputlisting[linerange={380-389},caption={Subroutine: Position the cursor at 2,0},label={lis:sub-at}]{/home/norman/SourceCode/Assembly eMagazine/Issue_011/Code/Sudoku/SudokuSolver.asm}

All registers except \texttt{D0} are preserved so the code begins
by stacking the ones it must keep safe. The \texttt{IOW.SCUR} trap
call is then set up with \texttt{D1} indicating the row column number
and \texttt{D2} the row number. \texttt{D3}, as before, is the timeout
for the trap.

On error, \texttt{D0} will hold the error code and the Z flag will
be cleared. If all went well, \texttt{D0} will be zero and the Z flag
will be set.

\subsubsection{Printing Strings}

To print a string, we use the code in Listing \ref{lis:sub-print-string}.
As before, everything except \texttt{D0} is preserved so we begin
by saving the registers that will be corrupted.

\lstinputlisting[linerange={399-404},caption={Subroutine: Printing strings},label={lis:sub-print-string}]{/home/norman/SourceCode/Assembly eMagazine/Issue_011/Code/Sudoku/SudokuSolver.asm}

The \texttt{UT.WTEXT} vectored routine is set up next, and this expects
the channel id to be in \texttt{A0} and a pointer to the string to
be printed in \texttt{A1}. The vector is loaded into \texttt{A2} and
executed. After restoring the working registers, we return with the
Z flag already set to indicate success or failure. \texttt{D0} will
contain any error codes.

\subsubsection{Opening the Console}

Listing \ref{lis:sub-console} is the console definition block, and
the code -- at \texttt{openScreen} -- to open the console, set the
ink, paper and strip colours, and apply any desired border settings.

\lstinputlisting[linerange={412-426},caption={Subroutine: Console definition and open code},label={lis:sub-console}]{/home/norman/SourceCode/Assembly eMagazine/Issue_011/Code/Sudoku/SudokuSolver.asm}

The console definition block is set up using some constants defined
way back at the start of the code, as shown previously in Listing
\ref{lis:equates}. You might wish to amend the \texttt{CON\_X} and
\texttt{CON\_Y} settings in Listing \ref{lis:equates} to move the
screen over a bit, if necessary on your system.

When this code is called, the application is in its starting state,
so we don't really care about the registers used or corrupted. The
console is opened and attributes applied using the \texttt{OPW.CON}
vectored utility which expects \texttt{A1} to be pointing at the definition
block.

On exit from \texttt{openScreen}, the Z flag will be set if all was
well, otherwise it will be clear and an error code will be in \texttt{D0}.

\subsubsection{The Menu}

Well, I say ``menu'' but it's not much of a menu -- there's no
vegetarian or vegan options for starters\footnote{Or indeed, for main course either!}.
Listing \ref{lis:sub-menu} shows the code, which doesn't care about
preserving registers other than \texttt{A0} which holds the console
channel id. Everything else is deemed expendable.

The menu is displayed starting at line 2, column 0 and offers the
three options to run the demo; load a puzzle; or quit. After displaying
the options, the code reads the keyboard waiting for some input. \texttt{A1}
is returned from \texttt{waitForInput} pointing at the start of the
buffer, where the length of the text is stored. We need the first
character in the buffer so the byte at \texttt{2(A1)} is loaded into
\texttt{D2} and bit 5 set to convert to lower case.

\texttt{D1.W} holds the size of the text that was entered by the user,
so if that was 1, we know the user pressed ENTER only and we are done
-- the demo puzzle is already loaded and nothing more needs to be
done.

\lstinputlisting[linerange={442-470},caption={Subroutine: Menu},label={lis:sub-menu}]{/home/norman/SourceCode/Assembly eMagazine/Issue_011/Code/Sudoku/SudokuSolver.asm}

Given that we now know some text was entered, we can check if \texttt{D2.B}
is 'q' and if so, exit the application via the code at label \texttt{die},
which terminates the application.

If the character in \texttt{D2.B} is not an 'l' (lower case L) then
we redisplay the menu. If it was an 'l' then we call out to \texttt{loadGame}
to process loading a different puzzle to solve. If Z was set on return,
all was well and we are done here, otherwise, we indicate that there
was a loading problem and exit. In this case, the inbuilt demo will
be solved.

Just prior to returning from this subroutine, the screen is once more
cleared from 2,0 across and downwards -- only the job name remains
untouched!

\subsubsection{Clearing Bits of the Screen}

As mentioned a few times already, we want to keep the job name printed
at the top of the screen. To do this we have to use two trap calls.
The code is shown in Listing \ref{lis:sub-cls-screen} and preserves
\texttt{A0} and \texttt{D3}. \texttt{D0}, \texttt{D1} and \texttt{A1}
are corrupted.

\lstinputlisting[linerange={480-488},caption={Subroutine: Clearing bits of the screen},label={lis:sub-cls-screen}]{/home/norman/SourceCode/Assembly eMagazine/Issue_011/Code/Sudoku/SudokuSolver.asm}

The first trap call is \texttt{IOW.CLRL} and will clear the screen
from the cursor position to the end of the line. After this call,
\texttt{IOW.CLRB} is used to clear all of the screen below the cursor
position.

\subsubsection{Loading Games}

Puzzle files are assumed to contain ASCII representations of a grid
and should contain at least 81 bytes representing the grid. This can
be split as your wish desires, into separate more manageable lines,
using linefeeds as and when required -- those are always ignored.
Any character which is not a linefeed or an ASCII digit, is considered
a space for a cell to be filled in.

Loading a puzzle preserves all registers except \texttt{D0}. The Z
flag will be set if the puzzle loaded and cleared otherwise.

Listing \ref{lis:sub-load-game-prompting-the-user} shows the code
executed when the user chooses to load a game from the menu. It begins
by setting the cursor position and clearing the screen before prompting
the user for a file name, then waits for the user to enter one. If
the user simply presses ENTER, we loop around and prompt again.

\lstinputlisting[linerange={502-509},caption={Subroutine: Loading a puzzle - prompting the user},label={lis:sub-load-game-prompting-the-user}]{/home/norman/SourceCode/Assembly eMagazine/Issue_011/Code/Sudoku/SudokuSolver.asm}

Once we have a potential filename, we drop into the code in Listing
\ref{lis:sub-load-game-open-file} where we attempt to open the file. 

\lstinputlisting[linerange={511-523},caption={Subroutine: Loading a puzzle - opening the file},label={lis:sub-load-game-open-file}]{/home/norman/SourceCode/Assembly eMagazine/Issue_011/Code/Sudoku/SudokuSolver.asm}

We use the \texttt{IOA.OPEN} trap call which requires an owning job
id in \texttt{D1} -- I'm using -1 for the current job -- an open
mode in \texttt{D3}, here I'm using the value 1 for OPEN\_IN so the
file must exist. Unfortunately, \texttt{A0} is required to point at
the string containing the filename, so I have to save the console
id in \texttt{A4}, which is not used or corrupted by the file opening
code. It's quicker to use a register than to stack one! \texttt{A1}
currently points at the filename, so that's copied to \texttt{A0}
and the trap executed. If the file opened, we skip to \texttt{lgReadFile}
to read the puzzle data, otherwise we display a message and start
again by heading back to Listing \ref{lis:sub-load-game-prompting-the-user}.

Listing \ref{lis:sub-load-game-setup} is next, and here we set up
the registers that will not change in the data reading loop. \texttt{D3}
is the timeout and as usual, I'm going infinite! \texttt{A3} is the
puzzle data start address for cell 0,0 and \texttt{D4} holds the count
of bytes we want to read from the file, there should be 81 bytes in
a puzzle. As with all \texttt{DBcc} loops, we load one less than we
need.

\lstinputlisting[linerange={525-528},caption={Subroutine: Loading a puzzle - setting up the read},label={lis:sub-load-game-setup}]{/home/norman/SourceCode/Assembly eMagazine/Issue_011/Code/Sudoku/SudokuSolver.asm}

At label \texttt{lgReadLoop} in Listing \ref{lis:sub-load-game-reading-data},
we see the start of the loop to read the 81 bytes of puzzle data from
the file. We use the \texttt{IOB.FBYT} trap call to read a single
byte at a time into register \texttt{D1}. If we hit any errors, including
EOF, before the load is complete, the program aborts -- perhaps not
the best idea in the world? Who fancies adding better error handling
then? More homework?

If the character in \texttt{D1} is a linefeed, we ignore it and return
to read the next character without affecting the counter of bytes
still to be read and processed in \texttt{D4}.

\lstinputlisting[linerange={530-551},caption={Subroutine: Loading a puzzle - reading data},label={lis:sub-load-game-reading-data}]{/home/norman/SourceCode/Assembly eMagazine/Issue_011/Code/Sudoku/SudokuSolver.asm}

At \texttt{lgDigits}, we test the character to see if it's an ASCII
digit, and if so, we subtract the ASCII code for a '0' from \texttt{D1}.
If it's not a digit, we just set \texttt{D1} to zero at \texttt{lgNonDigit}
and in both cases, end up at \texttt{lgStoreByte} where we write the
byte in \texttt{D1} into the puzzle grid at \texttt{A3}, incrementing
\texttt{A3} to point at the next cell's data space.

After storing the byte, the counter is decremented and if there is
more to do, we skip back to read another byte. If we have read all
81 bytes from the file, then we close it using \texttt{IOA.CLOS} at
label \texttt{lgCloseFile}, in Listing \ref{lis:sub-load-game-close-file}.

\lstinputlisting[linerange={553-558},caption={Subroutine: Loading a puzzle - closing the data file},label={lis:sub-load-game-close-file}]{/home/norman/SourceCode/Assembly eMagazine/Issue_011/Code/Sudoku/SudokuSolver.asm}

In QDOS, closing a file which was not open results in a ``File not
open'' error, but in SMSQ/E, there is no error reported. The code
in Listing \ref{lis:sub-load-game-close-file} assumes the latter
case and clears \texttt{D0} to indicate no errors.

The console is is copied back from \texttt{A4} to \texttt{A0} before
we return to the calling code, as we will be writing to the console
soon and need it!

It should be noted that loading a game makes no attempt to determine
if the data are valid. You could load a puzzle with repeating characters
in rows, columns and/or boxes, it doesn't care. What happens in this
case? The code sill still try to solve the puzzle, but will run for
quite some time, and will have to be aborted, probably, when you get
fed up!

\subsubsection{Reading Input From the Keyboard}

Listing \ref{lis:sub-get-user-input} shows the code used to obtain
input from the user. Any time we call this subroutine, the user is
expected to press ENTER. This character will be saved as a linefeed
in the input buffer and the count of characters obtained, in \texttt{D1},
will include the linefeed.

\lstinputlisting[linerange={571-586},caption={Subroutine: Getting user input},label={lis:sub-get-user-input}]{/home/norman/SourceCode/Assembly eMagazine/Issue_011/Code/Sudoku/SudokuSolver.asm}

We use the \texttt{IOB.FLIN} trap call to fetch a line of input. \texttt{D2}
tells the trap how big the buffer is and sets the maximum number of
bytes than can be returned. \texttt{A1} points at the buffer start
and this is copied to \texttt{A4} as we need to store the length word
there. \texttt{A1} is incremented by 2 to point at the position where
we want the data to be read into. \texttt{D3} already holds the timeout.

After the trap call, if there were errors, we skip back and attempt
to get input from the user again. If no errors were detected, the
length word is stored in the start of the buffer but then decremented
to delete the trailing linefeed character, which we know is definitely
there. \texttt{A4} is copied to \texttt{A1} to return the start of
the buffer to the caller.

If the user types more characters than will fit in the buffer, then
an overflow error will be returned and the final character in the
buffer will not be the linefeed. We don't specifically trap this error
in the code above as we simply reattempt in the case of errors being
detected.

The code expects \texttt{A0} to be the console id and trashes just
about everything else. \texttt{A1} is returned pointing to the start
of the buffer.

\subsubsection{Scanning for ESC}

I mentioned that puzzle data are not validated when loaded. If a bad
data set is loaded, then the code could run for some time and we need
a manner of aborting a run. The code in Listings \ref{lis:sub-scan-esc}
and \ref{lis:sub-keyrow-esc} carry out a quick scan of the keyboard,
looking for the ESC key.

Please note, this is the same as calling \texttt{KEYROW} in S{*}BASIC
and it's possible that pressing ESC in another application will be
detected by the Sudoku solver, and abort it's current run. Note also,
Listing \ref{lis:sub-keyrow-esc} is out of line here, it's at the
end of the file in reality, but is listed here as it pertains to this
subroutine.

\lstinputlisting[linerange={597-604},caption={Subroutine: Scanning for ESC},label={lis:sub-scan-esc}]{/home/norman/SourceCode/Assembly eMagazine/Issue_011/Code/Sudoku/SudokuSolver.asm}

Listing \ref{lis:sub-scan-esc} saves all registers. Only the Z flag
is affected and is set if ESC was not pressed. We use the \texttt{SMS.HDOP}
trap call to run a hardware operation to scan the keyboard for ESC,
but we can also detect ENTER, SPACE, '\textbackslash ' and the 4
arrow keys with the parameters in Listing \ref{lis:sub-keyrow-esc}which
are pointed at by register \texttt{A3}. After the trap, bit 3, the
ESC key bit, is tested and if set, ESC was pressed. If ESC was not
pressed, the bit will be clear and as such, the Z flag will be set.

\lstinputlisting[linerange={697-698},caption={Subroutine: KEYROW command for ESC},label={lis:sub-keyrow-esc}]{/home/norman/SourceCode/Assembly eMagazine/Issue_011/Code/Sudoku/SudokuSolver.asm}

Listing \ref{lis:sub-keyrow-esc} is the parameter set required by
\texttt{SMS.HDOP} to scan \texttt{KEYROW(1)} on the keyboard.

\subsubsection{The Demo Board}

If the user simply presses ENTER at the menu, then the inbuilt demo
puzzle will be solved. Listing \ref{lis:sub-demo-puzzle} shows the
puzzle's data bytes where a zero is used to indicate an empty cell.
Note that the data are \emph{not} ASCII, they are binary values.

\lstinputlisting[linerange={613-624},caption={Subroutine: Demo puzzle data},label={lis:sub-demo-puzzle}]{/home/norman/SourceCode/Assembly eMagazine/Issue_011/Code/Sudoku/SudokuSolver.asm}

There is an alignment check after the puzzle's data as 81 bytes would
leave a dangling odd address. This is not a problem on 68020 emulators,
but will cause no end of problems for the old bare bones QL, so is
best avoided.

The puzzle data is shown here as separate rows of 9 bytes each, but
there are no linefeeds in the data, so it is simply a long string
of 81 bytes of data. Don't let the way it's printed confuse how it
is actually stored.

This 81 byte area is where the main code looks when solving a puzzle.
If the user loads a puzzle, it will be loaded here too, overwriting
the demo data.

\subsubsection{Messages}

Various messages used throughout the application are shown in Listings
\ref{lis:sub-grid-h-sep} and \ref{lis:sub-messages}. Listing \ref{lis:sub-grid-h-sep}
is the horizontal separator used by the \texttt{printBoard} subroutine.

\lstinputlisting[linerange={633-637},caption={Subroutine: Grid horizontal separator},label={lis:sub-grid-h-sep}]{/home/norman/SourceCode/Assembly eMagazine/Issue_011/Code/Sudoku/SudokuSolver.asm}

Listing \ref{lis:sub-messages} contains all the other messages.

\lstinputlisting[linerange={643-691},caption={Subroutine: Messages},label={lis:sub-messages}]{/home/norman/SourceCode/Assembly eMagazine/Issue_011/Code/Sudoku/SudokuSolver.asm}
