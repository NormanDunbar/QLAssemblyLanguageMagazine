
\chapter{News}

\section{Beginners Corner}

In Beginners Corner this time, we take a look at simple file and screen
handling. The code to open a text file and to display it on screen
is provided with explanations in excruciating detail!

Buffer overruns are handled safely and without affecting the program's
output. EOF is a little more tricky in Assembly than in S{*}BASIC,
and this is -- hopefully -- well explained too.

\section{Sudoku Solver}

In this issue, the main article is all about solving Sudoku puzzles
using the QL and Assembly Language. This demonstrates recursion and
backtracking and uses a brute force ``algorithm'' to solve the puzzles. 

Well, I say ``solve'' but in reality, it stuffs a number into the
first blank cell it finds, and then tries, recursively, to solve the
new puzzle which has one less blank cell! There's no finesse about
it, it just hammers away until it -- eventually -- solves the puzzle,
or doesn't if the puzzle can't be solved.

\section{Multiplication}

After many years of avoiding the matter, I've finally decided to look
into how the QL could do multiplication of two 32 bit numbers to give
a 64 bit result. It turns out to be a lot less mind numbing than I
thought it would be. So, anyway, if I had to figure it out for myself,
you might as well profit from my trials and tribulations!

In this issue, there's a whole chapter about multiplying two 32 bit
\emph{unsigned} values to give a 64 bit result. Then I figured out
that I might as well add an extra bit of code to make signed number
multiplication ``easy'' as well. That turned out to be only a few
more lines.

\section{The QL Wiki}

For some time now, there has been a problem uploading and/or including
graphics on the \href{https://qlwiki.qlforum.co.uk/}{QL Wiki}\footnote{https://qlwiki.qlforum.co.uk}
in that sometimes it works, but on others, it barfs with an ``Error
403, Forbidden'' message. This affects me as well, and I have administrator
privileges on the Wiki.

Investigations are ongoing, and sadly, have been for quite some time.
We are of the opinion that the host company has done something and
barfed things for the Wiki. Rob Heaton is \emph{still} attempting
to get some form of support out of the hosts, but so far, this is
proving quite elusive, unfortunately.

This is not a good state of affairs to be in, but at the moment, we
are working on it as best we can, given the limitations of the hosts.
They did apparently resolve it a while back, but it has reared its
ugly head once again, but this time, we seem to have stumped them.

If you are trying to insert images into your Wiki pages, and are suffering
from this problem, please be patient. There is a \href{https://qlforum.co.uk/viewforum.php?f=26}{Wiki Topic}\footnote{https://qlforum.co.uk/viewforum.php?f=26}
on the QL Forum where you can report problems with images. Just to
keep prodding us admin types!

Occasionally I am able to get files uploaded, with a bit of hassle
from the ``403 Forbidden'' errors, but it can, eventually be done.
If you need to get some images uploaded, let me know on the above
link and I'll try my best to get them done for you.

\section{RIP GWASL}

My most used assembler these days is GWASS from George Gwilt which
allows me to assemble MC68020 instructions as well as those for the
MC68008. In the past I have used, and occasionally still do, George
Gwilt's other fine assembler, GWASL. This one is only for the MC68008
processor used in the original QL. 

Unlike its big brother, GWASS, GWASL throws errors if a ``blank''
line in the source code is not completely blank. If there are any
unprintable characters -- spaces or tabs etc -- GWASL will record
an ``illegal instruction'' error when it hits one of these lines.

In addition, GWASL doesn't like symbols with a dot in the name. So
\texttt{OPW.CON}, for example, is rejected. This means that it's not
possible to use the correct names for the SMSQ/E symbols and equates.

Because of this, GWASL, which I have used since 2009, is sadly being
retired. I do have the source code for GWASL but looking through it,
George has used some of his own libraries and macros for various parts
and I don't have those files. I \emph{might} do a disassembly on the
current binary, to see if I can fix it. No promises though!
