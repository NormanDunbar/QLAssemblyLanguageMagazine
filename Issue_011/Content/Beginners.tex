
\chapter{Beginners Corner}

\section{Introduction}

In this issue of Beginners Corner, we are taking a look at file and
simple screen handling.

\section{Reading Files}

This code in S{*}BASIC looks simple enough, and \emph{should} be able
to be converted quite simply to assembly, so let's do it.

\begin{lstlisting}[language={[Visual]Basic},showstringspaces=false,tabsize=4]
OPEN #3,"ram1_test_txt" 
REPeat readLoop 
  IF EOF(#3) Then EXIT readLoop: END IF
  a$ = INPUT(#3)
  PRINT a$
END REPeat readLoop 
CLOSE #3
\end{lstlisting}


\subsection{Converting to Assembly}

The following steps are required in the assembly language source code:
\begin{itemize}
\item Decide whether the code will be \texttt{CALL}ed or \texttt{EXEC}'d.
\item Set up a buffer to hold the data we read from the file. How big should
we make it? How long is a piece of string\footnote{Silly question! It's obviously twice as long as half a piece of string!}?
\item Open a channel to the screen. This can be a \texttt{CON\_} or a \texttt{SCR\_}
channel. \texttt{SCR\_} is most appropriate as we don't need to fetch
any input from it.
\item We can set the paper, strip and ink colours as required. The default
green on black isn't very nice on my old eyes, so that's got to go!
\item Clear the screen channel.
\item Open the file and handle errors if it's not found.
\item Loop around, reading text from the string and printing it, exiting
when end of file is detected.
\item Close the file at the end of the loop.
\item Exit back to S{*}BASIC.
\end{itemize}
As I prefer setting up and using separate jobs, I'll make this a job
and it will be \texttt{EXEC}'d.

\section{Code Walk-through}

As this is Beginner's Corner, the following code fragments will be
explained on excruciating detail. Experienced coders should probably
look away now!

The full code is in the code repository for this issue.

\subsection{Preliminaries}

The code file, \texttt{ReadingFiles\_asm}, begins with some constants
-- \emph{equates} -- which make life easier if we have to make changes
at some future point. They also have reasonably explanatory names,
which helps document the code. Listing \ref{lis:ReadingFiles-constants}
is the constants used in defining the screen channel we wish to use,
and a couple more to define buffer sizes, to indicate the current
job identifier and an infinite timeout.

\lstinputlisting[linerange={19-37},caption={ReadingFiles - Constants},label={lis:ReadingFiles-constants}]{/home/norman/SourceCode/Assembly eMagazine/Issue_011/Code/BeginnersCorner/ReadingFiles_asm}

Following on from the program constants, we have a list of the various
trap and vector codes to enable us to access routines within SMSQ/E
and QDOS. Even though I'm using the SMSQ/E names, that code will still
work on QDOS as they are the same as QDOS. Listing \ref{lis:ReadingFiles-smsq-e-constants}
shows the list of SMSQ/E constants we need in this program.

\lstinputlisting[linerange={43-53},caption={ReadingFiles - SMSQ/E Constants},label={lis:ReadingFiles-smsq-e-constants}]{/home/norman/SourceCode/Assembly eMagazine/Issue_011/Code/BeginnersCorner/ReadingFiles_asm}

And now the code proper begins. As I have decided to create a separate
job (or task) for this demonstration, we need a standard job header.
Listing \ref{lis:ReadingFiles-job-header} is the necessary code for
just about any job in SMSQ/E.

\lstinputlisting[linerange={60-71},caption={ReadingFiles - Job header},label={lis:ReadingFiles-job-header}]{/home/norman/SourceCode/Assembly eMagazine/Issue_011/Code/BeginnersCorner/ReadingFiles_asm}

At label \texttt{Start}, the code skips the header and branches off
to the \texttt{openScreen} label. Following that we have a 4 byte
filler followed by the word \$4AFB which SMSQ/E uses to recognise
a job header. Following the flag word, we have the job's name in the
form of a standard SMSQ/E string with a leading byte counter followed
by the bytes of the string.

The strings length word might look a bit strange, however, we simply
take the address of label \texttt{jobNameEnd}, subtract the current
address (that of \texttt{jobName}) from it, then subtract another
2. This gives me the length of the text making up the job name. It
also means that I can change the job name at any time and not have
to worry about counting characters!

\textbf{Tip}: The '{*}' used in this way, tells the assembler to use
the current address.

The last line in Listing \ref{lis:ReadingFiles-job-header} simply
makes sure that the next address will be even if the job name contained
an odd number of characters. It reserves zero words of space, but
because this is word sized, the assembler forces the current address
to an even address if necessary.

That completes the preliminaries of the job. Next we move on to the
code proper, opening the screen channel.

\subsection{Opening the Screen Channel}

Listing \ref{lis:ReadingFiles-screen-definition} is the definition
of the screen channel we wish to open. It consists of 4 bytes defining
colours and border sizes, followed by 4 words defining the screen
channel size. The values used here were defined as constants in Listing
\ref{lis:ReadingFiles-constants} and can be changed there if necessary.

\lstinputlisting[linerange={84-92},caption={ReadingFiles - Screen definition},label={lis:ReadingFiles-screen-definition}]{/home/norman/SourceCode/Assembly eMagazine/Issue_011/Code/BeginnersCorner/ReadingFiles_asm}

The definition block is used by the \texttt{OPW.SCR} vectored routine
to open a channel to the screen, set the paper, strip and ink colours,
as well as setting the border colour and width. Listing \ref{lis:ReadingFiles-open-screen-channel}
is the code to do exactly that.

\lstinputlisting[linerange={102-107},caption={ReadingFiles - Open screen channel},label={lis:ReadingFiles-open-screen-channel}]{/home/norman/SourceCode/Assembly eMagazine/Issue_011/Code/BeginnersCorner/ReadingFiles_asm}

The \texttt{OPW.SCR} vector requires the address of the definition
block in register \texttt{A1} and when executed will return the channel
id in register \texttt{A0} -- most channel functions use \texttt{A0}
for the channel id -- and there will be an error code, or zero, in
\texttt{D0}. If there were no errors, the \texttt{Z} flag will be
set and we test this to ensure we can continue.

In the event of errors, we exit the program via the \texttt{die} routine,
described later. Otherwise, we copy the screen channel id into register
\texttt{A5} as the code coming later to print to the screen will require
the screen channel id in \texttt{A0} and at that time, \texttt{A0}
will be holding the text file channel id.

\subsection{Clearing the Screen}

After opening the channel, we should clear the screen to get rid of
any existing detritus that might be in the space where we opened the
channel. Listing \ref{lis:ReadingFiles-clearing-the-screen-channel}
shows the code to clear the entire screen channel. SMSQ/E allows different
parts of the screen to be cleared, but we need to clear all of it.

\lstinputlisting[linerange={124-130},caption={ReadingFiles - Clearing the screen channel},label={lis:ReadingFiles-clearing-the-screen-channel}]{/home/norman/SourceCode/Assembly eMagazine/Issue_011/Code/BeginnersCorner/ReadingFiles_asm}

This code requires the screen channel in register \texttt{A0}, where
it currently is, and a timeout in register \texttt{D3}. We use an
infinite timeout in this case as we wish the action to complete before
returning to our code. Most SMSQ/E trap calls preserve the timeout
in \texttt{D3}, but when we open the text file, and when we print
data to the screen, \texttt{D3} will be corrupted, so we take a copy
into register \texttt{D6} for safety.

In the unlikely event of errors being detected when clearing the screen,
we exit the job via the routine at label, \texttt{die}.

\subsection{Opening the Text File}

Now the good stuff! We get to open the text file named \texttt{ram1\_test\_txt},
which must exist. In order to do this, we need to have certain registers
set as follows:
\begin{itemize}
\item The trap code, \texttt{IOA.OPEN} should be in \texttt{D0};
\item A job id, for the job to own the channel, is in \texttt{D1};
\item The open mode is in \texttt{D3};
\item \texttt{A0} points at the standard SMSQ/E string defining the job
name;
\end{itemize}
There are a number of open modes that we can use when opening a file:
\begin{itemize}
\item 0 = \texttt{OPEN} - the file specified must exist;
\item 1 = \texttt{OPEN\_IN} - the file specified must exist;
\item 2 = \texttt{OPEN\_NEW} - the file specified should not exist;
\item 3 = \texttt{OPEN\_OVER} - the file specified may, or may not exist;
\item 4 = \texttt{OPEN\_DIR} - opens an existing directory.
\end{itemize}
We are using the \texttt{OPEN\_IN} mode in this example, so the file
must exist. This is handy as we are intending to read and display
its contents!

Listing \ref{lis:ReadingFiles-opening-the-text-file} shows the code
that opens the file. In the event of errors, the job simply exits
in the usual manner, via \texttt{die}.

\lstinputlisting[linerange={141-148},caption={ReadingFiles - Opening  the text file},label={lis:ReadingFiles-opening-the-text-file}]{/home/norman/SourceCode/Assembly eMagazine/Issue_011/Code/BeginnersCorner/ReadingFiles_asm}

On return from the trap, the channel ID will be in register \texttt{A0}.
This has of course overwritten the screen channel id that we have,
but as we kept a copy of that in \texttt{A5}, it's not a problem.
Listing \ref{lis:ReadingFiles-text-file-definition} shows the text
file's definition in the code. This is slightly out of line, as the
definition is just below the screen definition block in Listing \ref{lis:ReadingFiles-screen-definition}
but is documented here for clarity.

\lstinputlisting[linerange={94-100},caption={ReadingFiles - Text file definition},label={lis:ReadingFiles-text-file-definition}]{/home/norman/SourceCode/Assembly eMagazine/Issue_011/Code/BeginnersCorner/ReadingFiles_asm}

As with the job name, we calculate the size of the file name to ensure
we get it correct, and so that we can easily change the file name,
if necessary, without having to keep counting characters.

Once we have the file open, we can start a loop to read the contents.

\subsection{Reading a Line of Text}

Listing \ref{lis:ReadingFiles-reading-a-buffer-of-text} is the code
which reads a buffer full of text from the text file, into our program.
How big is the buffer? How do we know?

In S{*}BASIC it's quite simple, just read a line of text into a string
variable. S{*}BASIC will read as many characters as it can into the
string, until it hits EOF or reads a linefeed. Our code has a fixed
buffer size, and I've made it deliberately small in this demonstration,
to show how a long line can be read without causing buffer overflows
and perhaps, corrupting the job code in the process. Normally, a buffer
would be big enough -- whatever that means.

\lstinputlisting[linerange={164-171},caption={ReadingFiles - Reading a buffer of text},label={lis:ReadingFiles-reading-a-buffer-of-text}]{/home/norman/SourceCode/Assembly eMagazine/Issue_011/Code/BeginnersCorner/ReadingFiles_asm}

As mentioned above, opening the text file corrupted \texttt{D3}, so
we copy it back from the safe place, register \texttt{D6}. The registers
required to read a line of text, terminated by a linefeed, from a
file, are:
\begin{itemize}
\item The trap code, \texttt{IOB.FLIN}, should be in \texttt{D0};
\item \texttt{D2} should hold the buffer size -- the maximum number of
characters we can read;
\item \texttt{D3} is the timeout;
\item \texttt{A0} is the channel id;
\item \texttt{A1} points at the location in the buffer where the text will
be copied into.
\end{itemize}
As we intend to print the text we read from the file, we set \texttt{A1}
to be two bytes higher than the start of the buffer. This leaves space
to store the string length word prior to printing. This converts the
buffer into a standard SMSQ/E string and makes for easy printing. 

On return from the trap, we test for errors and if none were found,
we skip to label \texttt{setString}, where we store the data length
in the two empty bytes at the beginning of the buffer, to make the
data into a string, ready to print. On return from this trap, the
following registers are set:
\begin{itemize}
\item \texttt{D0} has the error code, or zero;
\item \texttt{D1.W} has the size of the data successfully read, which includes
the linefeed, if one is present;
\item \texttt{A1.L} points to the first byte after the last byte of data
read, within the buffer.
\end{itemize}
Listing \ref{lis:ReadingFiles-input-buffer-definition} is again,
out of line with the actual code file, it comes at the very end, but
it is the buffer definition and is relevant to this discussion.

\lstinputlisting[linerange={253-254},caption={ReadingFiles - Input buffer definition},label={lis:ReadingFiles-input-buffer-definition}]{/home/norman/SourceCode/Assembly eMagazine/Issue_011/Code/BeginnersCorner/ReadingFiles_asm}

The size of the buffer is defined in our constants, way back in Listing
\ref{lis:ReadingFiles-constants}. I've deliberately kept it small
for this demonstration, as I previously mentioned.

In the event that we hit an error when reading the text file, what
do we do? Listing \ref{lis:ReadingFiles-text-file-error-handling}
shows the error handling code.

\lstinputlisting[linerange={172-178},caption={ReadingFiles - Text file error handling},label={lis:ReadingFiles-text-file-error-handling}]{/home/norman/SourceCode/Assembly eMagazine/Issue_011/Code/BeginnersCorner/ReadingFiles_asm}

If \texttt{D0} had been zero, then we would have successfully read
a line of text, terminated by a linefeed, which was smaller (or equal)
to the maximum buffer size allowed. 

If, on the other hand, no linefeed was found before we filled the
buffer, then we should receive \texttt{ERR.BFFL} on return from the
trap. This indicates that we filled the buffer. This is not a proper
error, it just means we have an extra long line in the file which
is bigger than the buffer. We can still print it, so we can skip to
\texttt{setString}.

The same is true if we detected \texttt{ERR.OVFL}. This is something
that Wolfgang Lenerz advised me to test for, in the last issue, as
sometimes SMSQ/E (or QDOS?) can return this error instead of \texttt{ERR.BFFL}.
The action is the same, it's not a proper error, we just filled the
buffer, so we can skip to \texttt{setString} ready to print the data.

The next error we need to test for is end of file, \texttt{ERR.EOF}.
If the end of file has been reached, did we read any data into the
buffer? By testing \texttt{D1.W} we can tell. If there was no data
read, \texttt{D1.W} will be zero and the Z flag will be set by the
test. In this case, we skip to \texttt{suspendJob} as there's nothing
else to do.

In assembly code, hitting EOF will raise the error \emph{even if}
some data were read into the buffer. In S{*}BASIC, the data would
be returned with no errors and the following read of the file would
raise the EOF error.

If there was data in the buffer, we still need to process it as normal,
so we drop into Listing \ref{lis:ReadingFiles-buffer-conversion-to-string}
where we set up the string, ready to be printed. This is described
next. After printing, however, we will skip back to the start of the
read loop, \texttt{readFile}, but this time, we will get the \texttt{ERR.EOF}
error again but \texttt{D1.W} will be zero, so we have a get out of
the loop clause!

\subsection{Printing the Text}

In order to print the string, we need the buffer to be in the style
of a standard SMSQ/E string. This requires the string length to be
in a word at the start of the string. When we read the data from the
file, it was written to the third byte in the buffer, leaving room
at the start for a word to hold the string (data) length. 

On return from the \texttt{IOB.FLIN} trap, register \texttt{D1.W}
has the number of bytes read. Listing \ref{lis:ReadingFiles-buffer-conversion-to-string}
shows that word being stored at the start of the buffer, converting
the buffer into a string, ready to print. 

\lstinputlisting[linerange={184-186},caption={ReadingFiles - Buffer conversion to a string},label={lis:ReadingFiles-buffer-conversion-to-string}]{/home/norman/SourceCode/Assembly eMagazine/Issue_011/Code/BeginnersCorner/ReadingFiles_asm}

In the event that a linefeed was detected, printing the data will
also print the linefeed, starting a new line in the screen channel
for the next line of text. If no linefeed was found, the data will
be printed, but no linefeed means that printing will begin again at
the next space in the screen channel. This way, long lines can be
printed correctly as each buffer full of text is simply printed next
to the previous one, until a linefeed is found.

Listing \ref{lis:ReadingFiles-printing-the-data} prints the buffer
as a standard string.

\lstinputlisting[linerange={197-203},caption={ReadingFiles - Printing the data},label={lis:ReadingFiles-printing-the-data}]{/home/norman/SourceCode/Assembly eMagazine/Issue_011/Code/BeginnersCorner/ReadingFiles_asm}

We currently have the channel id for the text file in \texttt{A0}
and we need the screen channel id for printing. As we kept a copy
of that channel id in register \texttt{A5}, we can swap the two channel
ids over. This means that \texttt{A0} ends up with the screen channel
id, and \texttt{A5} the text file id.

We then load \texttt{A2} with the vector for the \texttt{UT.WTEXT}
routine, which expects the following parameters:
\begin{itemize}
\item \texttt{A0} is the channel id;
\item \texttt{A1} is the start of the string to be printed, the length word.
\end{itemize}
After the code executes, \texttt{D1-D3} and \texttt{A1} are corrupted.
\texttt{D0} will hold zero or an error code. If errors are detected,
we bale out of the job via the \texttt{die} routine. Otherwise, we
swap the two channel ids back again, and skip back to the start of
the reading loop to get the next line of text from the file.

\subsection{Suspending the Job}

This section of the code just delays closing the files for a few seconds
so that the user can see the screen channel has the full text copied
from the text file. It pauses for 300 frames -- in the UK/Europe
there are 50 frames per second, in the USA it's 60. This gives a 6
second delay in the UK/Europe, and 5 seconds in the USA. To suspend
a job, you need the following register settings:
\begin{itemize}
\item The trap code, \texttt{SMS.SSJB} is in \texttt{D0};
\item The job id is in \texttt{D1};
\item The number of frames to suspend for is in \texttt{D3};
\item \texttt{A1} is set to the address of a byte which will be cleared
when the job's suspension is done and the job is active again. It
can be zero if not required.
\end{itemize}
On exit from the suspension, register \texttt{A0} is set to point
at the job's header, while \texttt{A1} is preserved. 

Now, in my copy of Pennel, I have a hand written note saying \emph{no
it bloody isn't!} so I'm assuming I hit a problem some time back and
noted it. Who is right? The documentation or ``past'' me? My motto
is \emph{Don't think, find out} so let's go! QMON2, here we come....

Setting a breakpoint at \texttt{suspendJob}, and running the job breaks
at that label. If I set \texttt{A1} to zero, then it is preserved
after the suspension is over. So far, so good. What if \texttt{A1}
was not zero? Well, I tried that too -- and it was also preserved.
I'm running on SMSQ/E so maybe the problem I noted was in QDOS and
has been fixed in SMSQ/E. 

Anyway, Listing \ref{lis:ReadingFiles-suspending-the-job} shows the
job suspension code.

\lstinputlisting[linerange={210-215},caption={ReadingFiles - Suspending the job},label={lis:ReadingFiles-suspending-the-job}]{/home/norman/SourceCode/Assembly eMagazine/Issue_011/Code/BeginnersCorner/ReadingFiles_asm}

\subsection{Closing Files}

While it is not strictly required to close files when finished with
them, it's considered good practice and should be done in applications
where a file is opened and read once, perhaps at the start, but never
used afterwards. In those jobs, holding the file open wastes resources,
and can prevent the file from being edited to make changes. 

The best advice is to keep files open only as long as you need to.
However, I don't follow my own advice at times, so don't do as I do,
do as I say! 

Listing \ref{lis:ReadingFiles-closing-all-files} shows the code to
close both files. As we are doing the same thing more than once, we
use a subroutine to close one file, and simply call it twice with
the text file id, then with the screen channel id.

\lstinputlisting[linerange={224-227},caption={ReadingFiles - Closing all files},label={lis:ReadingFiles-closing-all-files}]{/home/norman/SourceCode/Assembly eMagazine/Issue_011/Code/BeginnersCorner/ReadingFiles_asm}

The first file to be closed is the text file, simply because we have
the channel id in register \texttt{A0}, we call the subroutine \texttt{closeFile}
to do the actual closing. Closing a file doesn't usually return errors.
QDOS did return \texttt{ERR\_NO} (SMSQ/E uses \texttt{ERR.ICHN}) for
Channel not open, but SMSQ/E doesn't bother, and doesn't return an
error.

On return, we swap \texttt{A0} and \texttt{A5} to get the screen channel
id into \texttt{A0}, and head off to close that channel too.

The code to close one file is shown later, in Listing \ref{lis:ReadingFiles-closing-one-file}

\subsection{Done!}

Nearly done. The job has closed all its open files, and is ready to
exit. We pass any error codes from \texttt{D0} to \texttt{D3}, as
is required, and call the \texttt{SMS.FRJB} trap call to kill the
job and return the error code to S{*}BASIC. The parameters required
are:
\begin{itemize}
\item The trap code, \texttt{SMS.FRJB} is in \texttt{D0};
\item The error code to return to S{*}BASIC is in \texttt{D3}.
\end{itemize}
Obviously, this trap call never returns.

\lstinputlisting[linerange={234-238},caption={ReadingFiles - Death of a job},label={lis:ReadingFiles-death-of-a-job}]{/home/norman/SourceCode/Assembly eMagazine/Issue_011/Code/BeginnersCorner/ReadingFiles_asm}

The error code will be reported by S{*}BASIC if, and only if, the
job was started with \texttt{EXEC\_W}, \texttt{EW} or any similar
function which executes the job, then waits for it to complete.

Jobs started with \texttt{EXEC}, \texttt{EX} or similar, cannot report
the jobs exit code as \texttt{EXEC}, \texttt{EX} etc return immediately
after starting the job, they have no way of knowing how or if the
job ended as some jobs can run ``for ever'' such as clocks. S{*}BASIC
has no way of knowing how the job ended.

\subsection{Subroutine(s)}

There is a single subroutine in this issue's code. The code to close
a single file. Listing 

\ref{lis:ReadingFiles-closing-one-file} shows the code. Closing a
file requires these parameters:
\begin{itemize}
\item The trap code, \texttt{IOA.CLOS} in \texttt{D0};
\item The channel id in \texttt{A0}.
\end{itemize}
SMSQ/E returns no errors in \texttt{D0} but QDOS does, if the file
is not open, QDOS will return \texttt{ERR.ICHN}, which in QDOS terminology
is ERR\_NO.

\lstinputlisting[linerange={246-251},caption={ReadingFiles - Closing one file},label={lis:ReadingFiles-closing-one-file}]{/home/norman/SourceCode/Assembly eMagazine/Issue_011/Code/BeginnersCorner/ReadingFiles_asm}

\section{Assembling the Code}

If I were to assume that you have downloaded \href{http://www.dilwyn.me.uk/asm/gwassp22.zip}{GWASS}\footnote{http://www.dilwyn.me.uk/asm/gwassp22.zip}
for QPC2 and other 68020 based emulators, or are using the Quanta
version of the GST Qmac assembler for QLs with a 68008 processor,
and have the code saved as \texttt{ram1\_ReadingFiles\_asm}, then
assembling the code is as simple as this:

\subsection{With GWASS}
\begin{itemize}
\item \texttt{EXEC} \texttt{gwass60\_bin} to start the assembler;
\item Select the option to start assembling;
\item Type in the filename: \texttt{ram1\_ReadingFiles\_asm}
\item Wait.
\end{itemize}

\subsection{With Qmac}

To pass the commands directly via the command line:
\begin{itemize}
\item \texttt{EX qmac;``ram1\_readingFiles\_asm -data 2048 -filetype 1 -nolink
}~\\
\texttt{-bin ram1\_readingFiles\_bin''}
\end{itemize}
\textbf{Note:} The above command should be typed on one line - I've
had to split it for the PDF page width. 

Alternatively, you can type the command interactively:
\begin{itemize}
\item \texttt{EX qmac}
\item Type the options: \texttt{ram1\_readingFiles\_asm -data 2048 -filetype
1 -nolink }~\\
\texttt{-bin ram1\_readingFiles\_bin}
\item Wait
\end{itemize}
What you are doing here, in both cases, is telling the assembler to:
\begin{itemize}
\item Assemble the source file \texttt{ram1\_ReadingFiles\_asm};
\item Create an executable file (\texttt{-filetype 1}), with 2,048 bytes
of data space (\texttt{-data 2048});
\item Do not invoke the linker as it is not needed because everything is
in the same source file (\texttt{-nolink});
\item Create the output file named \texttt{ram1\_ReadingFiles\_bin} --
which will be in uppercase regardless of what you type here!
\end{itemize}

\subsection{Executing the Code}

After a successful assemble, and regardless of which assembler you
used, \texttt{ram1\_ReadingFiles\_bin} will be the executable job.
To run it:
\begin{itemize}
\item \texttt{EX} \texttt{ram1\_ReadingFiles\_bin}
\end{itemize}
On a successful execution, a small window will open, with black paper,
white ink and a white, one pixel border. Some lines of text will be
displayed, and after a delay of about 5 or 6 seconds, the screen channel
will be closed and vanish.

And finally, Figure \ref{fig:ReadingFiles-final-output} shows what
the final output should look like.

\begin{figure}[!h]
\begin{centering}
\includegraphics{\string"/home/norman/SourceCode/Assembly eMagazine/Issue_011/images/ReadingFiles\string".png}
\par\end{centering}
\caption{ReadingFiles final output\label{fig:ReadingFiles-final-output}}

\end{figure}


\section{Assembler Differences}

There are a couple of minor differences between assemblers, as you
can see, just invoking them is quite different.
\begin{itemize}
\item Qmac doesn't have the ability to assemble instructions for the 68020
processor, only for the 68008 as used in the original QL. Some emulators
use a virtual 68020 but these can still execute code assembled for
the 68008, the converse is not true though.
\item Qmac doesn't like filenames etc in double quotes, it expects single
quotes. \texttt{``ram1\_test\_txt''} is rejected with the double
quotes flagged as invalid characters, but \texttt{'ram1\_test\_txt'}
is accepted.
\item GWASS allows code without sections, but Qmac requires a header and
trailer as follows: 
\begin{lstlisting}
    section code

    ; Your code here

    end
\end{lstlisting}

There needs to be a section with a name at the top, and an end at
the bottom. You \emph{can} have lots of sections, with different names
but the above is the minimum necessary. GWASS doesn't need this.
\item In Qmac (and GWASS), you can tell it how much data space it requires:
\begin{lstlisting}
    section code
    data 8192

    ; Your code here

    end
\end{lstlisting}
\item Qmac has a command line option which can be used in make files, GWASS
uses a non-standard method of passing filenames:
\begin{lstlisting}
ex gwass60_bin; "A/1/filename_1_asm"
\end{lstlisting}
 which tells GWASS to assemble 1 file, named \texttt{filename\_1\_asm}.
\item Qmac doesn't like this sort of thing:
\begin{lstlisting}
label 
    equ *
\end{lstlisting}
as it complains about the label being invalid. This is a bit of a
bind as I use that a lot to show the end of a string and to save me
counting. However, I can do this instead:
\begin{lstlisting}
label equ *
\end{lstlisting}
 which is accepted, and makes sure anything following is word aligned.
\end{itemize}

\section{Whither GWASL?}

This was mentioned already in the News section of this issue, but
I'm repeating it here in Beginners Corner.

In the past I have used, and occasionally still do, George Gwilt's
assembler, GWASL, for the MC68008 processor used in the original QL.
Unfortunately, because GWASL throws errors whenever there is a blank
line which contains one or more spaces or tabs or other non-printing
characters; and cannot cope with the SMSQ/E symbols containing a dot
in their name, GWASL is sadly now being retired.

Because of this, I've resorted to using QMAC for the MC68008 processors
and GWASS for the bigger, MC68020, processors. QMAC has no problems
with those ``non-blank'' blank lines or SMSQ/E names..

\section{Summary}

So that's the Assembly Language version of how to open a file, read
it and display the contents on the screen, before closing it.

In future issues, I'll be continuing to delve into a few more of these
with, hopefully, enough explanation for beginners to get started with.

Get hold of the SMSQ/E Reference Manual from \href{https://www.wlenerz.com/qlstuff/\#qdosms}{Wolfgang Lenerz's web site}\footnote{https://www.wlenerz.com/qlstuff/\#qdosms}
for the official version. Alternatively, there are copies on Dilwyn's
pages:
\begin{itemize}
\item \href{http://www.dilwyn.me.uk/docs/manuals/QDOS_SMS\%20Reference\%20Guide\%20v4.5.pdf}{Here}\footnote{http://www.dilwyn.me.uk/docs/manuals/QDOS\_SMS\%20Reference\%20Guide\%20v4.5.pdf}
for the PDF for version 4.5; or
\item \href{http://www.dilwyn.me.uk/docs/manuals/QDOS_SMS\%20Reference\%20Guide\%20v4.5.odt}{Here}\footnote{http://www.dilwyn.me.uk/docs/manuals/QDOS\_SMS\%20Reference\%20Guide\%20v4.5.odt}
for the ODT (Libre Office) file for version 4.5.
\item If you want to ensure that you have the most recent versions of those
files, \href{https://www.wlenerz.com/qlstuff/\#qdosms}{Wolfgang Lenerz's web site}\footnote{https://www.wlenerz.com/qlstuff/\#qdosms}
is the place to look.
\end{itemize}
Get hold of GWASS \href{http://gwiltprogs.info/gwassp22.zip}{here}\footnote{http://gwiltprogs.info/gwassp22.zip}
for 68020 processors.

Download Qmac \href{http://www.dilwyn.me.uk/asm/gst/gstmacroquanta.zip}{here}\footnote{http://www.dilwyn.me.uk/asm/gst/gstmacroquanta.zip}
for 68008 processors.
