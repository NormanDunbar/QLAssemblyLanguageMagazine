
\chapter{32 Bit Multiplication}

I shall confess straight away, I am not a mathematician, nor do I
play one on TV. There may therefore be some weirdness in this chapter.
Feel free to correct or update me!

I don't know about you, but when I see code to do multiplication (or
worse, division) in Assembly Language, I tend to gloss over and skip
the meaty stuff. For some reason, I get a mental block. So I decided
to do something about it and find out if I could make myself understand
it.

Obviously, I'm starting simple with an \emph{unsigned} 32 bit by 32
bit multiplication routine which takes two registers as the inputs,
\texttt{D0} and \texttt{D1}, and returns the product in \texttt{D3}
and \texttt{D2} with \texttt{D3} being the high 32 bits.

\section{The MC68008 and MC68020}

I am aware that the MC68020 already has a 32 bit by 32 bit multiply
instruction, before anyone starts writing an email! The MC68008 only
has a 16 bit by 16 bit multiply instruction, and this code is, after
all, a learning experience for me. And if I have to learn it, then
don't think you are getting off so easily!

\section{The Theory}

In decimal, if we want to multiply the values 24 and 32, we don't
usually do the multiplication in one calculation of $24*32$ -- we
split it up into manageable chunks and perform four multiplications
and one addition, of the four results, to get the final product:

\begin{align*}
2*4 & =8\\
2*2*10 & =40\\
3*10*4 & =120\\
3*10*2*10 & =600\\
8+40+120+600 & =768
\end{align*}

In general, we end up with the formula:

\[
\left(U_{1}*U_{2}\right)+\left(U_{2}*T_{1}*Base\right)+\left(T_{2}*Base*U_{1}\right)+\left(T_{1}*Base*T_{2}*Base\right)
\]

which can be rearranged into the following:

\[
\left(U_{1}*U_{2}\right)+\left(U_{2}*T_{1}*Base\right)+\left(T_{2}*U_{1}*Base\right)+\left(T_{1}*T_{2}*Base^{2}\right)
\]

Where:
\begin{itemize}
\item `$T_{1}$' and `$T_{2}$' represent the ``tens'' digits of the two
numbers;
\item `$U_{1}$' and `$U_{2}$' represent the ``units'' digits of the
two numbers;
\item `Base' is the base of the number system in use, 10 in this example.
\end{itemize}
Substituting our two values, 24 and 32, into the equation, we get:

\[
\left(2*4\right)+\left(2*2*10\right)+\left(3*4*10\right)+\left(3*2*100\right)=768
\]

So far so good! It seems to work perfectly. Can we do the same thing
with our 32 bit register values? Yes we can! If you consider that
a 32 bit register value is really just $D_{high}*65,536+D_{low}$
then we can surely substitute these values into our equation. The
number base is therefore 65,536 in this case. As multiplying by 65,536
is simply a 16 bit shift left, then we end up with the following equation:

\[
\left(D0_{lo}*D1_{lo}\right)+\left(D0_{lo}*D1_{hi}<<16\right)+\left(D0_{hi}*D1_{lo}<<16\right)+\left(D0_{hi}*D1_{hi}<<32\right)
\]

Of course, we don't have to use the 16 bit words as our ``digits'',
we could easily\footnote{For certain values of ``easily''!} use the
4 bytes in each register as our ``digits'', but given that the MC68008
has a 16 bit multiply, then we might as well use it. 

\section{The Code}

Listings \ref{lst:Unsigned-multiplication-initialisation} through
\ref{lst:Unsigned-multiplication-all-done} show the code necessary
to convert the above theory into actual values.

\lstinputlisting[linerange={77-80},caption={Unsigned multiplication - initialisation},label={lst:Unsigned-multiplication-initialisation}]{/home/norman/SourceCode/Assembly eMagazine/Issue_011/Code/Multiply32Bit/SMult3232.asm}

Listing \ref{lst:Unsigned-multiplication-initialisation} is the initialisation
code for the unsigned multiplication routine. It saves the working
registers on the stack, and initialises the result, in \texttt{D3:D2},
to zero.

\lstinputlisting[linerange={82-86},caption={Unsigned multiplication - testing for zero},label={lst:Unsigned-multiplication-testing-for-zero}]{/home/norman/SourceCode/Assembly eMagazine/Issue_011/Code/Multiply32Bit/SMult3232.asm}

The code continues in Listing \ref{lst:Unsigned-multiplication-testing-for-zero}
where both input registers are tested to ensure that neither is holding
the value zero. If one of the registers is found to be zero, then
the code makes a rapid exit as the result will also be zero and this
has been initialised already.

\lstinputlisting[linerange={91-98},caption={Unsigned multiplication - step 1},label={lst:Unsigned-multiplication-step-1}]{/home/norman/SourceCode/Assembly eMagazine/Issue_011/Code/Multiply32Bit/SMult3232.asm}

As we have two non-zero values, we have to do the multiplication.
We begin with step one of four, and multiply the two low words of
registers \texttt{D0} and \texttt{D1} together. Listing \ref{lst:Unsigned-multiplication-step-1}
shows the code and you will note that the product of the two low words
is saved in register \texttt{D2.L}, which is the low 32 bits of the
64 bit result.

As we need the low word of \texttt{D0} again shortly, it is copied
over to register \texttt{D4}.

\lstinputlisting[linerange={103-107},caption={Unsigned multiplication - step 4},label={lst:Unsigned-multiplication-step-4}]{/home/norman/SourceCode/Assembly eMagazine/Issue_011/Code/Multiply32Bit/SMult3232.asm}

Listing \ref{lst:Unsigned-multiplication-step-4}, which may appear
out of sync, is the fourth of four steps. Don't worry, steps two and
three will be here soon! In this code, we are multiplying the two
high words of the input values together. To do this, we have to swap
the top 16 bits to the bottom as the \texttt{MULU} instruction works
on the lowest 16 bits of the input values.

The result is saved in register \texttt{D3.L} which is the high 32
bits of the 64 bit result.

You may be wondering about the step two and three. They are pretty
much identical. However, there's a catch. We need to calculate the
product of the high word of one register and the low word of the other,
which gives a 32 bit result. That's easy enough, but the catch is
this; we have to shift this result 16 bits leftwards before adding
it to D3:D2. 

We only have 32 bits in a register, and we need 48!

Listing \ref{lst:Unsigned-multiplication-step-2} shows the code for
step two.

\lstinputlisting[linerange={136-141},caption={Unsigned multiplication - step 2},label={lst:Unsigned-multiplication-step-2}]{/home/norman/SourceCode/Assembly eMagazine/Issue_011/Code/Multiply32Bit/SMult3232.asm}

When we reach this point, \texttt{D4.W} is holding the low word of
the value passed in register \texttt{D0.L}, and \texttt{D1} has had
it's two words swapped around so that the current low word was originally
the high word of the passed in value. We multiply these two together
and save the result in \texttt{D4.L}.

We now need to shift this value 16 bits leftwards, and to do this,
we employ register \texttt{D5}. \texttt{D4} is swapped to put the
high word into the low word, and this is copied to \texttt{D5} prior
to clearing the low word of \texttt{D4}. This leaves the register
pair \texttt{D5:D4} holding the value \$0000HHHH:LLLL0000 where HHHH
and LLLL are the high and low words of the product of \texttt{D0}
low and \texttt{D1} high words. (I wish I could write that in better
English!)

The intermediate result in \texttt{D5:D4} needs to be added to \texttt{D3:D2}
holding the result of the multiplication so far. Listing \ref{lst:Unsigned-multiplication-step-2-addition}
is the code to do this addition.

\lstinputlisting[linerange={147-149},caption={Unsigned multiplication - step 2 addition},label={lst:Unsigned-multiplication-step-2-addition}]{/home/norman/SourceCode/Assembly eMagazine/Issue_011/Code/Multiply32Bit/SMult3232.asm}

Adding the two low registers first might generate a carry, which will
also set the X (Extended) flag. This is added in to the addition of
the two high registers to ensure that the carry is properly propagated
into the high 32 bits of the result so far.

Step three of the calculation comes next and it is very similar to
step two, which we have just seen. This time the calculation is \texttt{D0}
high multiplied by \texttt{D1} low and again, shifted 16 bits leftwards
using registers \texttt{D5:D4} to hold the intermediate result.

\lstinputlisting[linerange={155-162},caption={Unsigned multiplication - step 3},label={lst:Unsigned-multiplication-step-3}]{/home/norman/SourceCode/Assembly eMagazine/Issue_011/Code/Multiply32Bit/SMult3232.asm}

\lstinputlisting[linerange={168-170},caption={Unsigned multiplication - step 3 addition},label={lst:Unsigned-multiplication-step-3-addition}]{/home/norman/SourceCode/Assembly eMagazine/Issue_011/Code/Multiply32Bit/SMult3232.asm}

Listings \ref{lst:Unsigned-multiplication-step-3} and \ref{lst:Unsigned-multiplication-step-3-addition}
show the code to carry out the multiplication and addition. There's
not much difference between Listing \ref{lst:Unsigned-multiplication-step-3}
and Listing \ref{lst:Unsigned-multiplication-step-2} other than swapping
the words in \texttt{D1} around to get them where we need them. 

As we want the two input registers \texttt{D0} and \texttt{D1} to
be preserved, we need to do a quick swap of register \texttt{D0} to
accommodate this. Listing \ref{lst:Unsigned-multiplication-restoring-d0}
shows the one liner that does the needful.

\lstinputlisting[linerange={175-176},caption={Unsigned multiplication - restoring D0},label={lst:Unsigned-multiplication-restoring-d0}]{/home/norman/SourceCode/Assembly eMagazine/Issue_011/Code/Multiply32Bit/SMult3232.asm}

\lstinputlisting[linerange={178-180},caption={Unsigned multiplication - all done},label={lst:Unsigned-multiplication-all-done}]{/home/norman/SourceCode/Assembly eMagazine/Issue_011/Code/Multiply32Bit/SMult3232.asm}

All that remains to be done is to restore the working registers and
exit. Listing \ref{lst:Unsigned-multiplication-all-done} shows the
necessary code.

\section{What About Signed Values?}

Signed multiplication is \emph{almost} as simple as unsigned, with
a small caveat. If the two values have different signs -- one negative
and one positive -- then the product is negative. If both have the
same sign -- both negative or both positive -- the product is always
positive.

In the unsigned code discussed in this chapter, multiplying -2 by
-1, \$FFFFFFFE by \$FFFFFFFF, results in \texttt{D3:D2} holding \$FFFFFFFD:00000002
when it should simply be \$00000000:00000002. Also, the N flag is
set to indicate a negative number. This is clearly wrong so we need
a better bit of code for signed multiplication.

Listings \ref{lst:Signed-multiplication-initialisation} through \ref{lst:Signed-multiplication-termination}
show the new code that I have added \emph{above} the \texttt{mult32}
code previously discussed in Listings \ref{lst:Unsigned-multiplication-initialisation}
through \ref{lst:Unsigned-multiplication-all-done}.

\lstinputlisting[linerange={11-13},caption={Signed multiplication - initialisation},label={lst:Signed-multiplication-initialisation}]{/home/norman/SourceCode/Assembly eMagazine/Issue_011/Code/Multiply32Bit/SMult3232.asm}

The initialisation code, in Listing \ref{lst:Signed-multiplication-initialisation},
simply saves register \texttt{D6} on the stack and then initialises
it. \texttt{D6.B} is used as a flag to determine if any adjustments
for the signs of the two values being multiplied, are necessary. It
is initially set to a value of $+1$ to indicate no adjustment required.

\lstinputlisting[linerange={19-23},caption={Signed multiplication - testing D0},label={lst:Signed-multiplication-testing-d0}]{/home/norman/SourceCode/Assembly eMagazine/Issue_011/Code/Multiply32Bit/SMult3232.asm}

\lstinputlisting[linerange={29-33},caption={Signed multiplication - testing D1},label={lst:Signed-multiplication-testing-d1}]{/home/norman/SourceCode/Assembly eMagazine/Issue_011/Code/Multiply32Bit/SMult3232.asm}

Listings \ref{lst:Signed-multiplication-testing-d0} and \ref{lst:Signed-multiplication-testing-d1}
show the code which checks both input values for negativity. If one
or both of the register values is negative, then the flag byte in
\texttt{D6} will be negated once for each negative value, and the
corresponding register value made positive.

\lstinputlisting[linerange={41-44},caption={Signed multiplication - doing the multiplication},label={lst:Signed-multiplication-doing-multiplication}]{/home/norman/SourceCode/Assembly eMagazine/Issue_011/Code/Multiply32Bit/SMult3232.asm}

After checking the two values, we have a guarantee of two positive
values in the two registers. As shown in Listing \ref{lst:Signed-multiplication-doing-multiplication}
we call the unsigned multiplication code already discussed.

\lstinputlisting[linerange={50-52},caption={Signed multiplication - adjusting the result},label={lst:Signed-multiplication-adjusting-the-result}]{/home/norman/SourceCode/Assembly eMagazine/Issue_011/Code/Multiply32Bit/SMult3232.asm}

On return, we test \texttt{D6.B} and if it is positive, no adjustment
of the result is necessary. If it is negative, we drop into Listing
\texttt{\ref{lst:Signed-multiplication-adjusting-the-result}} to
adjust the result, this is simply a case of negating the low register,
\texttt{D2}, and then negating with extend (aka carry) the upper register,
\texttt{D3}.

\lstinputlisting[linerange={54-56},caption={Signed multiplication - termination},label={lst:Signed-multiplication-termination}]{/home/norman/SourceCode/Assembly eMagazine/Issue_011/Code/Multiply32Bit/SMult3232.asm}

All that remains to do is to restore \texttt{D6} and return to the
caller with the result in registers \texttt{D3:D2}. Listing \ref{lst:Signed-multiplication-termination}
has the code.

\section{A Final Warning}

So, that's multiplication in a nutshell. I hope I've managed to explain
it in an understandable manner. It's certainly taken me long enough
to get up the will to understand it myself! Bear in mind that when
using the inbuilt multiplication instructions of the MC68020 that
the flags will be correctly set, however, the code in these routines
might not be as accurate. The following is what I found in testing.
\begin{itemize}
\item The Z (Zero) flag will be correctly set if one or more of the input
values is zero, or if some weird combination of values results in
a zero product. Is such a thing valid?
\item The N (Negative) will be correctly set in either signed or unsigned
multiplication, if the top bit of the result is a 1. That's bit 31
of register D3.
\item The V (Overflow) flag will be set if the multiplication overflows.
Unsigned multiplication of two 32 bit values must always result in
a 64 bit product, I can't get it to overflow into 65 bits. I suspect
therefore that there's a problem with signs. My suspicion is that
the overflow flag will be set if a product overflows into the top
bit of register \texttt{D3} thus becoming negative when it should
be positive.
\item The C (Carry) and X (Extend) flags \emph{should} be zero. So far in
testing, they have been.
\end{itemize}
Oh yes, I mentioned that MC68020 has a 32 bit multiplication. All
of the code in Listings \ref{lst:Unsigned-multiplication-initialisation}
through \ref{lst:Unsigned-multiplication-all-done} and \ref{lst:Signed-multiplication-initialisation}
through \ref{lst:Signed-multiplication-termination} can be replaced
by:

\begin{lstlisting}[caption={MC68020 - unsigned 32bit multiplication},label={lis:MC68020-unsigned-32-bit-multiplication}]
mult32
    move.l d0,d3
    mulu.l d1,d3:d2
    rts
\end{lstlisting}

and:

\begin{lstlisting}[caption={MC68020 - signed 32bit multiplication},label={lis:MC68020-signed-32-bit-multiplication}]
smult32
    move.l d0,d3
    muls.l d1,d3:d2
    rts
\end{lstlisting}

And, the flags will always be correctly set afterwards.

\section{And Finally}

No! No! No! Not under any circumstances! I flatly refuse to do the
same for 32 bit division by a 32 bit number. Sorry, no can do! If
anyone wants a guest chapter in a forthcoming issue, be my guest and
write me a chapter on 32 bit division! 
