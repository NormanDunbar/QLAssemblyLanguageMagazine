\chapter{Powers of Two}

Some more messing about with a bit of code I'm writing for my Arduino
required a given number to be adjusted to the next power of two, unless
that number was already a power of two. So, the value 6 is not a power
of two and would result in a new value of 8, while 4 is already a
power of two and thus, would not be changed.

I managed to get this task accomplished \textendash{} it was for a
circular buffer which can be set up at any size, but the size must
be a power of two, and fit into 8 bits, unsigned \textendash{} in
case you were wondering!

As I'm a bit short on ideas for stuff to write about for this eComic,
I wondered how easy it would be to convert a few hundred bytes of
C++ code into Assembly Language? With a 68020 processor, or QPC2,
and George's GWASS assembler, it was rather simple, and took far less
bytes than on my Arduino! It was a little more difficult with a 68008
and GWASL though.

\section{The Algorithm}

The way to determine the next power of 2 value for a number is reasonably
simple, but there's a catch, a number might already be a power of
2. This is ``easy'' to determine as there will be a single set bit
in the number, so we could count the set bits to determine if the
number is already a power, and return it if so. Too difficult! 
\begin{itemize}
\item Subtract 1 from the number;
\item Find the most significant set bit;
\item Work out a value for a number with just that bit set;
\item Return double the number.
\end{itemize}

\subsection{How it Works}

Ok, we know what to do, how does it work? And why subtract 1 at the
start? Let us assume 8 bit values, for simplicity, and to stop me
typing 32 ones or zeros across the page!

If we take an example of the value 65, this has the binary value 0100
0001. The highest set bit is bit 6 for a value of 64. But as there
are other bits set in the number, 65 is obviously greater than 64.
The next power of 2 greater than 65 is 128. Even though we didn't
do the required subtraction, we would correctly return $2*64$, or
128.

If, on the other foot, the value we started with was 64, it has a
binary value of 0100 0000. Returning $2*64$ would be 128, again,
but this would be incorrect as 64 is already a power of 2, so the
correct answer should be 64.

So, adding in the subtraction this time, we start with 64 \textendash{}
0100 0000 \textendash{} and subtract 1 to give 63, or 0011 1111. The
highest bit set here is bit 5, for a value of 32. Returning $2*32$
is indeed 64. But does that work with a higher value?

Taking 65 again, we still have a binary value of 0100 0001. When we
subtract 1 we get 64 \textendash{} 0100 0000 \textendash{} returning
$2*64$ does indeed still give the correct result of 128.

The algorithm works. Ok, what about zero? Does that end case work?

Subtracting 1 from zero gives -1, or 1111 1111. The most significant
bit set is bit 7 or 128. Returning $2*128$ would be 256, which has
the lower 8 bits clear, or zero. The closest 8 bit power of 2 to zero
is actually zero. This is incorrect as the closest power of 2 to zero
is $2^{0}$ or 1. Hmmm.

In my C++ code, I tested for this corner case, and simply returned
zero. However, in the code in Listing \ref{lis:MC60020-Power2}, it
actually doesn't need a corner case check as passing zero does correctly
result in 1 being returned. Spooky!

\section{Easy Version for 68020}

The code in Listing \ref{lis:MC60020-Power2} is the entire routine.
It is a massive 38 bytes long.

\begin{lstlisting}[caption={MC60020 - Power2\_asm},label={lis:MC60020-Power2},breaklines=true,showstringspaces=false,tabsize=4]
; This code finds the value of the "Next Power of Two" for any
; given number. 
;
; Call here with one (long) parameter.
; PRINT PEEK_L(start + 2) for the result.

start   bra.s   doit
result  ds.l    1

doit    lea result,a1       ; Result address
        move.l d1,d0        ; Passed parameter
        subq.l #1,d0        ; D0 might be a power of 2
        bfffo d0{0:32},d1   ; Find first 1 bit

; If we find a set bit, D1 has the "offset". Bit 31 = offset 0,
; bit 30 = offset 1 and so on. The bits are numbered from the
; MSB which is not the normal manner. To convert, subtract the
; offset from 31 to get the required bit number.

        neg.l d1            ; D1 = -D1
        add.l #31,d1        ; Same as subtracting!
        addq.l #1,d1        ; Just because!
        moveq #0,d2         ; For the result
        bset d1,d2          ; Set the result bit.
        move.l d2,(a1)      ; Save the result

done
        clr.l d0
        rts
\end{lstlisting}

The value we pass in will end up in register D1. For some reason,
I copy that into D0 (I forget why I did that!) but I could have saved
a couple of bytes here and there by leaving it alone! Silly me.

Anyway, the next step is to subtract 1 from D0 and then look for the
most significant set bit. On the 68020 we have the ability to use
bit fields, so that's what the \lstinline[basicstyle={\ttfamily}]!BFFFO D0{0:32},D1!
instruction does, it stands for \emph{Bit Fields Find First One}.
It looks in D0, starting at \emph{offset} 0 for 32 bits, for the first
set 1 bit. If there are no set bits, the Z flag will be set, and D1
will take on the bit field width, or 32, as it's value. 

If there is a set bit, its \emph{offset} will be placed in D1, however,
the offset is not the actual bit number. The offset, as the comments
indicate, is counted from bit 31 down towards bit 0. Normally we count
bits from the least significant end but not in a bit field, they count
from the most significant end. Confusing or what. We can easily convert
an offset into a bit number simply by subtracting it from 31.

We subtract D1 from 31 in the roundabout way of negating D1 and adding
31 to it as $-D1+31=31-D1$.

\section{Hard Version for 68008}

That was the easy case, when using the 68020 processor's useful \lstinline[basicstyle={\ttfamily},showstringspaces=false,tabsize=4]!BFFFO!
instruction, what about the original QL's 68008 processor - it doesn't
have this instruction?

Ok, going back to the examples above with 64 \textendash{} a power
of 2 already \textendash{} first. If we AND a value with the value
minus 1, and keep going until we get a zero answer, we have detected
the leftmost set bit. For example:
\begin{itemize}
\item Value = ??
\item Value = Value - 1 (in case it's already a power of 2)
\item Repeat loop
\item If (value and (value - 1)) = 0, return value {*} 2
\item Else value = (value \& (value - 1))
\item End repeat loop
\end{itemize}
For the initial value of 64, 0100 0000, we have:

\begin{lstlisting}[numbers=none]
64 - 1 = 63
63  = 0011 1111
62  = 0011 1110
AND = 0011 1110 = 62
61  = 0011 1101 
AND = 0011 1100 = 60
59  = 0011 1011
AND = 0011 1000 = 56
55  = 0011 0111
AND = 0011 0000 = 48
47  = 0010 1111
AND = 0010 0000 = 32
31  = 0001 1111 
AND = 0000 0000 = 0.
\end{lstlisting}

As 32 was the current value when we got zero, we return 64, which
is the next power of two to 64. Return $2*32=64$.

If you look at the binary values above, you will see that we delete
one of the lower significant 1s each time we AND with $(value-1)$.
When only a single 1 bit remains, the highest, we are done.

Continuing with the above examples, let's now do 65.

\begin{lstlisting}[showstringspaces=false,tabsize=4,numbers=none]
65 - 1 = 64
64  = 0100 0000
63  = 0011 1111
AND = 0000 0000 = 0.
\end{lstlisting}

As before, the current value was 64 when we got a zero from the AND
operation, so we exit and return the result of 128. That was quick!

Looking good, what about 1?

\begin{lstlisting}[showstringspaces=false,tabsize=4,numbers=none]
1 - 1 = 0
0  = 0000 0000
-1 = 1111 1111
AND = 0000 0000 = 0.
\end{lstlisting}

In this example, the value when we hit zero was zero, so returning
$2*0$ is \emph{not} the correct answer!

It appears that 1 is a special case which the code in Listing \ref{lis:MC68008-Power2}
must check for at the start. This code assembles to a massive 44 bytes
\textendash{} slightly larger than the 68020 code in Listing \ref{lis:MC60020-Power2}.

\begin{lstlisting}[caption={MC68008 - Power2\_asm},label={lis:MC68008-Power2}]
; This code finds the value of the "Next Power of Two" for any
; given number. The first few results are:
;
; Call here with one (long) parameter.
; PRINT PEEK_L(start + 2) for the result.

start   bra.s   doit
result  ds.l    1

doit    lea result,a1       ; Result address

; Special case. If D1 is 1, we expect 2 as the result. But
; we actually get 0. This is because ANDing D0 with 1-1 = 0.

        move.l d1,d0        ; Passed parameter
        cmpi.l #1,d0        ; Was it 1?
        beq.s done          ; Yes, return result (2)

setup
        subq.l #1,d0        ; D0 might be a power of 2
        move.l d0,d2        ; TEMP is D2

loop    move.l d0,d1        ; D1 = D0
        subq.l #1,d1        ; D1 = (D0 - 1)
        and.l d1,d2         ; TEMP = D0 & (D0 - 1)
        beq.s done          ; Zero = no more set bits.
        move.l d2,d0        ; D0 = TEMP
        bne.s loop          ; Not done yet.

done
        lsl.l #1,d0         ; D0 = 2 * D0
        move.l d0,(a1)      ; Save the result
        clr.l d0
        rts

\end{lstlisting}

In the code, the comments show the algorithm in use for any non-special
values \textendash{} basically, anything that isn't 1 \textendash{}
and uses D2 as the TEMP register, D0 is Value and D1 is Value - 1.

D0 is loaded from D1 and has 1 subtracted in case it is already a
power of 2. It is then copied into D2 ready for the main loop. In
the loop, D0 is again copied, this time over to D1, and has 1 subtracted.
This is ANDed with D2 and if the result is zero, we exit the loop
and return whatever is in D0 {*} 2.

If the result is not yet zero, we copy D2 into D0 as the new value,
and try again from the top of the loop. Eventually, we will get a
zero result and will bale out with a value to return.

If the value passed was 1, then we copy that into D0 as normal, and
test for the special case. If we find it, we skip over the main processing
and return $1*2$, which is the correct result.

\section*{Make a Procedure?}

It shouldn't be too hard to convert one of the two listings above
into a working SuperBASIC function:
\begin{itemize}
\item Fetch one parameter as a long integer into D1;
\item Call the code to do the working out, but grab D0 at the end as opposed
to storing it;
\item Allocate 2 extra bytes of maths stack for the result \textendash{}
there's 4 on there already;
\item Convert D0.L to a float and save on the maths stack;
\item Set D3 for a float;
\item Clear D0;
\item Done.
\end{itemize}

