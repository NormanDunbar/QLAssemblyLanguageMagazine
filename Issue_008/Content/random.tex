\chapter{Random Stuff}

Ever needed some randomisation in your assembly code? No, neither
have I until recently when I suddenly had a need to generate random
numbers from 1 to 6 inclusive. How is this possible, given that there
are no apparent vectors or traps to grab hold of a random number?

I could have cheated and had a look through my copies of 68000 coding
books \textendash{} but that would have been in breach of copyright,
probably, and best avoided. So I did the next best thing, I stole
some code from the SMSQ sources!

Almost nothing in the following is my own work, I have stolen it,
and only slightly modified it to suit what I needed it for. It does
work though, I'm happily generating numbers from 1 to 6 inclusive,
and no, it is not a dice (die) that I'm emulating even if the same
numbers are involved!

\section{Random Seed}

The code I was working on is to be used in a job, so I have a storage
location for my random seed which is an offset from the \lstinline!A6!
register. The following code takes that into consideration.

\begin{lstlisting}[caption={The random seed},firstnumber=1]
; In job code, this is an offset from the A6 register.

myRandSeed  equ     0

; The job code starts here
start       bra.s   myCode
            dc.l    0
            dc.w    $4afb

fname
    dc.w    fname_e-fname-2
    dc.b    "My Job's Name"

fname_e
    equ     *

myCode
    adda.l  a4,a6               ; A6 points to our data
    clr.l   d1                  ; Randomise the date
    bsr     randomise           ; Do it
    ...                         ; Do stuff
    bsr     rnd                 ; Generate a word 1 to 6 inclusive
    ...                         ; Do more stuff
\end{lstlisting}

The purpose of the job code is not relevant here, it will become apparent,
I hope, when I get it finished \textendash{} either in this edition
or a future one. Coming soon\footnote{Given my record, for certain values of ``soon''!}
and all that!

I could have used the system random seed for my own numbers, but I
thought about it and didn't want to mess up any other programs that
could be running but which depend on a certain set of random numbers
based on a starting random seed. Unlikely, perhaps, but I decided
to avoid the problem.

\section{Randomisation}

Now that we have the seed variable, we will need a manner of initialising
it with a new value. SuperBASIC does this by taking the clock's value
in seconds if we don't supply a value ourselves, so that's good enough
for me too. You will note from the comments that this code is stolen
and only amended in a minor manner for my own needs.

\begin{lstlisting}[caption={Randomise function},firstnumber=24]
;------------------------------------------------------------
; This is effectively ramdomise(date). The code is exactly as
; per the SBasic RANDOMISE function. I stole the code! 
; (sbsext/ext/rnd.asm)
;
; Enter with D1.L = 0 to randomise(date) or with D1.L = some
; specified value to randomise(D1).
;
; Preserves all registers.
;------------------------------------------------------------
randomise
    movem.l d0-d2/a0,-(a7)      ; Save workers
    tst.l   d1                  ; D1 passed with a value?
    bne.s   randomise_d1        ; Yes, skip the date
    moveq   #mt_rclck,d0        ; Read clock into D1
    trap    #1                  ; No errors, no need to
;                               ; call doTrap1.

randomise_d1
    move.l  d1,d2               ; D1 = D2 = HHHH LLLL
    swap    d1                  ; D1 = LLLL HHHH
    add.l   d2,d1               ; D1 = (LLLL+HHHH) (HHHH+LLLL)
    move.l  d1,myRandSeed(a6)   ; Save random seed
    movem.l (a7)+,d0-d2/a0      ; restore workers
    rts

\end{lstlisting}

The code should call the \lstinline!randomise! entry point either
with \lstinline!D1.L! holding zero, or the required starting value
for our seed. If we passed zero in D1, then the current date, in seconds,
is read from the system and used as a starting point.

Arriving a label \lstinline!randomise_d1!, we have a non-zero value
in D1.L and can use it to randomise the system. The value is copied
into \lstinline!D2.L! for safety, then D1 is swapped over to put
the low word into the high word and vice versa. If we started with
\lstinline!D1.L! holding \$12345678 we end up with \$56781234. This
value is then added to the original seed value in \lstinline!D2.L!
to give, in this example, \$68AC68AC.

Whenever \lstinline!randomise! is called, we end up with a new random
seed which has the high and low words identical. However, as soon
as we begin using the seed, that changes. Read on.

\section{Random Generation}

Before we continue, can I just point out that I am by no means a mathematician
and while I can look at what the code is doing, I have no idea what
formula it is using to do it! 

The first problem is to generate a random number from the random seed
and to update the seed. We need to do this to avoid generating the
same value over and over again!

\begin{lstlisting}[caption={Rnd 1 to 6 function - Part 1},firstnumber=49]
;------------------------------------------------------------
; This is effectively RND(1 TO 6). The code does exactly as
; per the SuperBASIC RND() function. I stole the code!
; (sbsext/ext/rnd.asm)
;
; D1 = Bottom of range = 1.
; D2 = Top of range + 1 = 7.
;
; Returns D1.W as RND(1 to 6) and obviously trashes D1.
;------------------------------------------------------------
rnd
    movem.l d0/d2/d4,-(a7)      ; Save workers
    move.l  myRandSeed(a6),d0   ; Get seed value
    move.w  d0,d4               ; Copy low word ???? LLLL
    swap    d0                  ; LLLL HHHH
    mulu    #$c12d,d0           ; D0.L = HHHH * 49453
    mulu    #$712d,d4           ; D4.L = LLLL * 28973
    swap    d0                  ; HHHH LLLL
    clr.w   d0                  ; HHHH 0000
    add.l   d0,d4               ; I have no idea!
    addq.l  #1,d4               ; New seed
    move.l  d4,myRandSeed(a6)   ; Save seed

\end{lstlisting}

After saving the registers we will be working with, except \lstinline!D1!
which we use to return the random number later, we grab hold of the
random seed and start messing about with it to generate a new seed. 

The high word of the seed, in \lstinline!D0!, is multiplied by 49,453
and the low word, in \lstinline!D4!, by 28,973 neither of which are
prime. Both are divisible by 7 if you don't fancy working it out!
The resulting long word is \lstinline!D0! is swapped and added to
\lstinline!D4! before \lstinline!D4! is incremented. This is our
new random seed and is saved accordingly ready for the next call to
the \lstinline!rnd! routine.

If you want to work through an example:

The original seed, \$68AC68AC, is copied to \lstinline!D0! and \lstinline!D4!.
After swapping \lstinline!D0!, which has no real effect on the first
call after a call to \lstinline!randomise!, we end up with \lstinline!D0.L!
= \$68AC68AC and \lstinline!D4.L! = \$xxxx68AC. We are going to multiply
so the high word of \lstinline!D4! is of no interest.

After the two multiplications, we now have \lstinline!D0.L! = \$68AC
{*} \$C12D = \$4EFC123C and \lstinline!D4.L! = \$68AC {*} \$712D
= \$2E46523C.

Swapping \lstinline!D0! and clearing the low word gives \$123C0000
which we then add to \lstinline!D4! to get \$4082523C which is then
incremented to \$4082523D and used as the next random seed.

After all that, we are now, \emph{finally}, ready to generate the
random integer between 1 and 6 that we want.

The code below is designed\footnote{For certain values of ``designed''!}
to only generate a random number between 1 and 6, inclusive, and these
two values are hard coded into registers \lstinline!D1! and \lstinline!D2!.
Should you wish to generalise the following code to pass your own
ranges, it should be quite simple.

\begin{lstlisting}[caption={Rnd 1 to 6 function - Part 2},firstnumber=71]
rndOneToSix
    moveq   #1,d1               ; Bottom of range
    moveq   #6+1,d2             ; Top of range + 1
    sub.w   d1,d2               ; Size of range

\end{lstlisting}

First we work out the range of values that we need. This is $\left(D2+1\right)-D1$
which in this case, always works out as $\left(6+1\right)-1$ giving
a range of 6. At this point we have a random

long integer in \lstinline!D4! and a range in \lstinline!D2!.

\begin{lstlisting}[caption={Rnd 1 to 6 function - Part 3},firstnumber=75]
    swap    d4                  ; LLLL HHHH of seed
    mulu    d2,d4               ; D4.HHHH * top
    swap    d4                  ; Take top word
    add.w   d4,d1               ; Add range bottom
;                               ; D1 = RND(1 to 6)
    movem.l (a7)+,d0/d2/d4      ; Restore workers.
    rts

\end{lstlisting}

Now, for some unknown reason, we swap \lstinline!D4!, our new random
number and seed, and multiply the previous high word by the range
of values we are looking for, and swap it back again. After this,
we add the bottom value of the range to the low word of \lstinline!D4!
and this is our value between 1 and 6. We then restore the working
registers and return the value in \lstinline!D1.W!.

Continuing the worked example from previously, D4 starts off as \$2E46523D
and we multiply the high word, \$2E46, by the range, 6, to get \$000115A4.
After swapping \lstinline!D4! back to get \$15A40001, we add on the
base of the range, 1, to get our actual random number, \$0002 in this
case.

Any mathematicians out there fancy writing up an explanation of \emph{exactly}
what the hell is going on there?
