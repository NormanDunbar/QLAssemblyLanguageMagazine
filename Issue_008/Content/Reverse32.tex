%% LyX 2.2.4 created this file.  For more info, see http://www.lyx.org/.
%% Do not edit unless you really know what you are doing.
\chapter{Reversing Bits}

Many years ago, I needed a routine to reverse the bits in a register
so that bit 0 ended up in bit 31, bit 31 was in bit 0 and so on. I
think I asked on the QL Mailing List and the responses I received
were pretty similar to the method I knew about - shifting bits right
from the input register through the Carry Flag and shifting left into
another register. It worked fine but I always thought that there would
be a \emph{better} solution. I never found one.

The other day, while doing some embedded (aka Arduino) fiddling, I
found a piece of code to reverse the order of bits in an 8 bit register,
which is what the Arduino's ATmega328P microcontroller has. I had
a look at the code and decided that I could adapt it to reverse all
32 bits on a QL register. This is what I came up with.

Bear in mind that in order to reverse 32 bits through the Carry Flag
you only need three registers - the source, the destination and a
counter for the 32 shifts for each register. That takes a total of
64 one bit shifts to reverse all the bits.

\section{Reversing 2 Bits}

Might as well start easy. If we start with the value 10 in our two
bit register, we can reverse the value to 01 as follows:
\begin{itemize}
\item AND 10 \emph{with 10;}
\item Shift the result \emph{right} by one bit;
\item AND 10 with 01;
\item Shift the result \emph{left} by 1 bit;
\item OR the results of the two AND operations.
\end{itemize}
So much for the theory, let's see if it works:

\begin{lstlisting}[numbers=none]
10 AND with 10 = 10.
10 >> 1 = 01.

10 AND with 01 = 00.
00 << 1 = 00.

10 OR 00 = 10.
\end{lstlisting}

Easy? We started with 10 and finished with 01, job done, we reversed
the two bits. So far so good, lets up things a bit and see what happens
with 4 bits.

\textbf{Note:} You can, if you wish, shift first then AND, it still
works.

\section{Reversing 4 Bits}

To reverse 4 bits, we would do something similar. If we start with
the value 1101, then all we have to do is:
\begin{itemize}
\item AND 1101 with 1100;
\item Shift the result \emph{right} by two bits;
\item AND 1101with 0011;
\item Shift the result \emph{left} by two bits;
\item OR the two results of the AND operations.
\end{itemize}
Again, let's see if it works:

\begin{lstlisting}[numbers=none]
1101 AND 1100 = 1100.
1100 >> 2 = 0011.

1101 AND 0011 = 0001.
0001 << 2 = 0100.

0011 OR 0100 = 0111.
\end{lstlisting}

Oops! That's not quite right. All we have really done, and indeed,
in the first two bit example, is \emph{swap} the top two bits with
the bottom two bits, we have not \emph{reversed} them. We need to
now swap the two pairs of two bit values in the above result, 0111.
Let's continue.

We are currently have 0111 as our intermediate result. This is 2 two
bit values, 01 and 11. We know that swapping the 2 bits in a two bit
value reverses them. Can we now reverse the pair of two bit values
\emph{at the same time}?

\begin{lstlisting}[numbers=none]
0111 AND 1010 = 0010.
0010 >> 1 = 0001.

0111 AND 0101 = 0101.
0101 << 1 = 1010.

0001 OR 1010 = 1011.
\end{lstlisting}

So, we started with 1101, swapped the two pairs of two bit values
over, then reversed both bits in each pair to receive the result 1011
which is a complete bit reversal of the original 4 bits. 

\section{Reversing 8 Bit values}

In theory then, we should be able to start with 8 bits, swap the two
pairs of 4 bits over, then swap the 4 pairs of two bit values over,
then reverse the bits in those 4 two bit values.

To swap the 4 bit values we follow the same principle as above:
\begin{itemize}
\item AND the value with 11110000;
\item Shift the result \emph{right} by 4 bits;
\item AND the value with 00001111;
\item Shift the result \emph{left} by four bits;
\item OR the two results of the AND operations;
\item Then carry out the steps for a 4 bit swap but do two at a time;
\item Swap/reverse the bits in each of the resulting two bit values.
\end{itemize}
Does it work? Lets try with \$C9 or 11001001:

\begin{lstlisting}[numbers=none]
11001001 AND 11110000 = 11000000.
11000000 >> 4 = 00001100.

11001001 AND 00001111 = 00001001.
00001001 << 4 = 10010000.

00001100 OR 10010000 = 10011100.
\end{lstlisting}

That's the 2 four bit values exchanged, but not yet reversed. We continue
with the process to swap the 2 two bit values in each of the 4 bit
values:

\begin{lstlisting}[numbers=none]
10011100 AND 11001100 = 10001100.
10001100 >> 2 = 00100011.

10011100 AND 00110011 = 00010000.
00010000 << 2 = 01000000.

00100011 OR 01000000 = 01100011.
\end{lstlisting}

Now we simply reverse the bits in each of the 4 two bit values:

\begin{lstlisting}[numbers=none]
01100011 AND 10101010 = 00100010.
00100010 >> 1 = 00010001.

01100011 AND 01010101 = 01000001.
01000001 << 1 = 10000010.

00010001 OR 10000010 = 10010011.
\end{lstlisting}

And that's working correctly too, 11001001 has been bit reversed to
become 10010011.

\section{Reversing 16 Bit Values}

With a 16 bit value, we would:
\begin{itemize}
\item Swap the pair of 8 bit values around;
\item Swap the 4 four bit values around;
\item Swap the 8 two bit values;
\item Reverse the 8 two bit values.
\end{itemize}
Do you see a pattern developing? To swap the two n/2 bit values in
an 'n' bit value:
\begin{itemize}
\item AND the value with a mask of n/2 ones and n/2 zeros;
\item Shift the value \emph{right} by n/2 bits;
\item AND the value with a mask of n/2 zeros and n/2 ones;
\item Shift the value \emph{left} by n/2 bits;
\item OR the two results to obtain a new value where the two n/2 bit values
have been swapped.
\end{itemize}
This works all the way down until the final processing of two bit
values and swapping those over actually reverses the bit in the pairs
of bits, giving the final result.

\section{Reversing 32 Bit Values}

You should be able to work out the bit shifts and masks required to
swap around the two 16 bit values in a 32 bit value? If you said ``Yes,
use the \lstinline[numbers=none,breaklines=false]!SWAP! instruction''
you would be correct - there is no need to do the \emph{mask and shift
dance} to swap them over, we already have a single instruction to
do exactly that!

It's time for some code. Listing \ref{lis:Reverse32_asm-Header-comments}
is the comments at the head of the file which explains how to call
the demo code from SuperBASIC or Assembly, and how to extract the
result.

\begin{lstlisting}[caption={Reverse32\_asm - Header Comments.},label={lis:Reverse32_asm-Header-comments},firstnumber=1]
; A small function to reverse the bits in a long word.
; So, 1111 1111 0000 0000 1100 1100 1010 1010 will become
;     0101 0101 0011 0011 0000 0000 1111 1111
;
; Norman Dunbar
; June 25 2020.
;
; Call this code from SuperBASIC as follows:
;
; CALL address, value
; PRINT bin$(PEEK_L(address + 2), 32)
;
; Where 'address' is the address of the label 'entry'.
;
; To use the code in Assembly:
;
; Call 'reverse32Bits' with D0.L as the value to reverse.
; The code exits with the reversed bits in D0.L.
;
\end{lstlisting}

Listing \ref{lis:Reverse32_asm-SuperBASIC-entry-point} is the SuperBASIC
code entry point. The code should be \lstinline[numbers=none,breaklines=false]!CALL!ed
and passed a single value to be bit reversed.

\begin{lstlisting}[caption={Reverse32\_asm - SuperBASIC Entry Point.},label={lis:Reverse32_asm-SuperBASIC-entry-point},numbers=left,firstnumber=20]
entry
    bra.s   start

saveD0
    dc.l    1

start
    move.l  d1,d0
    bsr.s   reverse32Bits
    lea     saveD0,a3
    move.l  d0,(a3)
    moveq   #0,d0
    rts
\end{lstlisting}

As you cannot pass a value to register \lstinline[numbers=none,breaklines=false]!D0!
from SuperBASIC, the value in \lstinline[numbers=none,breaklines=false]!D1.L!
is copied into \lstinline[numbers=none,breaklines=false]!D0.L! and
the bit reversal code in Listing \ref{lis:Reverse32_asm-Reverse32Bits}
is called to reverse the bits. The result is extracted from \lstinline[numbers=none,breaklines=false]!D0.L!
and stored in the long word at label \lstinline[numbers=none,breaklines=false]!saveD0!
from where it can be \lstinline[numbers=none,breaklines=false]!PEEK_L!'d
by SuperBASIC to retrieve the reversed bit value.

The code for the actual bit reversal is shown in Listing \ref{lis:Reverse32_asm-Reverse32Bits}.
This routine starts by saving all the working registers and testing
\lstinline[numbers=none,breaklines=false]!D0! for zero. Zero is already
``reversed'' so we bale out early if this special case is detected.

If we intend to carry on, the table of mask values is assigned to
\lstinline[numbers=none,breaklines=false]!A0! and we start by swapping
over the two 16 bit values in the 32 bit register. That's the simple
bit out of the way. The masks we have in the table are those we will
use to swap over 16 bit values, then 8, then 4 and finally the 16
pairs of two bit values.

Register \lstinline[numbers=none,breaklines=false]!D4! holds the
number of shifts we need to do at each step in the process.

\begin{lstlisting}[caption={Reverse32\_asm - Reverse32Bits Routine.},label={lis:Reverse32_asm-Reverse32Bits},firstnumber=33]
reverse32Bits
    movem.l d1-d4/a0,-(a7)  ; Save the workers
    tst.l   d0              ; Zero?
    beq.s   reverseDone     ; Yes, done
    lea     maskTable,a0    ; Mask table
    swap    d0              ; The easy 16 bits are swapped...
    moveq   #8,d4           ; Shift counter

reverseLoop
    move.l  (a0)+,d1        ; Get first/next mask
    beq.s   reverseDone     ; Finished
    move.l  d1,d2           ; Copy mask
    not.l   d2              ; Invert mask copy
    move.l  d0,d3           ; Copy value
    and.l   d1,d0           ; Mask
    and.l   d2,d3           ; Inverted mask
    lsr.l   d4,d0           ; Shift top down
    lsl.l   d4,d3           ; Shift bottom up
    or.l    d3,d0           ; Combine the bits
    lsr.b   #1,d4           ; Reduce shift count
    bra.s   reverseLoop     ; And again

reverseDone
    movem.l (a7)+,d1-d4/a0  ; Restore workers
    rts

maskTable
    dc.l    $FF00FF00       ; 1111111100000000 1111111100000000
    dc.l    $F0F0F0F0       ; 1111000011110000 1111000011110000
    dc.l    $CCCCCCCC       ; 1100110011001100 1100110011001100
    dc.l    $AAAAAAAA       ; 1010101010101010 1010101010101010
    dc.l    0
\end{lstlisting}

The code at \lstinline[numbers=none,breaklines=false]!reverseLoop!
does all the hard work. On entry \lstinline[numbers=none,breaklines=false]!A0.L!
points at the first mask we need, so that is loaded into \lstinline[numbers=none,breaklines=false]!D1.L!
and copied immediately to register \lstinline[numbers=none,breaklines=false]!D2.L!
where the bits are inverted to give the second mask we need. The \lstinline[numbers=none,breaklines=false]!maskTable!
only stores one of each pair of mask values. If the mask value is
zero, we are done and we exit the loop.

The value to be reversed is copied into \lstinline[numbers=none,breaklines=false]!D3.L!
and \lstinline[numbers=none,breaklines=false]!D0.L! is \lstinline[numbers=none,breaklines=false]!AND.L!'d
with the mask in \lstinline[numbers=none,breaklines=false]!D1.L!.
That gives the first result, prior to the shifts. \lstinline[numbers=none,breaklines=false]!D3.L!
is \lstinline[numbers=none,breaklines=false]!AND.L!'d with the inverted
mask in \lstinline[numbers=none,breaklines=false]!D2.L! which gives
the second result, prior to the shifts. 

Both registers are then shifted in the appropriate direction by the
number of bits in \lstinline[numbers=none,breaklines=false]!D4.B!
before the value in \lstinline[numbers=none,breaklines=false]!D3.L!
is \lstinline[numbers=none,breaklines=false]!OR.L!'d into \lstinline[numbers=none,breaklines=false]!D0.L!.
All that remains is to divide the shift count in \lstinline[numbers=none,breaklines=false]!D4.B!
by two before we jump back into the top of the loop.

When we are all done reversing the bits in \lstinline[numbers=none,breaklines=false]!D0.L!,
we restore the working registers and return to the caller with the
reversed bits in register \lstinline[numbers=none,breaklines=false]!D0.L!.

The table at \lstinline[numbers=none,breaklines=false]!maskTable!
holds 4 masks which are used when swapping over the two n/2 bit values
in an n bit value. As you can see only the mask for the first \lstinline[numbers=none,breaklines=false]!AND.L!
instruction is stored. This is because the mask used in the second
\lstinline[numbers=none,breaklines=false]!AND.L! instruction is the
inverted value, and the \lstinline[numbers=none,breaklines=false]!NOT.L!
instruction will give us that mask.
