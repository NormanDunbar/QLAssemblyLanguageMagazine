
\chapter{Maths Stack Update}

Years ago I wrote an article for \emph{QL Today} about the Maths Stack
in QDOS and how the value in \texttt{A1}, on entry to a function or
procedure, was \emph{not} pointing to the top of the maths stack,
as documented in the various QDOS manuals, and books around at that
time. In brief, I discovered that on entry, we have three possibilities:
\begin{itemize}
\item If \texttt{A1} is \emph{negative}, it means that the function/procedure
in question has been called as part of an expression, and \texttt{A1}
shows how much space has been used on the stack for the expression,
so far. The code would resemble this: 
\begin{lstlisting}[basicstyle={\ttfamily},showstringspaces=false,tabsize=4]
PRINT 1234 * myFunction(...)
\end{lstlisting}
\item If \texttt{A1} is \emph{positive}, it means that the maths stack has
some free space available for use without any requirement to allocate
more. 
\item If \texttt{A1} is \emph{zero}, it means that the function/procedure
in question has been called on it's own, or at the beginning of an
expression. \texttt{A1} shows how much space has been used on the
stack for the expression, so far. The code would resemble this: The
code would resemble this: 
\begin{lstlisting}[basicstyle={\ttfamily},showstringspaces=false,tabsize=4]
PRINT myFunction(...)
\end{lstlisting}
 or 
\begin{lstlisting}[basicstyle={\ttfamily},showstringspaces=false,tabsize=4]
PRINT myFunction(...) * 1234
\end{lstlisting}
 or any expression where \lstinline[basicstyle={\ttfamily},showstringspaces=false,tabsize=4]!myFunction()!
is at the beginning.
\end{itemize}
This is covered in my eBook which you can download from \href{https://github.com/NormanDunbar/QLAssemblyLanguageBook/releases/latest}{GitHub}\footnote{\url{https://github.com/NormanDunbar/QLAssemblyLanguageBook/releases/latest}}
-- section 7.8 is the location you will need.

There was a recent exercise by Derek Stewart to scan Andy Pennell's
\emph{QDOS Companion} book and make it available as a PDF. Andy did
give his blessing to this, in case anyone is worried about copyright.
There's a thread on the QL Forum, \href{https://qlforum.co.uk/viewtopic.php?f=12&t=4137}{here}\footnote{\url{https://qlforum.co.uk/viewtopic.php?f=12\&t=4137}},
which shows the process and updates so far. See this issues News chapter
for download details.

As part of the ensuing discussion, it was mentioned that Andy's book
didn't correctly document A1 when entering a procedure or function.
I responded with the above information and was brought before the
select committee\footnote{Per Witte} to explain:
\begin{itemize}
\item \emph{You don't say whether your findings apply to: JM, JS, Minerva,
SMSQ/E etc. It could, in other words just be random.}
\item \emph{I'm no authority, but I would advise to not rely on this unless
it were documented to be valid across the board, and to be there for
a particular reason, ie as a result of the way things by necessity
are done, or as a particular service to keyword writers. I wont go
into reasons as I hope they are self evident. }
\end{itemize}
Which are excellent points of order.

Let's dive into the maelstrom then!

\section{The Process Outline}

In order to test I will be following these few steps:
\begin{itemize}
\item Create a function with no parameters, and another with one parameters,
as the test code. A \texttt{TRAP \#15} instruction will be placed
at the very start of the function code. On normal running, this has
no effect, but when I do some fiddling in the commands of QMON2, it
will immediately jump into QMON2 and I can debug from there.
\item When the debugger has control, examine the registers paying close
attention to A1.
\item The code will be tested on SMSQ/E under QPC2. It has already been
tested in QDOS under the JS ROM, back in the day. I don't have Minerva
or other ROMS, but I do have emulators which do!
\item The functions will be tested stand-alone, and as part of an expression
where it occurs at the beginning and another where it appears after
the beginning.
\item The functions will be tested in integer, float and string (coercion)
expressions.
\end{itemize}

\section{The Test Functions}

Listing \ref{lst:TestFunction-fn=00005C_0} is the code I'll be using
for the test function \texttt{FN\_0} and Listing \ref{lst:TestFunction-fn=00005C_1}
is the code for function \texttt{FN\_1}. The function names match
the number of parameters they take. I have not shown the code that
links the functions into S{*}BASIC, but this is present in the code
download for this issue.

There's nothing special about the two functions, other than the \texttt{TRAP
\#15} instructions which appear at the start of each function. This
causes execution to jump into QMON2, if that has been loaded and set
up correctly. \texttt{FN\_0} returns a word integer of zero as it's
result, while \texttt{FN\_1} returns its parameter, incremented by
1, as another word integer.

\lstinputlisting[linerange={39-47},firstnumber=39,numbers=left,caption={Test Function: FN\_0},label={lst:TestFunction-fn=00005C_0}]{/home/norman/SourceCode/Assembly eMagazine/Issue_012/Code/MathsStack/testFunctions.asm}

\lstinputlisting[linerange={49-71},firstnumber=49,numbers=left,caption={Test Function: FN\_1},label={lst:TestFunction-fn=00005C_1}]{/home/norman/SourceCode/Assembly eMagazine/Issue_012/Code/MathsStack/testFunctions.asm}

Obviously, as \texttt{FN\_0} takes no parameters, we need to allocate
space on the maths stack before we can return a result. We do this
by fetching the current maths stack pointer from \texttt{SV\_ARTHP(A6)}
into \texttt{A1}, and requesting \texttt{D6.W} bytes of space for
the result. In this case, \texttt{D6.W} is 2 as we are returning a
word integer. After the space has been allocated, it is possible that
the maths stack, plus its contents, has been moved, so we need to
refresh \texttt{A1} from \texttt{SV\_ARTHP(A6)}. 

\texttt{FN\_1}, on the other hand, takes a word integer parameter
and as such, has already got enough space on the maths stack to return
its result. For \texttt{FN\_1} then, \texttt{D6.W} is zero to indicate
that we don't need any space so we simply store the value in \texttt{D7.W}
into the word pointed to by \texttt{(A6,A1.L)}.

Listing \ref{lst:TestFunctions-return-results} is the code that handles
the returning of results and the allocation of maths stack space,
as required, in order to do so.

\lstinputlisting[linerange={73-97},firstnumber=73,numbers=left,caption={Returning results},label={lst:TestFunctions-return-results}]{/home/norman/SourceCode/Assembly eMagazine/Issue_012/Code/MathsStack/testFunctions.asm}

\section{Debugging with QMON2}

After loading the QMON2 binary, we need to tell it to execute when
a \texttt{TRAP \#15} instruction is executed. to do this we simply
execute QMON2 in S{*}BASIC channel \#1 -- so that when it executed,
it comes up in the same channel -- and then tell it to intercept
those \texttt{TRAP \#15} instructions. This is done as follows:

\begin{lstlisting}[basicstyle={\ttfamily},showstringspaces=false,tabsize=4]
QMON #1

Qmon> TL 14
Qmon> g
\end{lstlisting}

That's all there is to do. Now whenever a \texttt{TRAP \#15} instruction
is executed, QMON2 will intercept it, and break execution, giving
us full control over the system at the instruction immediately after
the \texttt{TRAP \#15} instruction.

The \texttt{TL} command in QMON2 will execute QMON2 every time a \texttt{TRAP}
\emph{higher} than that indicated, is executed.

After setting up QMON2, testing was done by calling \texttt{PRINT}
with various expressions as its parameters, each using the appropriate
test function in one of three places:
\begin{itemize}
\item As the only term in the expression.
\item As the first term in the expression.
\item As the final term in the expression.
\end{itemize}
My debugging session looked something like this:

\begin{lstlisting}[basicstyle={\ttfamily},showstringspaces=false,tabsize=4]
X=1234
X% = 1234
x$="1234"

PRINT FN_0
PRINT FN_0 * X
PRINT FN_0 * X%
PRINT FN_0 * X$

PRINT X * FN_0
PRINT X% * FN_0
PRINT X$ * FN_0

PRINT FN_1
PRINT FN_1 * X
PRINT FN_1 * X%
PRINT FN_1 * X$

PRINT X * FN_1
PRINT X% * FN_1
PRINT X$ * FN_1
\end{lstlisting}


\section{Results}

On SMSQ/E the results are slightly different from that in QDOS with
the JS ROM. There was no difference when the function called was \texttt{FN\_0}
or \texttt{FN\_1}, they both showed the same values in \texttt{A1.L}
when QMON2 intercepted execution. Table \ref{tab:Test-function-results}
shows the results of testing on SMSQ/E.

\begin{table}[!h]
\begin{centering}
\begin{tabular}{|c|c|c|}
\hline 
\textbf{Test} & \textbf{A1 (Hex)} & \textbf{A1 (Decimal)}\tabularnewline
\hline 
\hline 
Function only & 0 & 0\tabularnewline
\hline 
Function {*} integer & 0 & 0\tabularnewline
\hline 
Function {*} float & 0 & 0\tabularnewline
\hline 
Function {*} string & 0 & 0\tabularnewline
\hline 
Integer {*} function & FFFFFFFC & -4\tabularnewline
\hline 
Float {*} function & FFFFFFF8 & -8\tabularnewline
\hline 
String {*} function & FFFFFFFA & -6\tabularnewline
\hline 
\end{tabular}
\par\end{centering}
\caption{Test function results\label{tab:Test-function-results}}

\end{table}

I repeated the tests with other arithmetic operators, the results
are the same as Table \ref{tab:Test-function-results} regardless
of the operator in question. 

Looking at the figures, it appears that on SMSQ/E, the value in \texttt{A1}
at the start of a procedure or function is 2 bytes greater than that
used by QDOS under the same conditions.
\begin{itemize}
\item When \texttt{A1} is zero, the SMSQ/E result is the same as the QDOS:
JS result, no space yet used on the maths stack.
\item When \texttt{A1} is negative, the SMSQ/E result appears to show an
extra 2 bytes over the size of the space used on the maths stack so
far. So for a word sized integer, 2 bytes, the value in \texttt{A1}
is -4, for a 6 byte float it's -8 and for a string, which was never
tested on QDOS: JS, it's always -6 regardless of the length of the
string.
\item I was unable to remember\footnote{I'm getting old. When I learn new stuff these days, something else
has to be forgotten to make room for it!} how I managed to get a positive value in \texttt{A1} back in the
QDOS: JS days, and I was unable to get one in SMSQ/E.
\end{itemize}

\section{Summary}

Even without testing Minerva ROMs under QDOS, it is obvious that there
are indeed differences -- at least between QDOS and SMSQ/E. For this
reason, it's best to ignore what I've said in the past, and simply
treat \texttt{A1} on entry to a procedure or function as \emph{not
being a suitable value for the top of the maths stack} so it must
always be loaded from \texttt{BV\_RIP(A6)} on QDOS or from \texttt{SV\_ARTHP(A6)}
on SMSQ/E.
\begin{itemize}
\item If your function takes no parameters, \texttt{A1} \emph{must} be loaded
before attempting to request maths stack space to return the result.
\item If your function takes any parameters, then until such time as you
call the appropriate vector to fetch the parameters, \texttt{A1} is
not usable as the maths stack top. After fetching the parameters,
it will be correctly set.
\end{itemize}

