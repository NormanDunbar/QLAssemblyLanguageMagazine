
\chapter{Quickie Corner}

In this issue's ``Quickie Corner'' I shall take a brief look at program structure. What I mean by this is how to set up the various constructs much loved in structured programming. Plus some other hopefully useful stuff as well. I won't be showing any fully executable code, just the basics.

It's all very well, in SuperBASIC and S*BASIC\footnote{Henceforth referred to as simply S*BASIC.}, having stuff like \texttt{REPeat}\ldots{}\texttt{END REPeat} or \texttt{IF\ldots{}ELSE\ldots{}END IF}, for example, but how do we do this in Assembly Language? 

Some of what we will discuss is hopefully familiar from S*BASIC, but other constructs may not be. Worry not!

Read on.

\section{Repeat Loop}

The simplest of all the looping constructs. You have a loop start and a loop end. Between the two is the loop body, which is where all the work you wish to carry out repeatedly is located. The loop end simply passes program control back to the beginning of the loop.

\begin{lstlisting}[caption={REPEAT \ldots END REPEAT},label={lis: Repeat loop}]
repeat
    ; Do loop body code here.
    ...

endRepeat
    bra loopStart
\end{lstlisting}

Of course Listing \ref{lis: Repeat loop} is an infinite loop; there is no ``get out'' clause, so once it starts it will continue until power is removed or the world ends! This sort of loop is acceptable under certain circumstances, but normally, users need an opportunity to finish things off nicely. Listings \ref{lis: Repeat loop with exit} and \ref{lis: Repeat loop with alternative exit} should improve things by offering an exit clause. I'm assuming that \register{D0} is zero if we wish to exit. Other exit conditions are available!

\begin{lstlisting}[caption={REPEAT \ldots EXIT \ldots END REPEAT},label={lis: Repeat loop with exit}]
repeat
    ; Do loop body code here.
    ...
    
endRepeat
    ; Test for exit condition and ...
    ; continue looping until D0 = 0.
    tst.l d0   
    bne repeat
    
exitRepeat
    ; Continue onwards from here.
    ...
\end{lstlisting}

\begin{lstlisting}[caption={Alternative REPEAT \ldots EXIT \ldots END REPEAT},label={lis: Repeat loop with alternative exit}]
repeat
    ; Do loop body code here.
    ...

endRepeat
    ; Test for exit condition and ...
    ; continue looping until D0 = 0.
    tst.l d0
    beq exitRepeat

    ; It appears we are not yet done.
    bra repeat

exitRepeat
    ; Continue onwards from here.
    ...
\end{lstlisting}

Obviously, Listing \ref{lis: Repeat loop with exit} is the better option as it is slightly shorter, however, there may be code where it is not the better option. It all depends.

The test for the exit condition can come anywhere in the loop body. You can test at the start, at the end, or part way through---it all depends of what the code is doing.

\section{Repeat \ldots\protect Until Loop}\label{section: repeat until loops}

This is not a structure available on the QL, but it is available in other languages, so might be useful to know about. The body of the loop will be executed at least once as the condition test is at the bottom of the loop. Had the condition test been at the top of the loop, it would have been a \nameref{section: while loops} as discussed in Section \ref{section: while loops}.

Listing \ref{lst:Repeat until loop} shows a minimal example where we test \register{D0} and it it is zero, we don't wish to execute the loop body---effectively \texttt{REPEAT\ldots{}UNTIL D0 = 0}.

\begin{lstlisting}[caption={REPEAT UNTIL \ldots},label={lst:Repeat until loop}]
repeatLoop
    ; Do loop body code here
    ...
    
until
    ; Do condition test last.
    tst.l d0
    brne repeatLoop
    
    ; Continue from here.
    ...
\end{lstlisting}

You will note that the test of \register{D0} is negated in order to keep looping around. 

This construct is also known as the \texttt{Do\ldots{}Until loop}.

\section{While \ldots\protect End While Loop}\label{section: while loops}

This is another structure that isn't available on the QL, but is in other languages. It is similar to the \nameref{section: repeat until loops} discussed in Section \ref{section: repeat until loops} but with the condition test at the top of the loop rather than at the end. As the test is carried out at the top of the loop, the loop body may not be executed if the condition is true on entry to the loop.

Listing \ref{lst:while loop} shows a minimal example of this structure---effectively, \texttt{While D0 = 0}.

\begin{lstlisting}[caption={WHILE \ldots END WHILE},label={lst:while loop}]
whileLoop
    ; Do condition test first.
    tst.l d0
    brne exitWhile

    ; Do loop body code here
    ...
    
endWhile
    bra whileLoop

exitWhile
    ; Continue from here.
    ...
\end{lstlisting}

You will, hopefully, note that the condition has again been reversed, we loop back to \asmlabel{whileLoop} if \register{D0} is not equal to zero

\section{For \ldots\protect End For Loop}

\texttt{FOR} loops are available on the QL. In Assembly Language we can count up or down, as we can in S*BASIC, but the CPU in the QL, and emulators, has a looping instruction---\opcode{dbra} also known as \opcode{dbf}---Decrement  and branch until false. As the terminating condition is False, the loop will only terminate when the counter register decrements from zero to $-1$.

Counting upwards can't use that instruction though. Listing \ref{lst: for loop counting upwards} shows a small example of a FOR loop in Assembly. The condition to determine the end of the loop is at the top of the loop. This example is effectively \texttt{FOR d0 = 0 TO 10 STEP 1\ldots{}END FOR}.

\begin{lstlisting}[caption={FOR \ldots END FOR forwards STEP},label={lst: for loop counting upwards}]
forLoop
    clr.b d0                ; For do = 0 ...
    
nextFor
    cmpi.b #10,d0           ; To 10 ...
    beq exitFor
    
    ; Do loop body code here.
    ...

endFor
    addq.b #1,d0            ; ... Step 1
    bra nextFor

exitFor
    ; Continue from here
    ...
\end{lstlisting}

Counting downwards, of course, can make use of \opcode{dbra} as Listing \ref{lst: for loop counting downwards} shows. This example is effectively \texttt{FOR d0 = 10 TO 0 STEP -1\ldots{}END FOR} but uses values from 9 down to -1.

\begin{lstlisting}[caption={FOR \ldots END FOR backwards STEP},label={lst: for loop counting downwards}]
forLoop
    moveq.b #10-1,d0        ; For d0 = 10 ...
    
nextFor
    tst.b,d0                ; To 0 ...
    beq exitFor
    
    ; Do loop body code here.
    ...
    
endFor
    dbra d0,nextFor         ; Step -1
    
exitFor
    ; Continue from here
    ...
\end{lstlisting}

In Assembly, it is just as simple to use the actual values from 10 to 1 in the register but the \opcode{dbra} instruction wouldn't then be possible as that instruction, as we know, terminates the loop when the register value is -1 as opposed to zero.

\section{For \ldots\protect Next \ldots\protect End For Loop}

Sometimes, in a \texttt{FOR} statement, we want to skip one or more values in the loop. In those cases we make a test and if true, we call \texttt{NEXT} in our loop body. Listing \ref{lst: for loop with next} shows an example where we execute \texttt{NEXT} if the value in \register{D0} is 7.

\begin{lstlisting}[caption={FOR \ldots NEXT \ldots END FOR},label={lst: for loop with next}]
forLoop
    clr.b d0                ; For do = 0 ...
    
nextFor
    cmpi.b #10,d0           ; To 10 ...
    beq exitFor
    
    cmpi.b #7,d0            ; Is D0 = 7? If so ...
    bra nextFor             ; ... NEXT.
    
    ; Do loop body code here.
    ...
    
endFor
    addq.b #1,d0            ; ... Step 1
    bra nextFor
    
 exitFor
    ; Continue from here
    ...
\end{lstlisting}

\section{If \ldots\protect End If}

The \texttt{IF} statement can lead you down a rabbit hole or two! There are all sorts of potential conditions, signed and unsigned. For more details, see this issue's \nameref{chpater: beginners corner} on page \pageref{chpater: beginners corner} for details. For the descriptions below, I'll be sticking to equals and not equals, as these are the simplest to explain and understand; plus we don't have to consider signs either!

Listing \ref{lst: if then endif} has the code for a simple \texttt{IF D0 = 10 THEN\ldots{}END IF} statement.

\begin{lstlisting}[caption={If \ldots THEN \ldots END IF},label={lst: if then endif}]
ifTest
    ; If D0.B <> 10 Then skip to endIf.
    cmpi.b #10,d0
    bne endIf
    
thenSection
    ; Do code for D0 = 10 here.
    ...
    
endIf
    ; Continue from here.
    ...
\end{lstlisting}

Once again, you will notice that we reverse the flow and the test. The code is actually testing that \register{D0} is not equal to 10 and skipping the body of the code, rather than testing if it is equal to 10 and executing the code.

\section{If \ldots\protect Else \ldots\protect End If}

Listing \ref{lst: if then else endif} shows an \texttt{IF D0 = 10 THEN\ldots{}ELSE\ldots{}END IF} statement. There's only a small difference here, after executing the code when \register{D0} was equal to 10, there's a jump over the code that gets executed when \register{D0} wasn't equal to 10.


\begin{lstlisting}[caption={If \ldots THEN \ldots ELSE \ldots END IF},label={lst: if then else endif}]
ifTest
    ; If D0.B <> 10 Then skip to elseSection.
    cmpi.b #10,d0
    bne elseSection
    
thenSection
    ; Do code for D0 = 10 here.
    ...
    
    ; And when done, skip the ELSE clause.
    bra endIf
    
elseSection   
    ; Do code for D0 <> 10 here.
    ... 
    
endIf
    ; Continue from here.
    ...
\end{lstlisting}

\section{Exit and Next}

You've seen \texttt{exit} already. Based on some condition or other, we \texttt{exit} the construct by jumping to just past the end of it; or we jump back to the beginning for \texttt{next}. Listing \ref{lst: exit and next examples} shows simple examples of each.

\begin{lstlisting}[caption={EXIT and NEXT},label={lst: exit and next examples}]
loopStart
    ; Do some loop body stuff.
    ...
    
    ; Do a `next loop' if D0 = 0 at this point.
    tst.l d0                ; Is D0 = 0?
    beq loopStart           ; If so, next loop
    
    ; Do more loop body stuff.
    ...
    
    ; Do an `exit loop' if D7 = 0 at this point.
    tst.l d7                ; Is D7 = 0?
    beq loopExit            ; If so, exit loop  
    
loopEnd
    bra loopStart           ; Let's go round again!
    
loopExit
    ; Rest of code ....
    ...
\end{lstlisting}

\section{Summary}

Well, maybe this wasn't such a ``quickie'' as I first thought. However, we can now write code in a sort of structured manner. I suspect most of us ``seasoned'' Assembly programmers have been doing this sort of thing automatically for years anyway! But maybe someone will have learned something. I did!