
\chapter{News}



\section{George Gwilt}

Those of you who know of my old scribblings in \emph{QL Today} and what used to be my QDOS Internals web site, will know of George Gwilt. He was an incredible programmer and knew almost everything about Assembly Language for the QL. He maintained the Turbo SuperBASIC compiler for years and assisted Marcel in getting QPC to emulate a 68020 processor. His Assemblers, \emph{gwasl} and \emph{gwass} are tools I have been using for years.

Sadly, in March 2024, George passed away. His son, Richard, posted a message to the QL Forum with the details. George, although he hasn't been heard of for a while, will be sadly missed by many people. I shall certainly miss his input to almost everything I ever wrote about QL Assembly Language.

RIP George.

George's son posted again with details of a hoard of QL ``stuff'' that they had found in a room at George's house. This was to be taken to the tip if nobody wanted it. Graeme Gregory drove to Edinburgh to rescue everything and is\footnote{Or ``was'' depending on when you read this!} going to catalogue everything and help with disposal/sale/donations as appropriate.

\section{Long Delay Since Previous Issue}

You may have noticed a long delay since the previous issue? Well, My apologies for that, I've been a little busy. Some of you know that I had a book published in April 2020---\emph{Arduino Software Internals}---well, the publishers requested that I do a second edition to cover changes in the Arduino world since 2020.

Unfortunately, they now only accept manuscripts written with Microsoft Word, or \LaTeX{} but using their own document template. My book was originally written using ASCIIDoctor---a completely different source file format---so had to be converted over to \LaTeX{}. I haven't use Microsoft Word since I was working at Morrisons Supermarket's head office some years back. Word would trash my documents on a regular basis, so I tend to avoid it whenever possible!

\emph{Arduino Software Internals 2nd Edition} is now at the production stage, so I'm available for other things now.

My second Arduino book---\emph{Arduino Interrupts}---was published in December 2023. That was actually originally written in \LaTeX{}---hooray!---but using a different template---boo!---so needed some work done to convert it to the correct template. That wasn't as bad as a total rewrite, but still took a while.

Luckily, my third Arduino book---\emph{Arduino Assembly Language}---was only at the beginning the writing stage, so switching templates wasn't too much of a problem.

Unfortunately, all this work, plus the fact that we are selling our house and hopefully, moving back to Scotland, has put any QL and eMagazine work on the back burner. Sorry.


\section{Qdosmsq.dunbar-it.co.uk}

A while back, Alison and I closed our IT business and ``retired''. I had been keeping my hosting payments going even after closing down the business, but in February 2024 I decided that enough was enough and cancelled my hosting contract. My blog, \url{http://dunbar-it.co.uk/blog} was moved to a new host on Github. It's now at \url{http://blog.dunbar-it.co.uk}\footnote{Although, I noticed recently that some images are not displaying so I need to fix that at some stage!}. Yes, I know, I'm not ``https'' secure, but my ISP requires paying for that and I'm a Scotsman living in Yorkshire!!!

My old \url{http://qdosmsq.dunbar-it.co.uk} website is down for the foreseeable future. I have a backup, of course, and something will return, online, ``soon''. If you try to access it at the moment, you will simply get a message that the site cannot be found. I haven't set anything up yet as I may need to extract the data from the Wiki format that I was using previously.

One inadvertent foul-up of cancelling my hosting plan, is the fact that I have lost my online list of subscribers to this eMagazine. I won't now be able to send out emails like I used to, when a new issue is available. I do have some backups but I need to extract the data from the backend database, which might take a while.

Also, the various contact email addresses---\url{xxxxx@qdosmsq.dunbar-it.co.uk}---that I used to use are now also gone. Any feedback on the topics in this and future issues should be directed at the email address in the \nameref{section: feedback} section on page \pageref{section: feedback}.

\section{QDOS Companion}

Derek Stewart has scanned Andy Pennell's \emph{QDOS Companion} book
and made it available as a PDF file. Andy gave his blessing to this,
in case anyone is worried about copyright. There's a thread on the
QL Forum, \href{https://qlforum.co.uk/viewtopic.php?f=12&t=4137}{here}\footnote{\url{https://qlforum.co.uk/viewtopic.php?f=12\&t=4137}},
which you can follow for updates and corrections etc.

Derek has produced a PDF file containing the book, in as near to its
original, format as possible, and the source code is also available
as a plain Libre Office ODF file, in case you wish to add corrections
and regenerate your own PDF file.

Files can be downloaded from the Sinclair QL account on GitHub, \href{https://github.com/SinclairQL/QDOS-Companion}{here}\footnote{\url{https://github.com/SinclairQL/QDOS-Companion}}.
To do so, simply click on the file you want to download, then click
the ``download'' button on the screen that appears next. If you
are after the PDF, don't wait for it to display, just click the button---it 
doesn't need to display in full to be downloaded.

A number of corrections have been made to the book already.

\section{QL Advanced User Guide}

Derek has also scanned Adrian Dickens' \emph{QL Advanced User Guide}
from 1984. Unfortunately, at the time of writing, Derek's requests
to Adder Technologies -- as Adder Publications is now known -- have
so far fallen on deaf ears. Derek is requesting permission to put
his scans, and corrections etc, of the book into the public domain.
It is not possible for Derek to do this without permission due to
copyright issues. Hopefully, Derek's persistence will pay off and
the book will soon be made available.
