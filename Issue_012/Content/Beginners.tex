
\chapter{Beginner's Corner}\label{chpater: beginners corner}

\section{Introduction}

In this issue of Beginner's Corner, we are taking a look at condition
codes and, unavoidably, signed and unsigned numbers and tests. I say
``unavoidably'' as I often find myself wondering which of the condition
codes are signed and which are unsigned---and it all depends on
what type of numbers you are dealing with.

\section{The Numbers}

You need to know about numbers first, before we dive into the condition
codes. 

\emph{Question}: If a register holds the byte value 255
is it signed or unsigned?

\emph{Answer}: Yes!

It can be either because it actually depends on what you, the programmer,
want it to be. In many SMSQ/E trap calls, there needs to be a timeout.
This is usually a positive value but if infinite timeout is requested,
the value $-1$ is used. $255$ in Decimal is the same bit pattern
as $-1$ in Decimal. So how does the system know the difference between
$-1$ and $255$?

The leftmost bit of the number, bit $31$; $15$; or $7$, for long;
word; or byte values, represents the sign of the number. 

If the sign bit is a $1$, the number is negative, if it is a zero,
the number is positive. 

This representation of signed numbers is known as 2'\emph{s Compliment}.
There is a lot of detail and explanation on \href{https://en.wikipedia.org/wiki/Two\%27s_complement}{Wikipedia}\footnote{\url{https://en.wikipedia.org/wiki/Two\%27s\_complement}}
-- if you are interested. 

For unsigned numbers in $B$ bits, the range of values is $0$ through
$2^{B}-1$, while for a signed number, with the same number of bits,
the range is $-\frac{2^{B}}{2}$ through $\frac{2^{B}}{2}-1$.

From this, we can work out that for $8$; $16$; and $32$ bit numbers,
byte; word; and long, we get:
\begin{itemize}
\item 8 bits: signed values range from $0$ through $255$;  while unsigned range from $-128$ through $127$.

\item 16 bits: signed values range from $0$ through $65,535$;  while unsigned range from $-32,768$ through $32,767$.

\item 32 bits: signed values range from $0$ through $4,293,967,295$; while unsigned range from $-2,147,483,648$ through $2,147,483,647$.
\end{itemize}

When you look at a number in binary, it's \emph{relatively} easy to determine the decimal
value simply by adding up all the powers of two where there is a 1
bit present, for example:

10100101 is $2^{7}+2^{5}+2^{2}+2^{0}=165$.

If the number is signed, then it's slightly more complicated as there
are at least three different methods. 

\textbf{Option 1}: You take the negative of the leftmost power of two bit---bit $31$; $15$; or $7$---and add
all the remaining powers of two where a 1 bit is present. For example:
\begin{itemize}
\item 10100101 is $-2^{7}+2^{5}+2^{2}+2^{0}=-91$.
\end{itemize}
\textbf{Option 2}: Flip all the bits, convert to an unsigned value,
add 1 and change the sign to make the value negative, for example:
\begin{itemize}
\item 10100101 flipped is 01011010;
\item Converted to an unsigned value and adding 1 gives $2^{6}+2^{4}+2^{3}+2^{1}+1=91$;
\item Changing the sign gives $-91$ as before.
\end{itemize}
\textbf{Option 3}: Calculate the unsigned value, and subtract $2^{B}$ where $B$ is the number of bits. For example:
\begin{itemize}
\item 10100101 is still $2^{7}+2^{5}+2^{2}+2^{0}=165$;
\item The number of bits is eight and we know $2^{8}$ is $256$;
\item Subtract $256$ from $165$ to get $-91$;
\end{itemize}
Pick the version you prefer.

Right, that's enough about numbers, signed or otherwise, for now at
least!

\section{The Flags}

So, moving slightly ahead, we now need to think about the Status Register,
often known as SR, where the MC680xx CPUs hold their status flags.
Some of these are available to the programmer for use in code to determine
the outcome of a calculation; a register load; a test; and so on.

If you need more information, grab hold of \href{https://github.com/NormanDunbar/QLAssemblyLanguageBook/releases/latest}{my eBook}\footnote{\url{https://github.com/NormanDunbar/QLAssemblyLanguageBook/releases/latest}},
based on many years of writing Assembly Language articles for the,
sadly defunct, \emph{QL Today} magazine. That will explain all.

In the meantime, all you need to know at the moment is that various
flags in the SR are set or cleared according to the results of some
operation. If the flags are set as the result of an arithmetic operation;
or by loading a value into a register, for example, then some flags
indicate whether the number is signed or unsigned---in a roundabout
way!

Ok, I lied, slightly, sorry. What the flags indicate is whether the
number is positive, negative -- amongst others -- regardless of
whether you intended the number to be signed or unsigned. Table \ref{tab:MC680xx-Status-Register}
shows the full set of flags and Table \ref{tab:MC680xx-Condition-Codes}
shows what they are set to for certain conditions which can be tested.

\begin{table}[!h]
\begin{centering}
\begin{tabular}{|c|c|>{\raggedright}p{0.7\textwidth}|}
\hline 
\textbf{Flag} & \textbf{Flag Name} & \textbf{Description}\tabularnewline
\hline 
\hline 
X & Extend & Usually the same as Carry, but used separately. (The manual is no
help!)\tabularnewline
\hline 
N & Negative & Indicates if the top bit is set or clear.\tabularnewline
\hline 
Z & Zero & Indicates the last operation resulted in a zero result.\tabularnewline
\hline 
V & Overflow & The last operation caused arithmetic overflow.\tabularnewline
\hline 
C & Carry & The last operation resulted in a carry.\tabularnewline
\hline 
\end{tabular}
\par\end{centering}
\caption{MC680xx Status Register\label{tab:MC680xx-Status-Register}}
\end{table}


\section{The Condition Codes}

\begin{table}[!h]
\begin{centering}
\begin{tabular}{|c|>{\centering}p{0.35\textwidth}|>{\centering}p{0.3\textwidth}|c|}
\hline 
\textbf{Condition} & \textbf{Flags} & \textbf{Description} & \textbf{Signed}\tabularnewline
\hline 
\hline 
CC & C=0 & Carry Clear & U\tabularnewline
\hline 
CS & C=1 & Carry Set & U\tabularnewline
\hline 
EQ & Z=1 & Equal/Zero & U and S\tabularnewline
\hline 
NE & Z=0 & Not Equal/Not Zero & U and S\tabularnewline
\hline 
HI & C=0 and Z=0 & Higher & U\tabularnewline
\hline 
LS & C=1 or Z=1 & Lower or Same & U\tabularnewline
\hline 
PL & N=0 & Plus/Positive & S\tabularnewline
\hline 
MI & N=1 & Minus/Negative & S\tabularnewline
\hline 
GE & (N=1 and V=1) or\\
(N=0 and V=0) & Greater Than or Equal & S\tabularnewline
\hline 
GT & (N=1 and V=1 and Z=0) or \\
(N=0 and V=0 and Z=0) & Greater Than & S\tabularnewline
\hline 
LE & (Z=1) or \\
(N=1 and V=0) or \\
(N=0 and V=1) & Less Than or Equal & S\tabularnewline
\hline 
LT & (N=1 and V=0) or \\
(N=0 and V=1) & Less Than & S\tabularnewline
\hline 
VC & V = 0 & Overflow Clear & S \tabularnewline
\hline
VS & V = 1 & Overflow Set & S \tabularnewline
\hline
F & N/A & False & U and S \tabularnewline
\hline
T & N/A & True & U and S \tabularnewline
\hline
\end{tabular}
\par\end{centering}
\caption{MC680xx Condition Codes\label{tab:MC680xx-Condition-Codes}}
\end{table}


\section{Condition Codes}

You can see from Table \ref{tab:MC680xx-Condition-Codes} that some condition codes are used with signed operations while others are unsigned. It's your  code and your data, so you are the one who decides which to use. If you are performing signed arithmetic, for example, then you should be using the signed condition codes; for unsigned arithmetic, the unsigned codes. 

The condition codes can be used to determine if the code will branch after an operation---\texttt{CMP}; \texttt{SUB}; \texttt{MOVE}; etc---using the \texttt{Bcc} or \texttt{DBcc} instructions as appropriate.

The \texttt{Bcc} instruction simply branches if the condition codes are set as desired by the instruction.

Listings \ref{lst: signed condition codes} and \ref{lst: unsigned condition codes} show a few examples of the use of the \texttt{Bcc} instruction for signed and unsigned values.

\begin{lstlisting}[caption={Signed condition codes},label={lst: signed condition codes}]
    ; Signed value in D4.
    tst.l d4        ; Sets the flags according to value of D4.
    beq d4IsZero    ; Branch if D4 is zero. (Signed/Unsigned)
    blt d4IsMinus   ; Branch if D4 is negative. (Signed)
    bgt d4IsPlus    ; Branch if D4 is positive. (Signed)
\end{lstlisting}    

\begin{lstlisting}[caption={Unsigned condition codes},label={lst: unsigned condition codes}]
    ; Unsigned value in D4.
    tst.l d4        ; Sets the flags according to value of D4.
    beq d4IsZero    ; Branch if D4 is zero. (Signed/Unsigned)
    bls d4IsMinus   ; Branch if D4 is negative. (Unsigned)
    bhi d4IsPlus    ; Branch if D4 is positive. (Unsigned)
\end{lstlisting}

You should note that while signed condition codes allow for a test for greater than; greater than or equal; equal; less than; and less than or equal, unsigned values only have higher; equal; lower or same. This means that in Listing \ref{lst: unsigned condition codes}, we had to do the check for equality first because the test for lower would include equal had we done the \texttt{bls} instruction first.

In the above examples, we tested the actual value in \register{D4} but we could have tested after a \texttt{CMP} or \texttt{SUB} instruction, as per the example code in Listing \ref{lst: signed condition codes with CMP}.

\begin{lstlisting}[caption={Signed condition codes with CMP},label={lst: signed condition codes with CMP}]
    ; Signed value in D4.
    cmp.l d4,d5     ; Sets the flags according to value of D5 minus D4.
    beq d4equals    ; Branch if D4 = D5.
    blt d4IsLess    ; Branch if D4 < D5. (Signed)
    bgt d4IsMore    ; Branch if D4 > D5. (Signed)
\end{lstlisting}  

The \texttt{DBcc} family of instructions are similar but have the advantage of decrementing a counter register and branching \emph{until}---not \emph{while}---the condition is true. \texttt{DBEQ D0, someLabel}, for example, means that register \register{D0} will be decremented by 1 and if the Z flag is set, the branch \emph{will not} take place. If Z is clear, the branch will be taken. Confused? I was too when I first started learning MC680xx Assembly.

If you remember that \text{DBcc} means ``decrement and branch \emph{until} `cc' is true'' then you won't go wrong.

\texttt{DBF} also known as \texttt{DBRA} (decrement and branch always), and \texttt{DBT} need a little extra explanation. \texttt{DBF}/\texttt{DBRA} always branch---because the condition, False, is never, ever, True. These instructions only stop branching when the counter register reaches $-1$. 

\texttt{DBT} is a tad weird to say the least, it means that the branch will never be taken as the test always is true.


\section{Assembling Code}

\textbf{Note:} This section will remain in the Beginners Corner until the end of the universe; or until I stop writing the magazine! Not everyone is au-fait with assembling code, especially as there are quite a few assemblers for the QL.

Assembling the code requires that you have an assembler installed.
See Section \ref{sec:Tools-and-Manuals}, \nameref{sec:Tools-and-Manuals}
for links to allow you to obtain the assemblers that I frequently
use for this eMagazine.

Once you have installed an assembler, assembling the code is simple
and is described in the following sections. Pick the section for your
own assembler.

\subsection{With GWASS}
\begin{itemize}
\item \texttt{EXEC} \path{win1_gwass60_bin} to start the assembler;
\item Select option 1 to start assembling;
\item Type in the filename: \path{ram1_mycode_asm} for example.
\item Wait.
\end{itemize}

\subsection{With Qmac}

To pass the commands directly via the command line:
\begin{itemize}
\item \texttt{EX \path{win1_qmac};~\textquotedbl\path{ram1_mycode_asm} -data 2048 -filetype 1 -nolink}~\\
\texttt{-bin \path{ram1_mycode_bin}\textquotedbl}
\end{itemize}
\textbf{Note:} The above command should be typed on one line - I've
had to split it for the PDF page width. 

Alternatively, you can type the command interactively:
\begin{itemize}
\item \texttt{EX \path{win1_qmac}}
\item Type the options: \texttt{\path{ram1_mycode_asm} -data 2048 -filetype
1 -nolink }~\\
\texttt{-bin \path{ram1_mycode_bin}}
\item Wait
\end{itemize}
What you are doing here, in both cases, is telling the assembler to:
\begin{itemize}
\item Assemble the source file \path{ram1_mycode_asm};
\item Create an executable file (\texttt{-filetype 1}), with 2,048 bytes
of data space (\texttt{-data 2048});
\item Do not invoke the linker as it is not needed because everything is
in the same source file (\texttt{-nolink});
\item Create the output file named \path{ram1_mycode_bin}---which
will be created with its name in uppercase regardless of what letter case you type here!
\end{itemize}

\subsection{Executing the Code}

After a successful assemble, and regardless of which assembler you
used, \path{ram1_mycode_bin} will be the executable job. To run it, you have a choice:
\begin{itemize}
\item \texttt{EXEC} \path{ram1_mycode_bin}
\item \texttt{EXEC\_W} \path{ram1_mycode_bin}
\end{itemize}

The first will start the job running and return immediately to S*BASIC. You will never see any error codes that the job returns to S*BASIC. The second will start the job running and wait for it to complete before returning to S*BASIC. In this case, if the job returns an error code, you will see the error message displayed in channel \#0 when the job terminates.

Of course, this assumes that you have written a multitasking job in your code. If the code is \texttt{CALL}able from S*BASIC, then that's different.



\section{Summary}

Hopefully, this chapter will have gone some way in helping you understand the various tests and conditions that you might need to cater for in your code. Knowing this will also help in writing structured Assembly Language programs. 

The next chapter covers this very topic.

\section{Tools and Manuals\label{sec:Tools-and-Manuals}}

Get hold of the SMSQ/E Reference Manual from \href{https://www.wlenerz.com/qlstuff/\#qdosms}{Wolfgang Lenerz's web site}\footnote{\url{https://www.wlenerz.com/qlstuff/\#qdosms}}
for the official version. Alternatively, there are copies on Dilwyn's
pages:
\begin{itemize}
\item \href{http://www.dilwyn.me.uk/docs/manuals/QDOS_SMS\%20Reference\%20Guide\%20v4.5.pdf}{Here}\footnote{\url{http://www.dilwyn.me.uk/docs/manuals/QDOS\_SMS\%20Reference\%20Guide\%20v4.5.pdf}}
for the PDF for version 4.5; or

\item \href{http://www.dilwyn.me.uk/docs/manuals/QDOS_SMS\%20Reference\%20Guide\%20v4.5.odt}{Here}\footnote{\url{http://www.dilwyn.me.uk/docs/manuals/QDOS\_SMS\%20Reference\%20Guide\%20v4.5.odt}}
for the ODT (Libre Office) file for version 4.5.

\item If you want to ensure that you have the most recent versions of those
files, \href{https://www.wlenerz.com/qlstuff/\#qdosms}{Wolfgang Lenerz's web site}\footnote{\url{https://www.wlenerz.com/qlstuff/\#qdosms}}
is the place to look.
\end{itemize}


Get hold of GWASS \href{https://github.com/SinclairQL/GeorgeGwilt/blob/main/Gwass/gwassp22.zip}{here}\footnote{\url{https://github.com/SinclairQL/GeorgeGwilt/blob/main/Gwass/gwassp22.zip}}
for 68020 processors. Click the link then click on the funny looking downward pointing arrow on the far right to start the download.

Download Qmac \href{http://www.dilwyn.me.uk/asm/gst/gstmacroquanta.zip}{here}\footnote{\url{http://www.dilwyn.me.uk/asm/gst/gstmacroquanta.zip}}
for 68008 processors.
