
\chapter{Beginner's Corner}

\section{Introduction}

In this issue of Beginner's Corner, we are taking a look at Condition
codes and, unavoidably, signed and unsigned numbers and tests. I say
`unavoidably' as I often find myself wondering which of the condition
codes are signed and which are unsigned -- and it all depends on
what type of numbers you are dealing with.

\section{The Numbers}

You need to know about numbers first, before we dive into the condition
codes. 

\emph{Question}: If a register holds the byte value 255\textsubscript{Dec}/FF\textsubscript{Hex}/11111111\textsubscript{Bin}
is it signed or unsigned?

\emph{Answer}: Yes!

It can be either because it actually depends on what you, the programmer,
wants it to be. In many SMSQ/E trap calls, there needs to be a timeout.
This is usually a positive value but if infinite timeout is requested,
the value $-1$ is used. $255$ in Decimal is the same bit pattern
as $-1$ in Decimal. So how does the system know the difference between
$-1$ and $255$?

The leftmost bit of the number, bit $31$, $15$ or $7$, for long,
word or byte values, represents the sign of the number. 

If the sign bit is a $1$, the number is negative, if it is a zero,
the number is positive. 

This representation of signed numbers is known as 2'\emph{s Compliment}.
There is a lot of detail and explanation on \href{https://en.wikipedia.org/wiki/Two\%27s_complement}{Wikipedia}\footnote{https://en.wikipedia.org/wiki/Two\%27s\_complement}
-- if you are interested. 

For unsigned numbers in $B$ bits, the range of values is $0$ through
$2^{B}-1$, while for a signed number, with the same number of bits,
the range is $-\frac{2^{B}}{2}$ through $\frac{2^{B}}{2}-1$.

From this, we can work out that for $8$, $16$ and $32$ bit numbers,
byte, word and long, we get:
\begin{itemize}
\item 8 bit bytes, range from $0$ through $255$ unsigned, $-128$ through
$127$ signed;
\item 16 bit words, range from $0$ through $65,535$ unsigned, $-32,768$
through $32,767$ signed;
\item 32 bit longs, range from $0$ through $4,293,967,295$ unsigned, $-2,147,483,648$
through $2,147,483,647$ signed.
\end{itemize}
When you look at a number in binary, it's easy to determine the decimal
value simply by adding up all the powers of two where there is a 1
bit present, for example:

10100101 is $2^{7}+2^{5}+2^{2}+2^{0}=165$.

If the number is signed, then it's slightly more complicated as there
are at least three different methods. 

\textbf{Option 1}: You take the leftmost power of two and subtract
all the remaining powers of two where a 1 bit is present. For example:
\begin{itemize}
\item 10100101 is $2^{7}-\left(2^{5}+2^{2}+2^{0}\right)=-91$.
\end{itemize}
\textbf{Option 2}: Flip all the bits, convert to an unsigned value,
add 1 and change the sign to make the value negative, for example:
\begin{itemize}
\item 10100101 flipped is 01011010;
\item Converted to an unsigned value gives $2^{6}+2^{4}+2^{3}+2^{1}+1=91$;
\item Changing the sign gives $-91$ as before.
\end{itemize}
\textbf{Option 3}: Calculate the unsigned value, and subtract from
$2^{B}$ where $B$ is the number of bits, then change the sign to
make the value negative. For example:
\begin{itemize}
\item 10100101 is still $2^{7}+2^{5}+2^{2}+2^{0}=165$;
\item The number of bits is 8 and we know $2^{8}$ is $256$;
\item Subtract $165$ from $256$ to get $91$;
\item Changing the sign gives $-91$, yet again!
\end{itemize}
Pick the version you prefer.

Right, that's enough about numbers, signed or otherwise, for now at
least!

\section{The Flags}

So, moving slightly ahead, we now need to think about the Status Register,
often known as SR, where the MC680xx CPUs hold their status flags.
Some of these are available to the programmer for use in code to determine
the outcome of a calculation, a register load, a test and so on.

If you need more information, grab hold of \href{https://github.com/NormanDunbar/QLAssemblyLanguageBook/releases/latest}{my eBook}\footnote{https://github.com/NormanDunbar/QLAssemblyLanguageBook/releases/latest},
based on many years of writing Assembly Language articles for the,
sadly defunct, \emph{QL Today} magazine. That will explain all.

In the meantime, all you need to know at the moment is that various
flags in the SR are set or cleared according to the results of some
operation. If the flags are set as the result of some arithmetic operation,
or by loading a value into a register, for example, then some flags
indicate whether the number is signed or unsigned -- in a roundabout
way!

Ok, I lied, slightly, sorry. What the flags indicate is whether the
number is positive, negative -- amongst others -- regardless of
whether you intended the number to be signed or unsigned. Table \ref{tab:MC680xx-Status-Register}
shows the full set of flags and Table \ref{tab:MC680xx-Condition-Codes}
shows what they are set to for certain conditions which can be tested.

\begin{table}[!h]
\begin{centering}
\begin{tabular}{|c|c|>{\centering}p{0.7\textwidth}|}
\hline 
\textbf{Flag} & \textbf{Flag Name} & \textbf{Description}\tabularnewline
\hline 
\hline 
X & Extend & Usually the same as Carry, but used separately. (The manual is no
help!)\tabularnewline
\hline 
N & Negative & Indicates if the top bit is set or clear.\tabularnewline
\hline 
Z & Zero & Indicates the last operation resulted in a zero result.\tabularnewline
\hline 
V & Overflow & The last operation caused arithmetic overflow.\tabularnewline
\hline 
C & Carry & The last operation resulted in a carry.\tabularnewline
\hline 
\end{tabular}
\par\end{centering}
\caption{MC680xx Status Register\label{tab:MC680xx-Status-Register}}
\end{table}


\section{The Condition Codes}

\begin{table}[!h]
\begin{centering}
\begin{tabular}{|c|>{\centering}p{0.35\textwidth}|>{\centering}p{0.3\textwidth}|c|}
\hline 
\textbf{Condition} & \textbf{Flags} & \textbf{Description} & \textbf{Signed}\tabularnewline
\hline 
\hline 
CC & C=0 & Carry Clear & U\tabularnewline
\hline 
CS & C=1 & Carry Set & U\tabularnewline
\hline 
EQ & Z=1 & Equal/Zero & U and S\tabularnewline
\hline 
NE & Z=0 & Not Equal/Not Zero & U and S\tabularnewline
\hline 
HI & C=0 and Z=0 & Higher & U\tabularnewline
\hline 
LS & C=1 or Z=1 & Lower or Same & U\tabularnewline
\hline 
PL & N=1 & Plus/Positive & S\tabularnewline
\hline 
MI & N=1 & Minus/Negative & S\tabularnewline
\hline 
GE & (N=1 and V=1) or\\
(N=0 and V=0) & Greater Than or Equal & S\tabularnewline
\hline 
GT & (N=1 and V=1 and Z=0) or \\
(N=0 and V=0 and Z=0) & Greater Than & S\tabularnewline
\hline 
LE & (Z=1) or \\
(N=1 and V=0) or \\
(N=0 and V=1) & Less Than or Equal & S\tabularnewline
\hline 
LT & (N=1 and V=0) or \\
(N=0 and V=1) & Less Than & S\tabularnewline
\hline 
\end{tabular}
\par\end{centering}
\caption{MC680xx Condition Codes\label{tab:MC680xx-Condition-Codes}}
\end{table}


\section{Condition Codes}

Signed and unsigned condition codes.

Bcc, DBcc, CMP. Explain.

Signed: EQ, GE, GT, LS, LT, MI, NE, PL, 

Unsigned: EQ, HI, LE, NE, 

Others: F, T, VC, VS,

\section{Assembling the Code}

Assembling the code requires that you have an assembler installed.
See Section \ref{sec:Tools-and-Manuals}, \nameref{sec:Tools-and-Manuals}
for links to allow you to obtain the assemblers that I frequently
use for this eMagazine.

Once you have installed an assembler, assembling the code is simple
and is described in the following sections. Pick the section for your
own assembler.

\subsection{With GWASS}
\begin{itemize}
\item \texttt{EXEC} \texttt{gwass60\_bin} to start the assembler;
\item Select option 1 to start assembling;
\item Type in the filename: \texttt{ram1\_??????????\_asm}
\item Wait.
\end{itemize}

\subsection{With Qmac}

To pass the commands directly via the command line:
\begin{itemize}
\item \texttt{EX qmac;``ram1\_??????????\_asm -data 2048 -filetype 1 -nolink
}~\\
\texttt{-bin ram1\_??????????\_bin''}
\end{itemize}
\textbf{Note:} The above command should be typed on one line - I've
had to split it for the PDF page width. 

Alternatively, you can type the command interactively:
\begin{itemize}
\item \texttt{EX qmac}
\item Type the options: \texttt{ram1\_??????????\_asm -data 2048 -filetype
1 -nolink }~\\
\texttt{-bin ram1\_??????????\_bin}
\item Wait
\end{itemize}
What you are doing here, in both cases, is telling the assembler to:
\begin{itemize}
\item Assemble the source file \texttt{ram1\_??????????\_asm};
\item Create an executable file (\texttt{-filetype 1}), with 2,048 bytes
of data space (\texttt{-data 2048});
\item Do not invoke the linker as it is not needed because everything is
in the same source file (\texttt{-nolink});
\item Create the output file named \texttt{ram1\_??????????\_bin} -- which
will be in uppercase regardless of what you type here!
\end{itemize}

\subsection{Executing the Code}

After a successful assemble, and regardless of which assembler you
used, \texttt{ram1\_??????????\_bin} will be the executable job. To
run it:
\begin{itemize}
\item \texttt{EX} \texttt{ram1\_??????????\_bin}
\end{itemize}
On a successful execution, ..................................................

\section{Summary}

That, hopefully, will help make writing structured Assembly Language
programs more manageable. Let me know if there's anything you didn't
follow or which may need more explanation, and I'll do my best to
assist.

\section{Tools and Manuals\label{sec:Tools-and-Manuals}}

Get hold of the SMSQ/E Reference Manual from \href{https://www.wlenerz.com/qlstuff/\#qdosms}{Wolfgang Lenerz's web site}\footnote{https://www.wlenerz.com/qlstuff/\#qdosms}
for the official version. Alternatively, there are copies on Dilwyn's
pages:
\begin{itemize}
\item \href{http://www.dilwyn.me.uk/docs/manuals/QDOS_SMS\%20Reference\%20Guide\%20v4.5.pdf}{Here}\footnote{http://www.dilwyn.me.uk/docs/manuals/QDOS\_SMS\%20Reference\%20Guide\%20v4.5.pdf}
for the PDF for version 4.5; or
\item \href{http://www.dilwyn.me.uk/docs/manuals/QDOS_SMS\%20Reference\%20Guide\%20v4.5.odt}{Here}\footnote{http://www.dilwyn.me.uk/docs/manuals/QDOS\_SMS\%20Reference\%20Guide\%20v4.5.odt}
for the ODT (Libre Office) file for version 4.5.
\item If you want to ensure that you have the most recent versions of those
files, \href{https://www.wlenerz.com/qlstuff/\#qdosms}{Wolfgang Lenerz's web site}\footnote{https://www.wlenerz.com/qlstuff/\#qdosms}
is the place to look.
\end{itemize}
Get hold of GWASS \href{http://gwiltprogs.info/gwassp22.zip}{here}\footnote{http://gwiltprogs.info/gwassp22.zip}
for 68020 processors.

Download Qmac \href{http://www.dilwyn.me.uk/asm/gst/gstmacroquanta.zip}{here}\footnote{http://www.dilwyn.me.uk/asm/gst/gstmacroquanta.zip}
for 68008 processors.
