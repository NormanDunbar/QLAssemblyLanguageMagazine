
\chapter{Feedback}

\section{Circular Buffers}

When I announce Issue 9 on the QL Forum, Marcel queried the Circular
Buffer article in the issue. He was wondering if I had simply recreated
the Queue handling features of QDOSMSQ, those being \texttt{IOQ.SETQ},
\texttt{IOQ.TEST}, \texttt{IOQ.PBYT}, and \texttt{IOQ.GBYT}. 

Putting it simply, \emph{probably}! However, as I pointed out, it
was a fun exercise in creating the code from a C++ version, and debugging
a particularly insidious bug where allocating common heap space gets
rounded up!

Thanks Marcel, at least I know one person read it!

\section{Learning Assembly Language}

\emph{TMD2003} wondered about learning Assembly Language and started
\href{https://qlforum.co.uk/viewtopic.php?f=3&t=3976}{a thread}\footnote{https://qlforum.co.uk/viewtopic.php?f=3\&t=3976}
on the QL Forum\footnote{Yes, I know, it's not really feedback on the previous issue of the
eMagazine, but I though it was relevant.}. On the second page of the thread, \emph{Tinyfpga} issued this statement:

\emph{In Basic all I have to do start my journey into the programming
world is to:- Type into QD}

\begin{lstlisting}
OPEN #1,con_ 
OUTLN #1,310,60,50,300 
CLS #1 : BORDER #1,1,7 
INPUT #1,a$
\end{lstlisting}

\emph{I then save the text in a ram disk, press the execute button
and ,\textquotedbl hey presto\textquotedbl , I see an app on my
screen. What could be an easier introduction to programming?}

Well, I decided that that would be a good start to the new Beginner's
feature, so I've taken up \emph{Tinyfpga}'s ``challenge'' in this
issue and converted his \emph{easy introduction to programming} into
Assembly Language.

\section{Wolfgang's Feedback on Label Alignment}

TODO!
