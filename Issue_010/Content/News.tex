
\chapter{News}

\section{New Cover}

I had a note on the QL Forum that some people\footnote{Ok, one person!}
get a bit weird in the head when they look at the cover image for
this eMagazine. It's got something to do with all the holes, or the
appearance of same, so rather than losing any of my valued readers,
I've decided to get a new cover image with far fewer holes! 

It may sound amusing, but \href{https://en.wikipedia.org/wiki/Trypophobia}{Trypophobia}\footnote{https://en.wikipedia.org/wiki/Trypophobia}
is a known phobia, so holes are best avoided if you are a sufferer.
Hopefully the new cover will prevent any induced weirdness in my reader(s).
There's enough weirdness in the author!

\section{My Assembly Book}

In the last issue, I mentioned that \emph{Tinyfpga} had created a
printed version of my Assembly Language articles from the late, and
much missed, \emph{QL Today} magazine. Since then, Tiny (if I can
call him that!) has posted a couple of updates on the book to the
QL Forum at \href{https://qlforum.co.uk/viewtopic.php?f=3&t=3976&start=10\#p44445}{this post}\footnote{https://qlforum.co.uk/viewtopic.php?f=3\&t=3976\&start=10\#p44445}
and \href{https://qlforum.co.uk/viewtopic.php?f=3&t=3976&start=10\#p44446}{this post too}\footnote{https://qlforum.co.uk/viewtopic.php?f=3\&t=3976\&start=10\#p44446}.

It appears that everyone who purchases a copy gets to see the personal
details of the previous customer, which means potentially falling
foul of EU GDPR rules and regulations for personal data. There is
a fix to the problem, but if anyone has a better idea, please do post
the details on the forum in response to the two posts linked above.
Thanks.

\section{Beginner's Corner}

This issue sees the start of a new feature, which will take a look
at Assembly Language programming from a beginner's point of view.
I don't mean getting right back to basics and learning the instructions
etc -- that's nicely covered in the book mentioned above, or the
PDF version which you can download from \href{https://github.com/NormanDunbar/QLAssemblyLanguageBook/releases/latest}{my GitHub repository}\footnote{https://github.com/NormanDunbar/QLAssemblyLanguageBook/releases/latest}. 

What I do mean is a beginner's guide to Assembly Language Tools --
which assembler to use, what about debuggers etc, plus, it occurs
to me that I never delved into QDOS and the various utilities and
traps in my articles for \emph{QL Today}! So that's where we will
be going in the future.

\section{SMSQ/E}

And speaking of delving into QDOS, this issue marks the end of an
era. QDOS is no more, long live SMSQ/E! From now on, I will be converting
myself over to using the SMSQ/E versions of trap calls and vectors
etc. I feel, after some discussion on the forum, that anyone learning
this stuff nowadays -- and there are still some -- should be using
the up to date details. So, no more using \texttt{UT\_CON} as I'll
be using \texttt{OPW\_CON} instead!

For those who need a new manual to cover this new regime, the \emph{QDOS/SMS
Reference Manual} is available from the \href{http://www.dilwyn.me.uk/docs/manuals/index.html}{Sinclair QL Home Page}\footnote{http://www.dilwyn.me.uk/docs/manuals/index.html}
where you can grab a \href{http://www.dilwyn.me.uk/docs/manuals/QDOS_SMS\%20Reference\%20Guide\%20v4.5.pdf}{PDF version}\footnote{http://www.dilwyn.me.uk/docs/manuals/QDOS\_SMS\%20Reference\%20Guide\%20v4.5.pdf}
or a \href{http://www.dilwyn.me.uk/docs/manuals/QDOS_SMS\%20Reference\%20Guide\%20v4.5.odt}{Libre Office version}\footnote{http://www.dilwyn.me.uk/docs/manuals/QDOS\_SMS\%20Reference\%20Guide\%20v4.5.odt}
(ODT) as desired.

\subsection{SMSQ/E 3.38}

A new version of SMSQ/E is now available. Version 3.38 was announced
on 1st November on the \href{https://qlforum.co.uk/viewtopic.php?f=3&t=3957&p=44189\#p44189}{QL Forums}\footnote{https://qlforum.co.uk/viewtopic.php?f=3\&t=3957\&p=44189\#p44189}
and on the QL emailing list. To quote Wolfgang Lenerz:

\emph{SMSQE 3.38 is out now.}

\emph{As usual, you can get it at \href{http://www.wlenerz.com/smsqe/}{http://www.wlenerz.com/smsqe/}.}

\emph{The main news here is that, thanks to Alain Haoui's work, WMAN
can now draw real subwindow indices.}

\emph{The ways this is done is explained in the QPTR manual, the new
version of which can be found at my QL stuff site \href{http://www.wlenerz.com/qlstuff}{http://www.wlenerz.com/qlstuff}.}

\emph{There you can find the QPTR manual (in the documentation section),
and also the new QPTR bin file itself, which also implements index
drawing (in the programming section) as well as two demo/test programs
(one for Basic, made by Alain. Haoui, and one for assembler).}

\section{QPC2 Version 5.01}

Coincidentally with the new release of SMSQ/E, Marcel Kilgus announced
the release of QPC2 v5.01 on 1st November 2021. \href{https://qlforum.co.uk/viewtopic.php?f=19&t=3958\#p44190}{In his own words}\footnote{https://qlforum.co.uk/viewtopic.php?f=19\&t=3958\#p44190}:

\emph{Rejoice, QPC2 got a new release today. As always free as in
\textquotedbl free beer\textquotedbl . Details are here: \href{https://www.kilgus.net/2021/11/01/qpc2-v5-01-and-smsq-e-v3-38-released/}{https://www.kilgus.net/2021/11/01/qpc2-v5-01-and-smsq-e-v3-38-released/}}

And, yes, I know. The chances of a newer release of QPC and/or SMSQ/E
appearing \emph{before} this issue of the eMagazine are pretty high!

\section{And Finally...}

Apologies for the seemingly never ending list of URLs in the footnotes
to the pages in this section. This is mainly done for the reason that
some people actually prefer a printed version of my eMagazine and
it appears that there's a bug in the handling of clickable links on
the paper versions -- they simply don't work.
