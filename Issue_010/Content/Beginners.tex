
\chapter{Beginners' Corner}

\section{Introduction}

This is a new feature starting in this issue. It stems from a post
on the QL Forum from \emph{TMD2003} who was wondering about how to
get started learning Assembly Language as a ``noob''. The topic
is \href{https://qlforum.co.uk/viewtopic.php?f=3&t=3976}{this one}\footnote{https://qlforum.co.uk/viewtopic.php?f=3\&t=3976}
on the forum. A number of useful answers were given, some pointing
at my book and these eMagazines. 

\section{Do Basic Things}

On the second page of the thread, \emph{Tinyfpga} issued this ``sort
of'' challenge:

\emph{In Basic all I have to do start my journey into the programming
world is to:- Type into QD}

\begin{lstlisting}
OPEN #1,con_ 
OUTLN #1,310,60,50,300 
CLS #1 : BORDER #1,1,7 
INPUT #1,a$
\end{lstlisting}

\emph{I then save the text in a ram disk, press the execute button
and ,\textquotedbl hey presto\textquotedbl , I see an app on my
screen. What could be an easier introduction to programming?}

Well, I like a challenge, even when it's not really intended as one,
so I decided I would create the code to do the necessary.

The first thing to note is how simple it looks from SuperBASIC to
do what appears to be a simple thing. It's 4 lines of code, how hard
can that be?

Well, SuperBASIC takes the typed in commands, parses them and if all
is well, execution takes the tokenised code and converts it to various
calls to the ROM, Toolkits and such like, and the result is a seemingly
simply result. Under the covers there's a whole lot of work taking
place.

\subsection{Program Constants}

I created the code in the various listings in this chapter to do the
conversion from SuperBASIC to Assembly. I chose to create a job, that
multitasks alongside SuperBASIC but I could have made it a \texttt{CALL}able
routine instead. Let's dive in! Listing \ref{lis:Tinyfpga-Constant-Definitions}
shows the constants I used.

In the original, I had QDOS versions for the vectors and traps, for
this eMagazine, I've converted things to use SMSQ/E versions.

\begin{lstlisting}[caption={Tinyfpga - Constant Definitions},label={lis:Tinyfpga-Constant-Definitions}]
;==============================================================
; A simple multi-tasking job for TinyFPGA/TMD2003 to:
;
; Open #n,con_
; OUTLN #n,310,60,50,300
; CLS #n
; BORDER #n,1,7
; INPUT #n, some_text_from_user
; Die!
;==============================================================
; Norman Dunbar
; 20 November 2021.
;==============================================================

;--------------------------------------------------------------
; Some definitions to make life simple(r)!
;--------------------------------------------------------------
WHITE       equ 7               ; White colour
BLACK       equ 0               ; Black colour

BORDCOLOUR  equ WHITE           ; Border colour
BORDWIDTH   equ 1               ; Border width
PAPER       equ BLACK           ; Paper colour
INK         equ WHITE           ; Text colour

CON_W       equ 310             ; Console width
CON_H       equ 60              ;    "    height
CON_X       equ 50              ;    "    X position
CON_Y       equ 300             ;    "    Y position

BUFFERSIZE  equ 256             ; User input buffer size

;--------------------------------------------------------------
; I use GWASS as my assembler, it has the QDOS traps etc built
; in. It doesn't however, have the PE stuff, so these two are
; required.
;--------------------------------------------------------------
IOP_PINF    equ $70             ; Get PE information
IOP_OUTL    equ $7a             ; OUTLN

;--------------------------------------------------------------
; GWASS also doesn't know about SMSQ/E names, so these are now
; required.
;--------------------------------------------------------------
OPW.CON     equ $c6             ; Open a console, border, etc
IOW.CLRA    equ $20             ; CLS whole window
IOB.SBYT    equ $05             ; Print one byte to channel
IOW.ECUR    equ $0E             ; Enable cursor
IOB.FLIN    equ $02             ; Fetch a line of text plus LF
UT.WTEXT    equ $d0             ; Print some text
SMS.SSJB    equ $08             ; Suspend a job
SMS.FRJB    equ $05             ; Force remove a job

\end{lstlisting}

There's nothing much here apart from the new style, SMSQ/E names.
Most of these are trap codes but a couple are vectored routines which
can be used to call the trap routines but in a simpler manner. For
certain values of simpler, sometimes.

\subsection{Job Header}

Listing \ref{lis:Tinyfpga-Job-Header} shows the ``pretty much boilerplate''
code that all jobs need to have at the start. Seasoned readers can
skip this next explanation!

A job in SMSQ/E requires a standard job header. This is the same as
it was back in the old QDOS days, and consists of 10 bytes of boiler
plate code, followed by the job's name in the standard format of a
word defining the length of the name, followed by the bytes of the
name.

There are numerous ways to set up the first 6 bytes. I prefer a short
branch to the job's actual start followed by a long word of zero.
Back in the days when I wrote code for a living, this could have been
used to set up a serial number for copies of the program(s) - there's
enough room in a long word for $2^{32}$ different values. Minus 1
if you don't want a serial number of zero!

Regardless of how you set up the first 6 bytes, bytes 6 and 7 (starting
from zero) will always be the constant value of \$4AFB. This is the
marker word used by QDOS and SMSQ/E to indicate a job's code follows.

Immediately after the marker word, we have a word defining the size
of this job's name, here we see it is 8 bytes long, followed by the
bytes of the job name itself. In this case, I set the job name to
``TinyFPGA''.

\begin{lstlisting}[caption={Tinyfpga - Job Header},label={lis:Tinyfpga-Job-Header}]
;--------------------------------------------------------------
; Job's require a header. This is basically boilerplate, except
; for the job name's length and the name of the job. This will
; need to be EXEC/EXEC_W/EX or EW'd to execute it.
;--------------------------------------------------------------
Start
        bra.s open_console      ; Skip over job header
        dc.l 0                  ; 4 bytes, can be any value
        dc.w $4afb              ; Job flag, must be $4afb
        dc.w 8                  ; Length of job name
        dc.b "TinyFPGA"         ; Bytes of job name

\end{lstlisting}

Why does a job need a name? It doesn't! But if it has one, it makes
life easier when using the SuperBASIC \texttt{JOBS} command, or the
QPAC2 Jobs Thing to list the various jobs running in the system.

\subsection{Opening a Console Channel}

The first task in the challenge is to open a console channel. How
we do this is shown in Listing \ref{lis:Tinyfpga-Open-a-Console}.
Bear in mind that there are other Trap calls available to the programmer
to open files, and a console channel is just a file. Here we will
use a vectored utility which allows the ``open channel'' to be simplified
to one call. Without the vector we would have to open the channel,
set the paper, strip and ink colours, and set the border width and
colour too. I don;t know about you, but I prefer to do less typing
in my programs!

The 4 bytes at label \texttt{con\_def} define the border, paper and
ink attributes. Here I'm using some of the constants defined in Listing
\ref{lis:Tinyfpga-Constant-Definitions} which makes it easier if
I decide I don't like \emph{Tinyfpga}'s choice of colours, and want
to change things. 

The 4 words immediately following, at label \texttt{outln\_def}, are
used for two separate purposes. The first is when we open the console
channel -- they define the width, height, x position and y position
of the opened console channel.

The second use is when we try to OUTLN the channel. That function
requires a 4 word block of data in exactly this format, so we can
use the same code for two different things.

The fact that label exists between the two lumps\footnote{This is a technical term!}
of code makes no difference. A label doesn't generate any code or
data in the assembled program.

\begin{lstlisting}[caption={Tinyfpga - Open a Console},label={lis:Tinyfpga-Open-a-Console}]
;--------------------------------------------------------------
; Needs a channel open first. This can be done in a couple of
; ways, but this is probably the easiest. It will open a con_
; channel of the required size and border it.
;
; UT_CON uses all of the following from con_def and outln_def
; IOP_OUTLN only uses the latter.
;--------------------------------------------------------------
con_def
        dc.b BORDCOLOUR         ; Border colour
        dc.b BORDWIDTH          ; And width
        dc.b PAPER              ; Paper/strip colour
        dc.b INK                ; Ink colour

outln_def
        dc.w CON_W              ; Width
        dc.w CON_H              ; Height
        dc.w CON_X              ; X pos
        dc.w CON_Y              ; Y pos

open_console
        lea con_def,a1          ; Parameters
        move.w OPW.CON,a2       ; C6 = CONSOLE required
        jsr (a2)                ; Open Console & set params
        bne die                 ; If it failed - bale out

\end{lstlisting}

Having got the console channel definitions out of the way, we can
open it and set the attributes with a call to the vectored \texttt{OPW.CON}
utility. This call requires that the \texttt{A1.L} register points
at the byte defining the border colour, thus \texttt{con\_def}, and
that's all. 

To call a vectored utility we get a word from the ROM (as was in the
old days) at a certain address, in SMSQ/E this address is labelled
\texttt{OPW.CON} and this contains the address where the actual code
we wish to execute lives. In the ROM, there is a long list of available
vectored routines which we can use in our own code. 

After calling the vectored utility, we will have an error code in
\texttt{D0.L} and, if nothing went wrong, \texttt{D0.L} will be zero,
the \texttt{Z} flag will be set for us, and \texttt{A0.L} will contain
the channel identifier for the newly opened channel. Note that this
is not the same as a SuperBASIC channel number. SuperBASIC holds a
table of channel identifiers and indexes that table using the channel
number, not the actual channel identifier.

If there was an error, the error code is in \texttt{D0.L} as noted,
and the \texttt{Z} flag will not be set. In this case, we simply jump
to the code at label \texttt{die}, where the error code will be returned
to SuperBASIC and the job aborted.

\subsection{Do We Have the Pointer Environment?}

\emph{Tinyfpga's} challenge was to \texttt{OUTLN} the opened channel.
To do this we need to ensure that the Pointer Environment is present.
The code in Listing \ref{lis:Tinyfpga-Check-for-Pointer-Environment}
does exactly this by calling the \texttt{IOP.PINF} function. This
is not a vectored utility, this is a trap call. These are slightly
different as they execute as \emph{exceptions} and not as normal subroutuines.

To test if the PE is present we need a channel identifier in register
\texttt{A0.L} -- we already have that from above; we need a timeout
in register \texttt{D3.W} and we need the value \$70 in register \texttt{D0.L}\footnote{It need not be actually the \emph{whole} of register \texttt{D0},
however, a \texttt{moveq} instruction fills the whole of the register.}. After setting up the registers, a \texttt{trap \#3} call is made
and on return, an error code will be found in register \texttt{D0.L}
however, the \texttt{Z} flag will not be set.

\begin{lstlisting}[caption={Tinyfpga - Check for Pointer Environment},label={lis:Tinyfpga-Check-for-Pointer-Environment}]
;--------------------------------------------------------------
; We only get here if it worked. A0 now holds the channel id.
; Most, if not all, QDOSMSQ code preserves the A0 register.
;--------------------------------------------------------------
; Check if the PE is installed. If not, ignore the error and
; skip to handle clearing the screen "manually".
;--------------------------------------------------------------
check_pe
        moveq #IOP.PINF,d0      ; IOP_PINF
        moveq #-1,d3            ; Timeout (Preserved)
        trap #3
        tst.l d0                ; Errors?
        bne.w cls_console       ; PE missing

\end{lstlisting}

Why is the \texttt{Z} flag not set? Because the trap code executes
as an exception, part of what it does before executing is to stack
the status register. After execution, the exception code returns to
user code using the \texttt{RTE} instruction, not \texttt{RTS}. The
\texttt{RTE} unstacks the old status register value and puts it back
into the status register. This obliterates any flags set within the
exception (ie, \texttt{trap}) code so none of the flags will represent
what happened within the trap code.

This means that every time we return from a trap call, we must test
if \texttt{D0.L} is zero or not. This will set the \texttt{Z} flag
accordingly and we can then tell if the \texttt{trap} call worked
or failed.

In this case, we don't really care if the PE is present or not. Well,
I assumed this to be the case based on the challenge. If the PE is
found, we can \texttt{OUTLN} the channel as desired -- and this will
\texttt{CLS} the channel too based on one of the parameters. If not
found, we can just \texttt{CLS} the channel anyway.

\subsection{Pointer Environment Found}

If the PE is present, we can \texttt{OUTLN} the channel. Listing \ref{lis:Tinyfpga-Pointer-Environment-Present}
shows the code to do this. Again, \texttt{IOP.OUTL} is a trap call,
not a vector. We are required to set register \texttt{D0} to \$7A
-- which is what \texttt{IOP.OUTL} is defined as; Register \texttt{D1.L}
should hold the X and Y shadow widths for the \texttt{OUTLN} -- we
are not using shadows; Register \texttt{D2}\footnote{\texttt{The documentation doesn't mention a size!}}
should hold 1 to preserve the window contents so that a previously
\texttt{OUTLN}'d window can be moved and the contents preserved, or
zero to not bother. As this is the first \texttt{OUTLN} call for this
window, we have to use zero.

\begin{lstlisting}[caption={Tinyfpga - Pointer Environment Present},label={lis:Tinyfpga-Pointer-Environment-Present}]
;--------------------------------------------------------------
; PE is present.
;--------------------------------------------------------------
; OUTLN the window,D3 (timeout) was preserved in IOP_PINF as 
; was A0 (channel id for the console).
;--------------------------------------------------------------
outln_console
        moveq #IOP.OUTL,d0      ; IOP_OUTL
        moveq #0,d1             ; No shadows
        moveq #0,d2             ; Don't preserve contents
        lea outln_def,a1        ; Window sizes (W,H,X,Y)
        trap #3                 ; Do it
        tst.l d0                ; Errors?
        bne die                 ; Yes, bale out
        
\end{lstlisting}

After the \texttt{trap} call, we check for errors as explained above,
and if there were any, we exit the job via the code at label \texttt{die}.
If there were no errors, we drop in to the following code to clear
the screen. 

\subsection{Clear Screen}

If the PE was found to be missing, we don't really care in this small
example as all we need from the PE is the \texttt{OUTLN} call, and
we drop in here to clear the console channel. If the PE was present,
the channel has been \texttt{OUTLN}'d, but we are here again to clear
it also. Listing \ref{lis:Tinyfpga-Clear-the-Screen} sets \texttt{D0.L}
to \$20 also known as \texttt{IOW\_CLRA}; \texttt{D3.W} is still the
same timeout as before; \texttt{A0.L} is still the console channel
identifier. Those are all we need, so we then \texttt{trap \#3} to
clear the screen. On return, \texttt{D0.L} is tested in the usual
manner and on any errors, we bale out of the job.

\begin{lstlisting}[caption={Tinyfpga - Clear the Screen},label={lis:Tinyfpga-Clear-the-Screen}]
;--------------------------------------------------------------
; PE is missing.
;--------------------------------------------------------------
; Clear the screen. The timeout and channel ID have been 
; preserved over the last two routines.
;--------------------------------------------------------------
cls_console
        moveq #IOW.CLRA,d0      ; CLS whole window
        trap #3                 ; Do it
        tst.l d0                ; Any errors, probably not
        bne die                 ; Yes, bale out

\end{lstlisting}


\subsection{Print a Prompt}

This was not part of the original challenge, but I added it when debugging
a problem. I decided to leave it in. The code in Listing \ref{lis:Tinyfpga-Print-a-Prompt}
simply prints a prompt of `>' to the console channel. This is facilitated
using \texttt{IOB.SBYT} which requires that \texttt{D0.L} is set to
\$05, aka \texttt{IOB.SBYT} one of our constants; \texttt{D1.B} is
set to the character to be sent to the channel; \texttt{D3.W} is the
same old timeout value and \texttt{A0.L} is the channel identifier.

After the trap, \texttt{D0.L} is tested in the usual manner and we
exit the job if any errors occurred.

\begin{lstlisting}[caption={Tinyfpga - Print a Prompt},label={lis:Tinyfpga-Print-a-Prompt}]
;--------------------------------------------------------------
; Print a ">" prompt to the channel. Timeout/channel Id still
; preserved.
;--------------------------------------------------------------
con_prompt
        moveq #IOB.SBYT,d0      ; Print one character
        moveq #'>',d1           ; The prompt character
        trap #3                 ; Do it
        tst.l d0                ; Any errors?
        bne die                 ; Yes, bale out

\end{lstlisting}


\subsection{Enable the Cursor}

This is also not part of the original channel, but a cursor must be
enabled when we want to get input from a console channel. We have
yet another trap call to help with this. The code in \texttt{D0.L}
is \texttt{IOW.ECUR} the timeout and channel identifier are in the
usual registers. Errors are tested in the normal manner as this is
a \texttt{trap} call, not a vectored call.

Listing \ref{lis:Tinyfpga-Enable-a-Cursor} shows the code.

\begin{lstlisting}[caption={Tinyfpga - Enable a Cursor},label={lis:Tinyfpga-Enable-a-Cursor}]
;--------------------------------------------------------------
; Enable the channel's cursor. We need one to get input from
; a console channel. Timeout & channel id preserved still.
;--------------------------------------------------------------
con_cursor
        moveq #IOW.ECUR,d0      ; Enable cursor
        trap #3                 ; Do it
        tst.l d0                ; Errors?
        bne die                 ; Yes, bale out

\end{lstlisting}


\subsection{Get Some Input}

Listing \ref{lis:Tinyfpga-Request-Input} shows the code we need to
obtain a line of input from the user. How long is a line? Well, that's
all down to the programmer. However, we define an input buffer, at
label input\_buffer, to be 256 bytes long, plus an extra 2 bytes.
\texttt{BUFFERSIZE} is one of our constants and defaults to 256, but
you can change it.

Why 2 extra bytes? We are accepting a string from the user. Strings
in SMSQ/E are defined as a word holding the size followed by the bytes
of the string -- the job name in this code, for example. If we wanted
to process the string in some way, we would need to know the length.

\texttt{IOB.FLIN} is the trap call we need to use. This will have
\texttt{D0.L} holding \$02; \texttt{D2.W} holding the maximum buffer
size we will allow; \texttt{A1.L} points at the destination for the
input we receive and \texttt{D3.W} and \texttt{A0.L} are the usual
timeout and channel identifier.

You will note \texttt{A1.L} is pointing at the third byte in the input
buffer, this is where the text will be stored, the first two bytes
are used for the size of the text we obtained.

After the trap call, if there are no errors, we reset \texttt{A1.L}
to the start of the buffer this time -- it was changed by the trap
call -- and store the size word there. We now have a proper SMSQ/E
string. The size word in this case comes from \texttt{D1.W} on return
from the trap, where it holds the size of the input received, \emph{including}
the terminating linefeed.

What happens if there was more input than the buffer size? Nothing,
the buffer will be filled to capacity and the trap call will return.
The last character in the buffer will not, therefore, be a linefeed
in this case.

\begin{lstlisting}[caption={Tinyfpga - Request Input},label={lis:Tinyfpga-Request-Input}]
;--------------------------------------------------------------
; Grab some input from the console. The timeout and channel id
; are still valid. We point A1 at input_buffer+2 as we need the
; start of the buffer to hold the length of the text that
; follows.
;--------------------------------------------------------------
get_input
        moveq #IOB.FLIN,d0      ; Fetch input with Linefeed
        move.w #BUFFERSIZE,d2   ; How big is my buffer?
        lea input_buffer+2,a1   ; Input buffer space
        trap #3                 ; Fetch input D1.W = size
        tst.l d0                ; Errors?
        bne die                 ; Fraid so

        lea input_buffer,a1     ; The buffer start this time
        move.w d1,(a1)          ; Store the input size (inc LF)
        bra print_input         ; Skip over inoput buffer

input_buffer
        ds.w    BUFFERSIZE+2    ; Buffer for data input

\end{lstlisting}

Setting A1.L to the start of the buffer sets us up nicely for printing
out the received text.

\subsection{Printing the Text}

This wasn't part of the challenge, but I added it to show that the
date we typed in to the channel was in fact well received and correctly
saved. Listing \ref{lis:Tinyfpga-Printing-the-Input} shows the code
we use to print out the received input text. This will include the
trailing linefeed is one was present.

This code uses a vectored utility, \texttt{UT.WTEXT}, which internally
calls another trap \#3 function to print out the text. Why have I
used the vector? The vector requires a pointer to an SMSQ/E string
in \texttt{A1.L} -- we already have that. It requires a channel identifier
and timeout in \texttt{A0.L} and \texttt{D3.W} -- we already have
that.

The corresponding trap call needs the string length in \texttt{D1.W}
and \texttt{A1.L} pointing to the bytes of the text. I prefer using
\texttt{UT.WTEXT}.

After the call, we know that the \texttt{Z} flag is set if no errors
occurred, so there's no need to test \texttt{D0.L} on return. If errors
are detected, we again exit the job.

Note, this vectored code destroys the timeout value in \texttt{D3.W}.
However, at this point we are done with the infinite timeout we have
been using.

\begin{lstlisting}[caption={Tinyfpga - Printing the Input},label={lis:Tinyfpga-Printing-the-Input}]
;--------------------------------------------------------------
; Print the input that the user gave us, including the line
; feed at the end. A1 points to the text's word size, D3 will
; be corrupted by this vector call (timeout) but the channel
; id in A0 will not.
;--------------------------------------------------------------
print_input
        move.w UT.WTEXT,a2      ; Print a string of bytes
        jsr (a2)                ; Print it
        bne.s die               ; Ooops, error

\end{lstlisting}


\subsection{Hang on a Few Seconds!}

Ok, we are done. Except to give the user a chance to see the text
printed on the channel, I've added yet another extra to the challenge
code. The currently running job, named ``Tinyfpga'', will be suspended
for a couple of seconds. Listing \ref{lis:Tinyfpga-Delay-Before-Ending}
shows how this is done.

\texttt{SMS.SSJB} is a trap call which suspends a job from execution
for the number of frames specified in the \texttt{D3.W} register.
\texttt{D1.L} holds the job identifier, or -1 for the current job,
and \texttt{A1.L} points at a byte which will be cleared when this
job resumes. As we have no need to signal our reappearance, we use
zero.

Note that the number of frames is 200. This is $4$ seconds in the
UK and countries with a 50 Hz mains frequency. In the USA it's 60
Hz, so in the USA the delay will be $3\frac{1}{3}$ seconds.

After the trap, we do not test for errors, we are about to die anyway.

\begin{lstlisting}[caption={Tinyfpga - Delay Before Ending},label={lis:Tinyfpga-Delay-Before-Ending}]
;--------------------------------------------------------------
; Suspend the job for a couple of second to let the user see
; the output. Then die. Will corrupt A0 but who cares!
;--------------------------------------------------------------
suspend_job
        moveq #SMS.SSJB,d0      ; Suspend a job
        moveq #-1,d1            ; This job
        move.w #200,d3          ; 4 seconds is 200 frames
        movea.l #0,a1           ; No byte to be cleared
        trap #1                 ; Suspend the job

\end{lstlisting}


\subsection{Death of a Job}

The job is now complete. We are not required to loop around and keep
running, so we cannot allow the job's code to simply stop, we need
to remove the job from the system. Listing \ref{lis:Tinyfpga-Death-of-a-job}
shows how to force remove a job using the \texttt{SMS.FRJB} trap call.

The error code in D0.L is copied into D3.L for return to SuperBASIC
-- more on that soon -- and the job identifier is loaded into D1.L,
we are using -1 again to indicate the current job. After the trap,
no code will be executed as the job \emph{is no more, it has shuffled
off its mortal coil and gone to meet its maker!}\footnote{From Monty Python's \emph{Dead Parrot} sketch.}

\begin{lstlisting}[caption={Tinyfpga - Death of a Job},label={lis:Tinyfpga-Death-of-a-job}]
;--------------------------------------------------------------
; The job is complete, remove it from the system. Any error
; codes in D0 are copied to D3 ready for EXEC_W/EW to collect.
; EXEC/EX don't bother.
;--------------------------------------------------------------
die
        move.l d0,d3            ; Any errors?
        moveq #SMS.FRJB,d0      ; Force Remove a job
        moveq #-1,d1            ; -1 means "this job"
        trap #1                 ; Kill this job

\end{lstlisting}


\subsection{Error Codes}

When the job is started using \texttt{EXEC}, or \texttt{EW}, it will
never tell you how it ended. There could have been errors at any stage
and you will never know about it. Why not? Because \texttt{EXEC}/\texttt{EW}
start up a job and then return. These commands do not wait for the
job to complete. How then can they be expected to be able to obtain
the job's error code when it finishes -- it might run for days after
all.

During testing, when the code wasn't crashing, I ran it with \texttt{EXEC\_W}
or \texttt{EW}. These commands wait for the job to complete before
returning to SuperBASIC. In this case, the job's error codes can be
returned to SuperBASIC.

\section{Assembling the Code}

If I were to assume that you have downloaded \href{http://www.dilwyn.me.uk/asm/gwassp22.zip}{GWASS}\footnote{http://www.dilwyn.me.uk/asm/gwassp22.zip}
for QPC2 and other 68020 based emulators, or \href{http://www.dilwyn.me.uk/asm/gwaslp08.zip}{GWASL}\footnote{http://www.dilwyn.me.uk/asm/gwaslp08.zip}
for the QL and 68008 based emulators, and have the code saved as \texttt{ram1\_Tinyfpga\_asm},
then assembling the code is as simple as this:
\begin{itemize}
\item \texttt{EXEC} \texttt{gwass60\_bin} or \texttt{EXEC gwasl\_bin} to
start the assembler;
\item Select the option to start assembling;
\item Type in the filename: \texttt{ram1\_Tinyfpga\_asm}
\item Wait.
\end{itemize}
After a successful assemble, \texttt{ram1\_Tinyfpga\_bin} will be
the executable job. To run it:
\begin{itemize}
\item \texttt{EXEC} \texttt{ram1\_Tinyfpga\_bin: REMark Alternatively, EX
ram1\_Tinyfpga\_bin}
\end{itemize}
On a successful execution, a small window will open, with black paper,
white ink and a white, one pixel border. A `>' prompt will be displayed
in the top left corner. Type some text and press ENTER. The text you
typed will be printed, the job will pause for 4 seconds, and then
vanish.

\section{Summary}

So that's the Assembly Language version of \emph{Tinyfpga}'s challenge.
There's a lot going on under the covers of SuperBASIC that programmers
almost never see. When you start delving ito Assembly Language, you
are responsible for just about everything! Thankfully, SMSQ/E provides
numerous utilities and features that you can call upon to make life
easier. 

In future issues, I'll be delving into a few more of these with, hopefully,
enough explanation for beginners to get started with.

Get hold of the SMSQ/E Reference Manual from:
\begin{itemize}
\item \href{http://www.dilwyn.me.uk/docs/manuals/QDOS_SMS\%20Reference\%20Guide\%20v4.5.pdf}{Here}\footnote{http://www.dilwyn.me.uk/docs/manuals/QDOS\_SMS\%20Reference\%20Guide\%20v4.5.pdf}
for the PDF version; or
\item \href{http://www.dilwyn.me.uk/docs/manuals/QDOS_SMS\%20Reference\%20Guide\%20v4.5.odt}{Here}\footnote{http://www.dilwyn.me.uk/docs/manuals/QDOS\_SMS\%20Reference\%20Guide\%20v4.5.odt}
for the ODT version.
\end{itemize}

