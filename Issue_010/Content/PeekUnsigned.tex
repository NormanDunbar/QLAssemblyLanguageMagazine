
\chapter{Unsigned Peeks}

There's a thread on the \href{https://qlforum.co.uk/viewtopic.php?f=3&t=4027&sid=7a218630d06b1d78c6bb436e1fbf32f9}{QL Forum}\footnote{https://qlforum.co.uk/viewtopic.php?f=3\&t=4027\&sid=7a218630d06b1d78c6bb436e1fbf32f9}
about how difficult it might be to write programs in S{*}BASIC on
SMSQ/E. There is a pile of useful information there. One of the more
interesting comments was about a hardware timer to get proper delays
in code, but this required the use of \texttt{PEEK\_L}. And, as we
all know, \texttt{PEEK\_L} returns a float, and is signed. 

I responded on \href{https://qlforum.co.uk/viewtopic.php?f=3&t=4027&p=45290\#p45290}{this post}\footnote{https://qlforum.co.uk/viewtopic.php?f=3\&t=4027\&p=45290\#p45290}
with an example of some code to get around that problem by adding
2\textsuperscript{31} if the result was negative. Sadly, in my posting,
I actually typed 2\textsuperscript{32}-- Duh! 

Marcel responded that \emph{someone could do a }\texttt{\emph{PEEK\_UL}}\emph{
function} to get an unsigned result. This caused confusion for a bit
as nobody knew that \texttt{PEEK\_UL} existed and when they tried,
it was, of course, not found. Marcel \emph{may} have been hinting
that \emph{someone could easily write an unsigned version of the }\texttt{\emph{PEEK\_L}}\emph{
(and }\texttt{\emph{PEEK\_W}}\emph{) functions}. So I did! And here
they is.

You'd \emph{think} there was nothing to it really, and there isn't,
but along the way to producing this code, I had to delve into some
old code of mine\footnote{Well, it's either mine or I stole it from somewhere, I can't remember
that far back!} to convert a long value to a floating point value on the maths stack.
For the life of me, I couldn't remember how I did it, so I fell down
that rabbit hole until I had a vague understanding of QL Floating
Point and the conversion code.

SMSQ/E has a maths package operation to convert a long value on the
maths stack into a float, however, QDOS doesn't have this option --
it's interestingly missing from the maths package op code -- so there
are two versions of the source code and binary files on the accompanying
code download for this issue.

The SMSQ/E version is 230 bytes while the QDOS version is longer,
at 258. The QDOS version is, however, safe to run on SMSQ/E as well.
Just don't try running the SMSQ/E version on QDOS -- I have no idea
what will happen.

\section{PEEK\_UW and PEEK\_UL}

\texttt{PEEK\_W} and \texttt{PEEK\_L} on the QL are signed. They return
positive and negative values. This is sometimes a bit of a pain because,
taking \texttt{PEEK\_L} as an example, fetching any value over \$7FFFFFFF
or 2,147,483,647, will result in the value going negative. This may
not be what you desire. The solution is easy, add 2\textsuperscript{15}
or 2\textsuperscript{31}, depending on whether you are peeking for
words or longs, to the result on return to S{*}BASIC. Sometimes you
forget, well, I do, and so, these two new functions will do the testing
and adding for us, leaving S{*}BASIC to get on with interpreting the
rest of the code.

Listing \ref{lis:Unsigned-Peeks-Equates} is the start of the source
code, and shows the various equates I've used in the code. These are,
as mentioned in the News Section, now using the SMSQ/E mnemonics,
don't worry if you are using QDOS as it's the values that count --
I could have called them Fred, Barney Wilma, Betty etc, but that's
not a meaningful set of names\footnote{Unless we are writing a Flintstones application I suppose.}. 

\lstinputlisting[linerange={10-26},caption={Unsigned Peeks - Equates},label={lis:Unsigned-Peeks-Equates}]{/home/norman/SourceCode/Assembly eMagazine/Issue_010/Code/PeekUnsigned/Peek_ul_SMSQ.asm}

The only thing I'd draw your attention to is the two flags I've set
up to determine if we are peeking words or longs, those being \texttt{peek\_w}
and \texttt{peek\_l}. I've given those a value which just happens
to correspond to the exponent for a floating point value of either
2\textsuperscript{15} or 2\textsuperscript{31} -- this is a cunning
plan to save me some work later on!

The code proper begins at the label start, which you can see in Listing
\ref{lis:Unsigned-Peeks-Extend-SBASIC} where we have the standard
code to link in new procedures and/or functions, and the requisite
definition block in Listing \ref{lis:Unsigned-Peeks-Procedure-function-definition-block}
where we are defining no new procedures, only two new functions, \texttt{PEEK\_UL}
and \texttt{PEEK\_UW}.

\lstinputlisting[linerange={31-35},caption={Unsigned Peeks - Linking into S*BASIC},label={lis:Unsigned-Peeks-Extend-SBASIC}]{/home/norman/SourceCode/Assembly eMagazine/Issue_010/Code/PeekUnsigned/Peek_ul_SMSQ.asm}

\lstinputlisting[linerange={40-50},caption={Unsigned Peeks - Procedure/Function definition block},label={lis:Unsigned-Peeks-Procedure-function-definition-block}]{/home/norman/SourceCode/Assembly eMagazine/Issue_010/Code/PeekUnsigned/Peek_ul_SMSQ.asm}

Both new functions share a lot of common code. To this end, I set
a flag to determine which is being called so that the places where
things differ, can easily tell the functions apart. \texttt{PEEK\_W}
begins, as Listing \ref{lis:Unsigned-Peeks-Peek-uw} shows, by setting
the correct flag in \texttt{D4.W}, before jumping off to join the
first section of common code.

\lstinputlisting[linerange={59-61},caption={Unsigned Peeks - Peek\_uw},label={lis:Unsigned-Peeks-Peek-uw}]{/home/norman/SourceCode/Assembly eMagazine/Issue_010/Code/PeekUnsigned/Peek_ul_SMSQ.asm}

Listing \ref{lis:Unsigned-Peeks-Peek_ul} is the beginning of the
\texttt{PEEK\_UL} function. It also sets a flag before dropping into
the common code for both functions. 

\lstinputlisting[linerange={69-80},caption={Unsigned Peeks - Peek\_ul},label={lis:Unsigned-Peeks-Peek_ul}]{/home/norman/SourceCode/Assembly eMagazine/Issue_010/Code/PeekUnsigned/Peek_ul_SMSQ.asm}

Both functions require a single parameter, so we subtract \texttt{A3.L}
from \texttt{A5.L} and if the result is not 8 bytes, then we exit
back to S{*}BASIC with a bad parameter error. \texttt{A3.L} points
to the last function parameter on the name table, and \texttt{A5.L}
to the first. Each name table entry takes 8 bytes.

If we do have 8 bytes of a difference, then we know that there's exactly
one parameter waiting, so we can go fetch it. Listing \ref{lis:Unsigned-Peeks-Fetching-one-parameter}
shows the code to fetch the address parameter as a long value.

\lstinputlisting[linerange={86-90},caption={Unsigned Peeks - Fetching one parameter},label={lis:Unsigned-Peeks-Fetching-one-parameter}]{/home/norman/SourceCode/Assembly eMagazine/Issue_010/Code/PeekUnsigned/Peek_ul_SMSQ.asm}

When fetching parameters for a procedure or function, \texttt{D3.W}
holds the count of parameters actually fetched. Just as an extra check,
we test this, in Listing \ref{lis:Unsigned-Peeks-Testing-parameters-fetched},
to make sure that we did indeed only fetch a single parameter.

\lstinputlisting[linerange={96-98},showlines=true,caption={Unsigned Peeks - Testing parameters fetched},label={lis:Unsigned-Peeks-Testing-parameters-fetched}]{/home/norman/SourceCode/Assembly eMagazine/Issue_010/Code/PeekUnsigned/Peek_ul_SMSQ.asm}

If we somehow managed to fetch more, or less, than one parameter,
we bale out to S{*}BASIC with a bad parameter error. If not, then
we continue to pull the passed address off of the maths stack into
\texttt{A2.L} and then we `peek' the long value at the requested
address into \texttt{D7.L} as per Listing \ref{lis:Unsigned-Peeks-Getting-the-address-and-peeking}

\lstinputlisting[linerange={104-106},caption={Unsigned Peeks - Getting the address, and peeking it},label={lis:Unsigned-Peeks-Getting-the-address-and-peeking}]{/home/norman/SourceCode/Assembly eMagazine/Issue_010/Code/PeekUnsigned/Peek_ul_SMSQ.asm}

At t his point, we have pretty much executed a \texttt{PEEK\_L} function
call, but we might want only the first word if we are executing \texttt{PEEK\_UW}.
If so, we need to return only the current high word of \texttt{D7.L},
so Listing ? is the code that does the check. If we are indeed running
\texttt{PEEK\_UW}, then the low word is cleared and then the upper
word is swapped into the lower word, giving the correct value in \texttt{D7.L}
for \texttt{PEEK\_UW}.

\lstinputlisting[linerange={112-116, 121-121},caption={Unsigned Peeks - Fixup for PEEK\_UW},label={lis:Unsigned-Peeks-Fixup-for-peek-uw}]{/home/norman/SourceCode/Assembly eMagazine/Issue_010/Code/PeekUnsigned/Peek_ul_SMSQ.asm}

Both functions come together again at label \texttt{pulGotPeek}, where
\texttt{D7.L} holds the possibly negative value we need to return
to S{*}BASIC. Unfortunately, this is also exactly where they diverge
again, however, I'm only discussing the SMSQ/E version here. I've
explained converting a long word into floating point the end of this
chapter, if you are interested?

The differences are these:
\begin{itemize}
\item SMSQ/E has a maths package operation, \texttt{OP.FLONG}, to take the
long word on the top of the maths stack, and convert it to a floating
point value at TOS, adjusting \texttt{A1.L} as necessary (subtracting
2 from it -- the difference between a 6 byte float and a 4 byte long)
\item QDOS requires the developer, me, to allocate an additional 2 bytes
on the maths stack. There is already 4 bytes available as we pulled
the address parameter as a long word, which takes 4 bytes.
\item It the space allocation succeeds then we can convert, manually, and
normalise the long word in \texttt{D7.L} into a float and stack it.
\item In the QDOS variant, the developer, me again, has to be very careful
to keep the TOS pointer in A1 correct at all times. 
\end{itemize}
Listing \ref{lis:Unsigned-Peeks-Floating-a-long-smsq-style} is the
simple way that SMSQ/E converts a long word to a floating point value.

\lstinputlisting[linerange={185-190},caption={Unsigned Peeks - Floating a Long SMSQ/E style},label={lis:Unsigned-Peeks-Floating-a-long-smsq-style}]{/home/norman/SourceCode/Assembly eMagazine/Issue_010/Code/PeekUnsigned/Peek_ul_SMSQ.asm}

We need to stack the value in \texttt{D7.L} so we can do that easily
as we know that there are 4 bytes available on the maths stack, and
that \texttt{A1.L} is still correctly pointing at the TOS where we
need our long word to be. As with almost everything S{*}BASIC, addresses
are relative to \texttt{A6}. 

After stacking \texttt{D7.L}, we simply call a subroutine, \texttt{pulDoMathsOp},
which can be seen later in Listing \ref{lis:Unsigned-Peeks-Execute-maths-package-operation},
to do the hard work of converting the value. If that worked, then
we skip off to \texttt{pulTestNegative}, where we join up with the
code for QDOS again.

\lstinputlisting[linerange={201-203},caption={Unsigned Peeks - Testing for negativity},label={lis:Unsigned-Peeks-Testing-for-negativity}]{/home/norman/SourceCode/Assembly eMagazine/Issue_010/Code/PeekUnsigned/Peek_ul_SMSQ.asm}

Listing \ref{lis:Unsigned-Peeks-Testing-for-negativity} checks the
byte at the high end of the mantissa for the floating point value
on the maths stack. Bit 31 of the mantissa is the sign bit and that
corresponds to bit 7 in this particular byte. If the bit is clear,
then we have a positive number as there's no need to do the addition
to make it positive.

If the bit is set, then the value is negative, and we need to add
2\textsuperscript{15} or 2\textsuperscript{31} depending on the
function currently executing. Listing \ref{lis:Unsigned-Peeks-Adjusting-for-negativity}
shows how I decided to do it.

\lstinputlisting[linerange={215-223},caption={Unsigned Peeks - Adjusting for negativity},label={lis:Unsigned-Peeks-Adjusting-for-negativity}]{/home/norman/SourceCode/Assembly eMagazine/Issue_010/Code/PeekUnsigned/Peek_ul_SMSQ.asm}

The first thing we have to do is request 6 bytes, enough room for
a new floating point value, on the maths stack. This is done by a
subroutine, \texttt{pulGetSpace} in Listing \ref{lis:Unsigned-Peeks-Allocating-maths-stack-space}
which will make sure that the space is allocated and adjusts \texttt{A1.L}
as necessary.

We can now stack the floating point representation of 2\textsuperscript{15}
or 2\textsuperscript{31} appropriately. Remember that flag I used
to determine whether we were in \texttt{PEEK\_UW} or \texttt{PEEK\_UL}?
Well, that value just happens to be the exponent for 2\textsuperscript{15}
or 2\textsuperscript{31}and so we can stack the exponent directly
from the flag register, \texttt{D4.W}. The mantissa for both values
is the same, so that gets stacked next. We now have \texttt{(A6,A1.L)}
pointing at 2\textsuperscript{15} or 2\textsuperscript{31} on TOS,
and the peek value at Next On Stack, NOS. We need to add them together.

SMSQ/E and QDOS both have the same maths package operation to add
two floats, \texttt{QA.ADD}. This has the side effect of replacing
NOS with the result of the addition and removes TOS making the existing
NOS the new TOS. \texttt{A1.L} is now 6 larger than previously, and
we need this to be saved away. The subroutine \texttt{pulDoMathsOp},
in Listing \ref{lis:Unsigned-Peeks-Execute-maths-package-operation}
does this for us.

The value at TOS is now guaranteed to be positive, so we can carry
on and return the result to S{*}BASIC. Listing \ref{lis:Unsigned-Peeks-Return-to-s-basic}
is how we do that.

\lstinputlisting[linerange={230-233},caption={Unsigned Peeks - Return to S*BASIC},label={lis:Unsigned-Peeks-Return-to-s-basic}]{/home/norman/SourceCode/Assembly eMagazine/Issue_010/Code/PeekUnsigned/Peek_ul_SMSQ.asm}

\texttt{D4.W} is set to 2 to indicate a float result is on the maths
stack at TOS and we return with no errors.

That's basically all there is to it! Except for a couple of small
subroutines. Listing \ref{lis:Unsigned-Peeks-Allocating-maths-stack-space}
is the subroutine \texttt{pulGetSpace} which requests space for the
maths stack and keeps the various pointers correctly aligned.

\lstinputlisting[linerange={247-256},caption={Unsigned Peeks - Allocating maths stack space},label={lis:Unsigned-Peeks-Allocating-maths-stack-space}]{/home/norman/SourceCode/Assembly eMagazine/Issue_010/Code/PeekUnsigned/Peek_ul_SMSQ.asm}

The code expects \texttt{D1.L} to contain the number of bytes needed
and \texttt{A1.L} to point at the maths stack TOS, relative to \texttt{A6.L}.
The \texttt{QA.RESRI} vector code trashes registers \texttt{D2}, \texttt{D3},
\texttt{A1} and \texttt{A2}, but we have to reload \texttt{A1.L} from
\texttt{SB\_ARTHP} anyway, so that's no matter. 

The subroutine requests \texttt{D1.L} bytes of space, then retrieves
the potentially new value for \texttt{A1.L} from \texttt{SB\_ARTHP}
before subtracting \texttt{D1.L} to get the new TOS. This value is
saved back in \texttt{SB\_ARTHP} as it must be.

On exit, \texttt{A1.L} is the new TOS pointer, ready for use by our
code.

Listing \ref{lis:Unsigned-Peeks-Execute-maths-package-operation}
is the final subroutine. This one, \texttt{pulDoMathsOp}, executes
a single maths package operation according to the code in \texttt{D0.W}.

\lstinputlisting[linerange={272-281},caption={Unsigned Peeks - Execute maths package operation},label={lis:Unsigned-Peeks-Execute-maths-package-operation}]{/home/norman/SourceCode/Assembly eMagazine/Issue_010/Code/PeekUnsigned/Peek_ul_SMSQ.asm}

The subroutine expects an operation code, as a word, in \texttt{D0.W}
and executes it. If all went well, then the TOS might have changed,
so to keep things right, the potential new TOS is pulled from \texttt{SB\_ARTHP}
back into \texttt{A1.L} before setting no errors in D0, and returning
to the caller.

If errors were detected, the code makes an early exit back to the
caller with \texttt{D0} set to the error code, and the \texttt{Z}
flag unset.

\subsection{QDOS Conversion of Long to Float}

AS mentioned above, QDOS requires manual intervention to convert a
long word into a floating point value. We already have room for a
long word, but we need an additional two bytes for a float, so first
of all, we have to request two extra bytes. Listing ? shows how we
do this by setting \texttt{D1.L} to 2, and calling the \texttt{pulGetSpace}
subroutine, in Listing ?, to do the hard work of getting the space
and keeping the pointers aligned. On return, \texttt{A1.L} is set
ready for us to use to stack a 6 byte float value. 

\lstinputlisting[linerange={134-136},caption={Unsigned Peeks - QDOS Making stack space},label={lis:Unsigned-Peeks-QDOS-making-stack-space}]{/home/norman/SourceCode/Assembly eMagazine/Issue_010/Code/PeekUnsigned/Peek_ul_QDOS.asm}

Unfortunately, QDOS is bereft in the maths package operations to convert
a long word to a float, so we have to do it manually. Listing \ref{lis:Unsigned-Peeks-QDOS-Convering-long-to-float}
does the hard work for us. At the end, we should have the exponent
value in register \texttt{D5.W} and the mantissa in register \texttt{D6.L}. 

\lstinputlisting[linerange={144-168},caption={Unsigned Peeks - QDOS Converting Long to Float},label={lis:Unsigned-Peeks-QDOS-Convering-long-to-float}]{/home/norman/SourceCode/Assembly eMagazine/Issue_010/Code/PeekUnsigned/Peek_ul_QDOS.asm}

How does it work? I was hoping you wouldn't ask!

\subsection{QL Floating Point Values}

The QL's internal floating point format is a 2 byte exponent and a
4 byte mantissa. What are they? Well, if you have a number such as
$1.23\times10^{3}$ then the mantissa is the $1.23$ and the exponent
is $3$ and the number is \textquotedbl normalised\textquotedbl{} when
the mantissa is \textquotedbl n.something\textquotedbl . Zero is a
special case.

$1.23\times10^{3}$ could easily be written as $12.3\times10^{2}$
or $123\times10^{1}$ they all result in 1,230. A properly normalised
number, in this notation, is a single digit prior to the decimal point,
so $1.23\times10^{3}$ is the correct value.

The exponent is the power of 10 that the mantissa is to be multiplied
by, to get the actual\footnote{Watch out for errors though, floating point is notorious for errors.
Many floating point values cannot be represented exactly. One third
for example. } value, in other words, the value is:

\[
mantissa*10^{exponent}
\]

The same is true in binary but there the value is $mantissa*2^{exponent}$
as we are working in base 2, not base 10. 

In SMSQ/E, the mantissa is normalised when bit 31 or bit 30 of the
mantissa is set. Bit 31 will be set if the number is negative, and
bit 30 if positive. The exponent only has the bottom 12 bits available
as the top nibble is used to determine if this is a decimal float
value, a binary float value or a hexadecimal float value.

12 bits is 2,048. \$800 in hexadecimal, so the exponent has 12 bits
of precision. However, it is offset by \$800, so the range is from
2\textsuperscript{-2,048} to 2\textsuperscript{2,047} which is from
3.094\textsuperscript{-617} to 1.616\textsuperscript{616} which
is a decent enough range. Because the exponent is offset, this makes
the floating point value equal to:

\[
mantissa*2^{exponent-\$800}
\]

Zero is a special case where the exponent and mantissa are both zero.

The bits in the mantissa represent fractions, but in power's of 2.
They can be thought of as $\frac{1}{2^{31-bit}}$ but , as with many
things mathematical, this can also be expressed as $2^{-(31-bit)}$.
The fractional parts of the mantissa start with bit 30, the most significant
bit, which represents $\frac{1}{2^{1}}$, $2^{-1}$ or 0.5, down to
bit 0, representing $\frac{1}{2^{31}}$, $2^{-31}$ or \emph{not very
much at all}!\footnote{Ok, ok, it's $\frac{1}{2,147,483,648}$, which is pretty small.}

A couple of examples:
\begin{itemize}
\item 10.0 is represented as \$0804 50000000
\item -10.0 is represented by \$07FC D0000000
\end{itemize}
Taking 10 first.

The mantissa is \%0101 0000 0000 0000 0000 0000 0000 0000 so we are
only interested in the \%0101 bits, the trailing zeros can be ignored.

The sign bit is zero, so the number is positive. So far so good. Bit
30, which represents 0.5 is set, as is bit 28 which represents 0.125,
the mantissa is therefore the addition of these fractions. The result
is 0.625.

The exponent is \$0804, subtracting \$0800 we get \$04, so 10 is 0.625
multiplied by 2\textsuperscript{4 }or 16. 10 is indeed equal to 0.625
times 16.

Negative 10 next.

The mantissa is \%1101 0000 0000 0000 0000 0000 0000 0000 and again,
we can ignore all the trailing zero, keeping only \%1101.

Bit 31 is 1 so we are dealing with a negative number.

Bits 30, 0.5, and 28, 0.125, are both set, so we have a value again,
of 0.625. Subtracting \$800 from \$7FC results in \$FFFC or -4, and
0.625 times 2\textsuperscript{-4} is -10. 

If we assume \texttt{D7.L} is zero, then we have the easy case where
we move it into \texttt{D5.W} and \texttt{D6.L}, it will set the \texttt{Z}
flag and we are ready to stack the value.

If we take \texttt{D7} as having the value \$1E240 which is 123,456
in decimal. We start off by setting up the various registers, \texttt{D5}
gets a copy of the low word, \texttt{D6} the whole \$1E240. Because
we are not dealing with zero, we drop down to set \texttt{D5} to the
first guess exponent, \$081F, and adding \texttt{D7} to itself which
results in \texttt{D7} = \$3C480. This didn't overflow, so we still
have a valid mantissa which is twice as big as it was.

As we have now doubled the mantissa, we must also halve the exponent,
which means subtracting 1 from \texttt{D5} to get \$081E.

The new mantissa is copied into \texttt{D6} where we will keep it
for those occasions when shifting it by \texttt{D0} bits results in
overflow. \texttt{D0} is initialised to \$10 -- we will attempt a
16 bit shift first of all.

We \emph{could} shift by a single bit each time, and halve the exponent
each time there's no overflow, but this method is quicker. If \texttt{D7}
\emph{can} be shifted 16 bits left, then that that's one shift, not
16\footnote{I have a funny feeling this code is something from Simon N. Goodwin!}.

We are now at label \texttt{pulNormalise} for the first time, and
here we enter the normalisation loop. We will multiply \texttt{D7}
by \textquotedbl lots of bits\textquotedbl{} each time as long as we
don't get overflow. If we do hit overflow, we have gone too far and
need to back up a bit, and multiply by half as many \textquotedbl lots
of bits\textquotedbl .

We copy the most recent valid mantissa from \texttt{D6} to \texttt{D7}
ready to try again, and shift it left by the \$10 bits held in \texttt{D0}.
This caused overflow as \texttt{D7} was holding \$3C480, which is
\%0000 0000 0000 0011 in the high word, so there were only 14 bits
we could have shifted. As \texttt{V} was set, we skip to the label
\texttt{pulTooBig} where we adjust \texttt{D0} from \$10 to \$08 and
skip back to \texttt{pulNormalise} again.

On the second pass through the loop, we copy the last valid mantissa
back into \texttt{D7}, and attempt to shift by 8 bits. This results
in \texttt{D7} now holding the value \$3C48000 but this time, there
was no overflow. As this was a successful multiplication of the mantissa,
we subtract the correct powers of two from the exponent, which happens
to be \texttt{D0}'s value, 8. Finally, we copy the new valid mantissa
into \texttt{D6}, just in case, and drop into \texttt{pulTooBig} to
divide \texttt{D0} down to 4, ready for another attempt.

On the third pass, we again copy the most recent valid mantissa back
to \texttt{D7}, and multiply by 4 bits. This gives a result of \$3C480000
in \texttt{D7} and no still overflow, so we have a new valid mantissa.
The exponent is adjusted by the correct powers of two and the new
mantissa is copied to a safe place in \texttt{D6} before \texttt{D0}
is again divided down to 2 at label \texttt{pulTooBig}.

The fourth pass through the loop multiplies \texttt{D7} up to \$F1200000
but causes overflow, we have to skip to \texttt{pulTooBig} to drop
\texttt{D0} down to 1 for another attempt.

The fifth pass through the loop shifts \texttt{D7} to get a result
for the mantissa of \$78900000 with no overflow. As before, the exponent
is divided to cater for the multiplication, and the new mantissa is
copied to \texttt{D6} for safety. At label \texttt{pulTooBig}, \texttt{D0}
is divided down but is now zero. We are now holding a normalised mantissa
in \texttt{D7}. The loop is finished and we drop to the label \texttt{pulNormalised}.

Remember when we calculated that only had 14 bits to shift to normalise
\texttt{D7}, and shifting by 16 caused overflow? Add up all the successful
shifts, $8+4+1$, then add one more for the initial doubling of \texttt{D7},
and you get 14.

Table \ref{tab:Floating-point-conversion-of-123456} shows each pass
and the various values as a summary.

\begin{table}
\begin{centering}
\begin{tabular}{|c|c|c|c|c|c|c|}
\hline 
\textbf{Pass} & \textbf{D7 Input} & \textbf{D5 Input} & \textbf{Shifts} & \textbf{Overflow} & \textbf{D7 Output} & \textbf{D5 Output}\tabularnewline
\hline 
\hline 
Start & \$0001 E240 & \$81F & - & - & - & -\tabularnewline
\hline 
Init & \$0001 E240 & \$81F & 1\tablefootnote{It was an addition, but that's equivalent to a shift.} & No & \$0003 C480 & \$81E\tabularnewline
\hline 
1 & \$0003 C480 & \$81E & 16 & Yes & - & \$81E\tabularnewline
\hline 
2 & \$0003 C480 & \$81E & 8 & No & \$03C4 8000 & \$816\tabularnewline
\hline 
3 & \$03C4 8000 & \$816 & 4 & No & \$3C48 0000 & \$812\tabularnewline
\hline 
4 & \$3C48 0000 & \$812 & 2 & Yes & - & \$812\tabularnewline
\hline 
5 & \$3C48 0000 & \$812 & 1 & No & \$7890 0000 & \$811\tabularnewline
\hline 
End & \$7890 0000 & \$811 & 0 & - & - & -\tabularnewline
\hline 
\end{tabular}
\par\end{centering}
\caption{Floating point conversion of 123456\label{tab:Floating-point-conversion-of-123456}}

\end{table}

At this point, the exponent is \$0811 and the mantissa is \$78900000.
Is this correct? We can convert the exponent to the correct powers
of two by subtracting \$800 to get \$11, or 17 in decimal.

The mantissa, \$78900000, is \%0111 1000 1001 0000 0000 0000 0000
0000 but as we can safely ignore the trailing zeros, we have \%0111
1000 1001 in the highest three nibbles. (Bits 31 - 20.)

The sign bit, bit 31, is a zero, so the mantissa is positive.

Bits 30, 29, 28, 27, 23 and 20 are set, but we need to subtract those
from 31 to get bits 1,2,3,4,8 and 11 for out fractional powers of
two. The mantissa is a fraction, in the format of 0.something, remember?

The bit values represent the powers of two that we take the reciprocal
of and add them up, to get the actual fractional value for the mantissa.
This sum is $\frac{1}{2^{1}}+\frac{1}{2^{2}}+\frac{1}{2^{3}}+\frac{1}{2^{4}}+\frac{1}{2^{8}}+\frac{1}{2^{11}}$
which works out as $\frac{1}{2}+\frac{1}{4}+\frac{1}{8}+\frac{1}{16}+\frac{1}{256}+\frac{1}{2048}$
and converting that to decimal fractions, not those nasty \emph{vulgar}\footnote{Fractions in the format $\frac{a}{b}$in the UK at least, are called
vulgar fractions.} ones, we get $0.5+0.25+0.125+0.0625+0.003,906,25+0.000,488,281$,
giving a mantissa of $0.941,894,531$.

We can now work out the value as $0.941,894,531*2^{17}$which, funnily
enough, is $123,456$, $2^{17}$ being $131,072$.

How easy was that then? 
