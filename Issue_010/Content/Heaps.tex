
\chapter{Heaps}

\textbf{Note}: Unless otherwise noted in the text, code listings in
this chapter are fragments only, they do not make up a full, working
application.

There are two kinds of heap in the QL:
\begin{itemize}
\item The \emph{Common Heap.}
\item \emph{User Heap(s).}
\end{itemize}
What's the difference? Well, details are below but in summary:
\begin{itemize}
\item Allocation requests in the common heap will have a 16 byte overhead
added to the space requested, and the amount requested will be rounded
up to a multiple of 16. However, if this leaves a 16 byte gap, then
the value will be rounded up to a multiple of 32 instead. \textbf{Warning}:
This is an implementation detail and you should not rely on the rounding
being 16 or 32. Check \texttt{D1} on return and make sure you got
what you requested\footnote{Thanks Marcel.};
\item The value returned in \texttt{D1}, the amount of space allocated,
\emph{includes} the 16 byte overhead plus the rounding;
\item The 16 byte overhead is prior to the address returned in \texttt{A0}
for the base of the space allocated;
\item When deallocating common heap, you don't need to remember the size
that was allocated;
\item Common heap addresses are absolute;
\item User heap space requests are rounded to multiples of 8 bytes;
\item There is an 8 byte overhead, but this is \emph{not added} to the size
requested;
\item The overhead is at the two long words pointed to by the address returned
in \texttt{A0} for the base of the area allocated;
\item When deallocating user heap space, you need to remember the size of
the space allocated.
\item User heap addresses are relative to \texttt{A6}.
\end{itemize}

\section{Common Heap}

The common heap is an area of RAM, allocated from the free space area
in the memory map, between the addresses pointed to by \texttt{SYS\_CHPB}
-- the base of the common heap area -- and \texttt{SYS\_FSBB} --
the base of the free memory area -- both relative to \texttt{A6}
of course! The common heap is mainly used by QDOS/SMSQ to hold such
things as channel definition blocks and other bits of working storage
required by various drivers. Individual jobs can also request areas
of space in the common heap for their own use.
\begin{itemize}
\item When a channel is closed the space allocated in the common heap will
be reclaimed automatically; 
\item Likewise, if a driver is (able to be) removed from the system, it's
working storage space will be reclaimed;
\item When a job is removed from the system, perhaps forcibly, then any
areas of common heap owned by the job are also reclaimed by the system.
\end{itemize}
All of the above can also ``voluntarily'' free up any space obtained
in the common heap, when it is done with.

To try to avoid fragmenting the common heap, if is advisable to free
up space in the opposite order from that in which is was allocated.
For example, if a job requests 10 Kb, 5 Kb then 30 Kb from the common
heap, it should, but it's not mandatory, to free the 30 Kb allocation,
then the 5 Kb allocation and finally, the 10 Kb allocation. This doesn't
work all the time -- some other jobs may have allocated space between
those mentioned and freeing will still leave the common heap a little
more fragmented than is desirable.

\textbf{Note}: There are similar problems with heap allocation on
other operating systems, not just the QL. The advice on those systems
is also, deallocate in the opposite order that you allocated.

A job should, really, request a single chunk of common heap and use
that as a user heap, allocating space from the user heap rather than
the common heap, especially if the allocations are small in size\footnote{Based on which definition of ``small'' I wonder?}
as this will help to reduce common heap fragmentation. User heaps
are discussed in the next section.

Fragmentation of the common heap can lead to situations such like
a channel which cannot be opened as there isn't enough \emph{contiguous}
free space to create a channel definition block. 

Note that when allocating common heap space, the amount of RAM allocated
will be rounded up to a multiple of 16 (or, potentially, 32) bytes.
Each chunk of common heap allocated will have a 16 byte overhead on
top of the space requested. For example:
\begin{itemize}
\item The code requests 10 bytes of common heap;
\item The allocation will be 16 bytes after rounding up;
\item The total space allocated from the common heap's free space will be
32 to include the 16 byte overhead;
\item The value returned in \texttt{D1}, the amount of space allocated,
will include the overhead and rounding -- it will be 32.
\end{itemize}
The 16 byte overhead holds the data detailed in Table \ref{tab:Common-Heap-Header}.

\begin{table}[!h]
\begin{centering}
\begin{tabular}{|c|c|>{\raggedright}p{0.6\textwidth}|}
\hline 
\textbf{Offset(A0)} & \textbf{Size} & \textbf{Description}\tabularnewline
\hline 
\hline 
-\$010 & Long & Length of this block.\tabularnewline
\hline 
-\$0C & Long & Either a pointer to the address of the I/O driver code which will
free this block; or a pointer to the next free area in the common
heap; or zero.\tabularnewline
\hline 
-\$08 & Long & ID of the job which owns this area of common heap.\tabularnewline
\hline 
-\$04 & Long & Address of a byte to be set when this area of heap is freed.\tabularnewline
\hline 
\end{tabular}
\par\end{centering}
\caption{Common Heap Header\label{tab:Common-Heap-Header}}
\end{table}

The long word at offset -\$0C will normally be zero for space allocated
by jobs -- not drivers -- as the free space list in the common heap
is maintained by a separate linked list of pointers, based on the
system variable \texttt{SYS\_FSBB}.

Most jobs I've looked at, including many\footnote{Ahem, all!} of
my own, don't bother with \emph{user} heaps -- see next section --
and simply allocate space in the \emph{common} heap as and when required.
Perhaps the authors, myself included, need to think about which heap
is best for the application?

\subsection{Traps}

Note that when allocating common heap space, the amount of RAM allocated
will be rounded up to a multiple of 16 (or 32) bytes. 

There are two traps to manage allocations in the common heap.

\subsubsection{Sms.achp}

This trap call used to be known as \texttt{MT\_ALLOC} in QDOS, but
is now called \texttt{SMS.ACHP} in SMSQ/E. The obligatory table of
parameters can be seen in Table \ref{tab:SMS.ACHP-Parameters}.

\texttt{D3} is interesting, Pennel doesn't mention it, but the SMSQ/E
Manual (version 4.5) says that memory will be allocated in Fast RAM
if \texttt{D3} is zero, and in ST compatible RAM if ``acsi'', on
those machines with ST and Fast RAM.

\begin{table}[!h]
\begin{centering}
\begin{tabular}{|c|>{\raggedright}p{0.5\textwidth}|}
\hline 
\multicolumn{2}{|c|}{\textbf{Calling Parameters}}\tabularnewline
\hline 
\textbf{Register} & \textbf{Usage}\tabularnewline
\hline 
D0.L & SMS.ACHP = \$18\tabularnewline
\hline 
D1.L & Number of bytes required.\tabularnewline
\hline 
D2.L & ID of owning job. -1 indicates the current job.\tabularnewline
\hline 
D3.L & Zero, or ``acsi'' for Atari TT and machines with ST and Fast RAM. \tabularnewline
\hline 
\multicolumn{2}{|c|}{\textbf{Return Parameters}}\tabularnewline
\hline 
\textbf{Register} & \textbf{Usage}\tabularnewline
\hline 
D0.L & Error code, or zero for no errors.\tabularnewline
\hline 
D1.L & Actual number of bytes allocated, including the 16 byte overhead.\tabularnewline
\hline 
D2 & Corrupted.\tabularnewline
\hline 
D3 & Corrupted.\tabularnewline
\hline 
A0.L & Base address of the allocated space. This is the first byte after
the header for the block.\tabularnewline
\hline 
A1 & Corrupted.\tabularnewline
\hline 
A2 & Corrupted.\tabularnewline
\hline 
A3 & Corrupted.\tabularnewline
\hline 
\multicolumn{2}{|c|}{\textbf{Errors in D0}}\tabularnewline
\hline 
ERR.IJOB & Invalid job ID\tabularnewline
\hline 
ERR.IMEM & Out of memory\tabularnewline
\hline 
\end{tabular}
\par\end{centering}
\caption{SMS.ACHP Parameters\label{tab:SMS.ACHP-Parameters}}

\end{table}


\subsubsection{Example of Use}

Listing \ref{lis:Common-Heap-Allocation-Example} shows a small example
of requesting 1 byte of space from the common heap, then returning
it to the heap. I tested the code by running it through QMON2 to check
on the numbers returned in the appropriate registers and confirm the
rounding and so on. I also checked the 16 bytes prior to the base
address in \texttt{A0} and the overhead data can be seen there.

The figures listed are examples from my execution, your figures, should
you attempt this code, will most likely differ,

\begin{lstlisting}[caption={Common Heap Allocation Example},label={lis:Common-Heap-Allocation-Example}]
sms.achp  equ $18               ; Allocate common heap
sms.rchp  equ $19               ; Free common heap 

start
    moveq #sms.achp,d0          ; Trap code
    moveq #1,d1                 ; A single byte
    moveq #-1,d2                ; For this job 
    moveq #0,d3                 ; For SMSQ/E on ST machines
    trap #1                     ; allocate
    tst.l d0                    ; Ok?
    bne.s allocated             ; Yes, free the heap space
    rts                         ; No, return error of S*BASIC

allocated
    moveq #sms.rchp,d0          ; Trap code
    trap #1                     ; Free heap space
    rts                         ; There are no errors, ever!

\end{lstlisting}

When I executed the code in Listing \ref{lis:Common-Heap-Allocation-Example}
and traced it with QMON2, I extracted the following detail:
\begin{itemize}
\item The base of the allocated block of RAM was at address \$1014A0, this
was the address passed back in \texttt{A0.L};
\item \texttt{D1.L} returned the value \$20 -- for 32 bytes allocated in
total;
\item The 16 bytes prior to the base address were:
\begin{itemize}
\item \texttt{-16(A0)} = \$00000020 = the size of the block allocated;
\item -\texttt{12(A0)} = \$00000000 = Pointer to next free section of common
heap, or driver deallocation code;
\item \texttt{-8(A0)} = \$00000000 = Owning job Id;
\item \texttt{-4(A0)} = \$00000000 = Address of byte to be set when this
block is freed.
\end{itemize}
\end{itemize}
Interesting would you say? I requested a single byte yet I see that
32 bytes were allocated. This confirms that the 16 byte overhead is
\emph{included} in the allocated space value returned in \texttt{D1},
which none of the docs\footnote{Pennel, Dickens and QDOS/SMSQ Reference Manual, 4.5.}
mention. 

I did some other tests with different request sizes and in all cases,
it appears that the rounding is to 16 bytes and not to 8\footnote{I wonder if Pennel says 8 because that's what QDOS did, but SMSQ/E
uses a 16 byte rounding instead?} as indicated in the documentation.

\subsubsection{Sms.rchp}

This trap call used to be known as \texttt{MT\_RECHP} in QDOS, but
is now called \texttt{SMS.RCHP} in SMSQ/E. The obligatory table of
parameters can be seen in Table \ref{tab:SMS.RCHP-Parameters}.

\begin{table}[!h]
\begin{centering}
\begin{tabular}{|c|>{\raggedright}p{0.5\textwidth}|}
\hline 
\multicolumn{2}{|c|}{\textbf{Calling Parameters}}\tabularnewline
\hline 
\textbf{Register} & \textbf{Usage}\tabularnewline
\hline 
D0.L & SMS.RCHP = \$19\tabularnewline
\hline 
A0.L & Base address of the allocated space.\tabularnewline
\hline 
\multicolumn{2}{|c|}{\textbf{Return Parameters}}\tabularnewline
\hline 
\textbf{Register} & \textbf{Usage}\tabularnewline
\hline 
D0.L & Ignore, no errors are returned.\tabularnewline
\hline 
D1L & Corrupted.\tabularnewline
\hline 
D2 & Corrupted.\tabularnewline
\hline 
D3 & Corrupted.\tabularnewline
\hline 
A0L & Corrupted.\tabularnewline
\hline 
A1 & Corrupted.\tabularnewline
\hline 
A2 & Corrupted.\tabularnewline
\hline 
A3 & Corrupted.\tabularnewline
\hline 
\multicolumn{2}{|c|}{\textbf{Errors in D0}}\tabularnewline
\hline 
\multicolumn{2}{|l|}{None -- the trap call never fails.}\tabularnewline
\hline 
\end{tabular}
\par\end{centering}
\caption{SMS.RCHP Parameters\label{tab:SMS.RCHP-Parameters}}
\end{table}

Listing \ref{lis:Common-Heap-Allocation-Example} shows an example
of the use of this trap.

\subsection{Vectors}

There are a pair of vectored routines, \texttt{MEM.ACHP} and \texttt{MEM.RCHP},
which enable code to manipulate space in the common heap. These vectors
are atomic and must be called with the processor running in Supervisor
Mode. They are normally used by device drivers to allocate space in
the common heap for channel definition blocks for the Open function
of the driver. The entire area allocated is zero filled if enough
RAM existed in one contiguous block.

The 16 byte header for the area allocated is not filled in by the
vectored code, it is the responsibility of the device driver code
to do this. 

Other than to mention that they exist, their use from within a device
driver is beyond the scope of this eMagazine, and so they will not
be discussed further.

\section{User Heaps}

If your job requires allocating small chunks of RAM, perhaps for a
linked list, or a tree of structures, then rather than slicing and
dicing the common heap into tiny bits, it is advisable to allocate
a large, single, chunk of common heap and use that as a user heap
to allocate the small chunks. The advantage of this process is that
when done, you simple deallocate the common heap space and free up
a large chunk on one go, rather than having to free up lots of small
chunks. This helps to prevent fragmentation of the common heap.

The steps involved in this process are:
\begin{itemize}
\item Allocate a suitable sized area of common heap to be used for your
user heap;
\item Link the allocated area into your job's user heap space;
\item Allocate space in the user heap, as and when required;
\item Use and abuse the user heap space allocated;
\item Optionally, but good practice, deallocate used space when finished
with;
\item Release the common heap at job end -- this may be manually done,
or left to QDOS/SMSQ to do it automatically.
\end{itemize}
Note that when allocating user heap space, the amount of RAM allocated
will be rounded up to a multiple of 8 bytes. Each chunk of user heap
allocated will have an 8 byte overhead on top of the space requested,
and possibly rounded up. For example:
\begin{itemize}
\item Request 10 bytes of user heap;
\item The allocation will be 16 bytes after rounding up;
\item The total space allocated from the user heap's free space will be
16 as the 8 byte overhead is \emph{not added} to the requested size!
\end{itemize}
The 8 byte overhead holds the data detailed in Table \ref{tab:User-Heap-Header}.

\begin{table}[!h]
\begin{centering}
\begin{tabular}{|c|c|>{\raggedright}p{0.6\textwidth}|}
\hline 
\textbf{Offset} & \textbf{Size} & \textbf{Description}\tabularnewline
\hline 
\hline 
\$00 & Long & Length of this block.\tabularnewline
\hline 
\$04 & Long & A \emph{relative} pointer to the next free space in the user heap
space.\tabularnewline
\hline 
\end{tabular}
\par\end{centering}
\caption{User Heap Header\label{tab:User-Heap-Header}}
\end{table}

The overhead will be written \emph{to the start of the allocated heap
space}, at the address returned in \texttt{A0}, and is considered
part of the user heap allocation. This is different to the overhead
in common heap space.

You will need to save the size of each and every user heap allocation
so that it can be returned to the heap's free space when done with.
You can do this manually -- possibly dangerous if you get the sizes
wrong -- if you only have a couple of allocations, or request 8 bytes
more for each allocation and use the space allocated from address
\texttt{8(A0,A6)} on return, rather than that from \texttt{0(A0,A6)}.
You can also, if applicable, simple rerun the code to create the user
heap space which will free up all allocated chunks in the user heap
in one fell swoop.

\subsection{Traps}

Note that when allocating user heap space, the amount of RAM allocated
will be rounded up to a multiple of 8 bytes. Each allocation will
itself have an 8 byte overhead as discussed in Table \ref{tab:User-Heap-Header},
the 8 bytes is taken out of the space requested and is the first two
long words in the allocated section of the heap. 

In other words, if you need 10 bytes, ask for 18 because the first
8 bytes, at the address returned in \texttt{A0}, will contain the
8 byte overhead. This is useful to keep a hold of as the length of
the block is needed when returning the allocated RAM back to the user
heap free space with \texttt{SMS.REHP}.

All addresses are relative to \texttt{A6} when allocating or deallocating
user heap space.

\subsubsection{Sms.alhp}

Table \ref{tab:SMS.ALHP-Parameters} shows the registers that need
setting up to call the \texttt{SMS.ALHP} trap to allocate memory in
a user heap. The return parameters are interesting too.

\begin{table}[!h]
\begin{centering}
\begin{tabular}{|c|>{\raggedright}p{0.5\textwidth}|}
\hline 
\multicolumn{2}{|c|}{\textbf{Calling Parameters}}\tabularnewline
\hline 
\textbf{Register} & \textbf{Usage}\tabularnewline
\hline 
D0.L & SMS.ALHP = \$0C\tabularnewline
\hline 
D1.L & Number of bytes required. Does not include the 8 byte overhead. Perhaps
ask for 8 extra bytes?\tabularnewline
\hline 
A0.L & Pointer to a pointer to the free space list. Relative to A6.\tabularnewline
\hline 
A6.L & Base address of job.\tabularnewline
\hline 
\multicolumn{2}{|c|}{\textbf{Return Parameters}}\tabularnewline
\hline 
\textbf{Register} & \textbf{Usage}\tabularnewline
\hline 
D0.L & Error code, or zero for no errors.\tabularnewline
\hline 
D1.L & Actual number of bytes allocated -- should be the number requested
and includes the 8 byte overhead.\tabularnewline
\hline 
D2 & Corrupted.\tabularnewline
\hline 
D3 & Corrupted.\tabularnewline
\hline 
A0.L & Base address of the allocated space. This points to the first byte
of the header for the block. Relative to A6.\tabularnewline
\hline 
A1 & Corrupted.\tabularnewline
\hline 
A2 & Corrupted.\tabularnewline
\hline 
A3 & Corrupted.\tabularnewline
\hline 
A6 & Preserved.\tabularnewline
\hline 
\multicolumn{2}{|c|}{\textbf{Errors in D0}}\tabularnewline
\hline 
ERR.IMEM & No area of free space was large enough to allocate.\tabularnewline
\hline 
\end{tabular}
\par\end{centering}
\caption{SMS.ALHP Parameters\label{tab:SMS.ALHP-Parameters}}
\end{table}


\subsubsection{Sms.rehp}

This trap call is used when initially adding an area of RAM to be
used as a user heap, or when freeing an allocation within the user
heap. In the former case, the long word at \texttt{(A6,A1.L)} should
be zero, in the latter, it will be some other, non-zero value. Listings
\ref{lis:User-Heap-Creation-Example} and \ref{lis:User-Heap-Deallocation}
show examples of both.

\begin{table}[!h]
\begin{centering}
\begin{tabular}{|c|>{\raggedright}p{0.5\textwidth}|}
\hline 
\multicolumn{2}{|c|}{\textbf{Calling Parameters}}\tabularnewline
\hline 
\textbf{Register} & \textbf{Usage}\tabularnewline
\hline 
D0.L & SMS.REHP = \$0D\tabularnewline
\hline 
D1.L & Length of space to link (back) into a heap\tabularnewline
\hline 
A0.L & Base address of the space to be linked in/back. Relative to A6.\tabularnewline
\hline 
A1.L & Pointer to a pointer to the free space list. Relative to A6.\tabularnewline
\hline 
A6.L & Base address of job.\tabularnewline
\hline 
\multicolumn{2}{|c|}{\textbf{Return Parameters}}\tabularnewline
\hline 
\textbf{Register} & \textbf{Usage}\tabularnewline
\hline 
D0.L & Ignore, no errors are returned.\tabularnewline
\hline 
D1L & Corrupted.\tabularnewline
\hline 
D2 & Corrupted.\tabularnewline
\hline 
D3 & Corrupted.\tabularnewline
\hline 
A0L & Corrupted.\tabularnewline
\hline 
A1 & Corrupted.\tabularnewline
\hline 
A2 & Corrupted.\tabularnewline
\hline 
A3 & Corrupted.\tabularnewline
\hline 
\multicolumn{2}{|c|}{\textbf{Errors in D0}}\tabularnewline
\hline 
\multicolumn{2}{|l|}{None -- the trap call never fails.}\tabularnewline
\hline 
\end{tabular}
\par\end{centering}
\caption{SMS.RCHP Parameters\label{tab:SMS.RCHP-Parameters-1}}
\end{table}


\subsubsection{User Heap Example}

Listing \ref{lis:User-Heap-Creation-Example} shows a small example
where a chunk of 64 Kb of common heap is requested from the system,
and, if successfully allocated, is converted to a user heap. the code
checks return values from traps -- it's assembled to be \texttt{CALL}ed
from SuperBASIC/SBASIC -- and if any errors occur, the code exits
back to SuperBASIC/SBASIC with the error code.

The allocated area of common heap is then converted to a user heap
by simply making sure that the free space address, pointed to by the
long word at \texttt{myHeap}, is zero, then calling the \texttt{SMS.REHP}
trap.

\begin{lstlisting}[caption={User Heap Creation Example},label={lis:User-Heap-Creation-Example}]
sms.achp  equ $18
sms.rehp  equ $0d

; First, allocate a 64Kb common heap area:
start   
    moveq #sms.achp,d0          ; Trap code
    move.l #65536,d1            ; 64Kb required
    moveq #-1,d2                ; This job will be the owner
    trap #1                     ; Allocate the space
    tst.l d0                    ; Did it work?
    beq.s heapOk                ; Yes

; Handle out of memory errors here.
    rts                         ; Back to SuperBasic
        
; We have a common heap, convert it to a user heap :
heapOk  
    moveq #sms.rehp,d0          ; Trap code
    suba.l a6,a0                ; We need A0 to be A6 relative
    lea myHeap,a1               ; Pointer to heap header
    move.l 0,(a1)               ; Indicate this is a new heap
    suba.l a6,a1                ; This needs to be relative A6
    trap #1                     ; That should do it (No errors)
    bra.s useHeap               ; Go and use the heap space

myHeap
    ds.l 1                      ; Pointer to free space

\end{lstlisting}

Once we have an area of RAM set aside as a user heap, we can begin
to use it. Listing \ref{lis:User-Heap-Allocation} is an example of
allocating 200 bytes from the newly created user heap.

\begin{lstlisting}[caption={User Heap Allocation},label={lis:User-Heap-Allocation}]
sms.alhp  equ $0c

; Allocate space in the user heap.
useHeap
    moveq #sms.alhp,d0          ; Trap code
    move.l #200,d1              ; I need 200 bytes of user heap
    lea myHeap,a0               ; MyHeap = Free space pointer
    suba.l a6,a0                ; Relative to A6        
    trap #1                     ; Allocate 200 bytes
    tst.l d0                    ; Did it work?
    bra.s doStuff               ; Go and use the allocation

; Handle out of user heap memory errors here
    rts                         ; Back to SuperBasic
        
; Now we have allocated some user heap, use it somehow
doStuff 
    adda.l a6,a0                ; Absolute the address

\end{lstlisting}

At this point, running via QMON2, I checked the allocated user heap
space to see if the 8 byte overhead was prior to the address returned
in \texttt{A0}, as per the common heap; or at the returned address.
The overhead is indeed at the address pointed to by \texttt{(A6,A0.L)}
on return and after ``unrelativing'' the address, \texttt{0(A0)}
holds the long word \$000000C8 which is the length of the block, the
long word at \texttt{4(A0)} is zero.

When we have finished using the 200 bytes, we can return it to the
user heap, in case we need more space at some other point in the code.
This uses the same trap call which created the user heap in the first
place, but this time, the pointer to the free space, \texttt{myHeap},
will not be zero as it points to the first free chunk of user heap.

\begin{lstlisting}[caption={User Heap Deallocation},label={lis:User-Heap-Deallocation}]
freeUser  
    moveq #sms.rehp,d0          ; Trap code
    move.l #200,d1              ; Size is 200 bytes
    suba.l a6,a0                ; A0 has to be relative a6
    lea myHeap,a1               ; Pointer to top of heap
    suba.l a6,a1                ; Which has also to be relative A6
    trap #1                     ; Deallocate the 200 byte area
    rts                         ; Back to SuperBasic

\end{lstlisting}

You will note that I have hard coded the block size in register \texttt{D1}
for this trap call. This is one way to do it especially if you only
required a couple of chunks of user heap, keeping a note of the sizes
isn't difficult in that case. However, if you are allocating lots
of user heap space, or chunks of many different sizes, what to do?

My advice would be, allocate space for 8 bytes plus what your code
needs, and use the data from \texttt{8(A6,A0.L)} onwards, and do not
touch anything below that address. When you are done with the space
and about to free it, simply load \texttt{D1} from \texttt{0(A6,A0.L)}
to get the block size, and \emph{Robert is your mother's brother}.
Obviously, if you have ``unrelatived'' the base address of the allocated
space, you would use the data from \texttt{8(A0)} and load \texttt{D1}
from \texttt{(A0)} prior to freeing the space again.

Once all the user heap space has been freed up and is no longer required,
the chances are that your application is about to exit. At this point,
it could exit and automatically free up the 64 Kb chunk of common
heap, without any further work on the code's part, or, the code could
be nice and free it's own allocation with the \texttt{SMS.RCHP} trap
call.

\subsubsection{Relative Addresses}

In the listings above, you will note that I add or subtract \texttt{A6}
from \texttt{A0} and \texttt{A1} at various places in the code. This
is because when using these traps to manipulate user heap space, those
registers have to be relative to \texttt{A6}. It's a bit of a faff
and there are a couple of ways around this problem:
\begin{itemize}
\item Do as I have done, and add or subtract \texttt{A6} as necessary, then
address user heap areas using offsets on \texttt{(A0)} as required.
\item Do all your addressing as offsets on \texttt{(A6,A0.L)} as necessary,
although this addressing mode takes 2 extra clock cycles over just
\texttt{(A0)}\footnote{True, but each and every \texttt{ADD} or \texttt{SUB} of \texttt{A6}
to/from \texttt{A0} or \texttt{A1} will cost you 6 clock cycles, so
there!};
\item Zero \texttt{A6} at the start of the code, and then the addresses
will be both absolute and relative at the same time, so you can use
offsets on \texttt{(A0)} or \texttt{(A6,A0.L)} as you prefer;
\item Use the appropriate vector calls rather than the trap calls, those
use absolute addresses. Speaking of which....
\end{itemize}

\subsection{Vectors}

There are a pair of vectored routines, \texttt{MEM.ALHP} and \texttt{MEM.REHP},
which are \emph{non-atomic}\footnote{In other words, the operation could get interrupted and the scheduler
entered.} versions of the user heap trap calls. They take exactly the same
parameters as the two trap calls, but do not require \texttt{A6} to
be considered. Even better, there is no need to mess around keeping
everything relative to \texttt{A6} as the two vectors don't care about
such necessities!

Converting the listings above to use vectors instead of traps, gives
us the code in Listing \ref{lis:User-Heap-Vectors-Example}.

\begin{lstlisting}[caption={User Heap Vectors Example},label={lis:User-Heap-Vectors-Example}]
sms.achp  equ $18
mem.rehp  equ $DA

; First, allocate a 64Kb common heap area:
start   
    moveq #sms.achp,d0          ; Trap code
    move.l #65536,d1            ; 64Kb required
    moveq #-1,d2                ; This job will be the owner
    trap #1                     ; Allocate the space
    tst.l d0                    ; Did it work?
    beq.s heapOk                ; Yes

; Handle out of memory errors here.
    rts                         ; Back to SuperBasic
        
; We have a common heap, convert it to a user heap :
heapOk
	move.w mem.rehp,a2          ; Vector 
    lea myHeap,a1               ; Pointer to pointer to free space
    clr.l (a1)                  ; Initialise user heap
    move.l #65536,d1            ; We have 64 Kb to play with
    jsr (a2)                    ; Convert to a user heap
    tst.l d0                    ; Did it work
    beq.s useHeap               ; Yes
    rts                         ; No

myHeap
	ds.l 1                      ; Pointer to free space

mem.alhp  equ $D8

; Allocate space in the user heap.
useHeap
    move.l #200+8,d1            ; I need 200 bytes of user heap
    lea myHeap,a0               ; MyHeap = Free space pointer
    move.w mem.alhp,a2          ; Vector
    jsr (a2)                    ; Get some user heap
    tst.l d0                    ; Did it work?
    bra.s doStuff               ; Yes
    rts                         ; No
        
; Now we have allocated some user heap, use it somehow
doStuff 
   move.l #$12345678,8(a0)      ; Avoid writing the header bytes
   ...

; Now deallocate the user heap space.
freeUser  
    lea myHeap,a1               ; Pointer to pointer to free space
	move.w mem.rehp,a2          ; Vector 
    move.l (a0),d1              ; Block length to free up
    jsr (a2)                    ; Free the space
    rts                         ; Back to SuperBASIC

\end{lstlisting}

If you trace the code using QMON2\footnote{Other monitors are available....}
then you will see that when you arrive at the label \texttt{doStuff},
the base address of the user heap allocated at \texttt{(A0)}, holds
the length of the block, which is 208 or \$D0\footnote{And that value confused me as QMON listed the instruction to load
\texttt{D1} as \texttt{MOVE.L \#\$D0,D1} and I was initially confused
as I didn't have an instruction to move \texttt{D0} into \texttt{D1}.
Then I read the screen a little bit better and understood!}.

\subsubsection{Freeing User Heap Space Quickly}

If you have, for example, some deeply recursive code which allocates
space in the user heap, how do you cope with an error whereby you
have to free up all the allocated bits that were ok until the problem
arrived? I'm thinking perhaps of an expression evaluator as a specific
example, but it could be a parser or a compiler building a symbol
table or parse tree etc. I'm also thinking of the problem where the
code doesn't just give up, but informs the user -- \emph{expression
too complex} or \emph{invalid operation} etc -- but then loops back
to the prompt for more input.

If, in the case of the expression evaluator, the code will have allocated
lots of chunks of user heap to build the expression tree\footnote{Usually an Abstract Syntax Tree or AST.}
then those nodes in the tree need to be deallocated before the next
expression can be evaluated, otherwise, at some point, the user heap
space will be full of nodes that are no longer required, but are hogging
all the space.

The easiest manner of deallocating all allocated space in the user
heap is simply to clear the pointer to the free space to zero, then
call the \texttt{SMS.REHP} trap or the \texttt{MEM.REHP} vector and
link the entire user heap to free space again. Something similar to
Listing \ref{lis:User-Heap-Total-Deallocation}.

\begin{lstlisting}[caption={User Heap Total Deallocation},label={lis:User-Heap-Total-Deallocation}]
start
    bsr getCommonHeap           ; Allocate heap space or die
    bra.s mainLoop              ; Skip to main loop

userHeap 
    dc.l 1

; On first entry:
; Link the allocated space in the common heap into a user heap.
;
; On subsequent entries:
; Wipe everything from the user heap.
mainLoop
    lea myHeap,a1               ; Pointer to pointer to free space
    clr.l (a1)                  ; Initialise user heap
    move.w mem.rehp,a2          ; Vector 
    move.l #heapSize,d1         ; Size of user heap
    jsr (a2)                    ; Convert/wipe user heap
    tst.l d0                    ; Did it work
    beq.s useHeap               ; Yes
    rts                         ; No, exit with error code
;
; None of these will return if an error occurs. The errors
; will be handled below, the stack unwound and any user
; heap allocations freed.
    bsr getUserInput            ; Get next expression or exit
    bsr lexer                   ; Build token list
    bsr parser                  ; Build AST
    bsr evaluate                ; Evaluate the expression
    bra.s mainLoop              ; Keep going

lexError
    bsr doLexError              ; Handle lexer errors
    bra.s mainLoop              ; And go again

parseError
    bsr doParseError            ; Handle parser errors
    bra.s mainLoop              ; And go again

EvalError
    bsr overflow                ; Check/Handle overflow
    bsr divZero                 ; Check/handle divide by zero
    bsr ...                     ; Etc
    bra.s mainLoop              ; And go again

    ...

\end{lstlisting}

On the first entry to \texttt{mainLoop}, the common heap allocation
is linked into the user heap's free space, the whole allocation is
free for use as a user heap. On subsequent passes through \texttt{mainLoop},
the user heap is effectively reinitialised, thus freeing up every
piece of allocated space which was allocated before the code went
into error recovery.

Not shown in the example code above is the handling of the \texttt{A7}
stack, which needs to be preserved at the start of the \texttt{mainLoop}
and reset after each and every error so that it is correctly set each
time we pass by the \texttt{mainLoop} address.

Can you tell I'm writing an expression evaluator then? 
